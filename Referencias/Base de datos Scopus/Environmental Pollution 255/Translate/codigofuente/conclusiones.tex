\section{Conclusiones}%
\label{sec:conclusiones}

\begin{enumerate}
	\item \item La concentración de desechos plásticos registrada (macro, meso y microplásticos) fue varias veces mayor que los valores reportados anteriormente en la llanura aluvial del río Paraná. Las comparaciones con otros estudios en todo el mundo siguen siendo difíciles, ya que los protocolos metodológicos aún no están estandarizados; sin embargo, sugieren niveles masivos de contaminación en este mega-río de América del Sur.
	\item Los macroplásticos registrados aquí tienen un origen doméstico (bolsas de compras, envoltorios de alimentos, botellas de bebidas y fragmentos de espuma de empaque), lo que sugiere una inadecuada recolección, procesamiento y disposición final de residuos en la región, que lamentablemente es recurrente en el Sur Global. La investigación adicional no debe pasar por alto los macroplásticos en esta región geopolítica, particularmente si se pretenden estimaciones confiables de los desechos plásticos globales que ingresan al océano desde los ríos.
	\item Los microplásticos secundarios (originados por la descomposición de artículos plásticos más grandes) eran más abundantes que los primarios (fabricados como microperlas, cápsulas, gránulos utilizados en la industria). Las microperlas (que se encuentran comúnmente en las regiones industrializadas) estaban ausentes en el río Paraná. Este hallazgo contrasta con estudios realizados en ambientes de agua dulce de países desarrollados, que sugieren una diferencia en los hábitos de consumo y los niveles de industrialización entre sociedades y economías del mundo desarrollado y en desarrollo.
	\item La mayoría de los desechos plásticos registrados proceden de un arroyo urbano altamente contaminando, que atraviesa la ciudad de Paraná. Los ríos urbanos, particularmente es el Sur Global, son vulnerables a diferentes procesos y actividades urbanas que provocan la contaminación y degradación del ecosistema hídrico.
	\item Registramos partículas microplásticas en el tracto digestivo del $100\%$ de los especímenes de \textit{P. lineatus}, la mayoría de ellas fibras. Si bien reconocemos el bajo número de peces recolectados, este hallazgo evidenció que los microplásticos han penetrado en las redes alimentarias acuáticas y nichos ecológicos en el río Paraná, lo que refuerza la necesidad de más estudios.
	      Contrariamente a nuestras expectativas, los elementos macroplásticos o mesoplásticos no servirían como sustitutos para los estudios de microplásticos (y viceversa), lo que sugiere que todos los tamaños de desechos plásticos deben considerarse en estudios posteriores.
\end{enumerate}
