\begin{table}[H]
	\centering
	\footnotesize\sf
	\begin{tabular}{| g | Sc | Sc |} \hline
		\rowcolor[gray]{.8}                                                                     & \scriptsize \thead{Resumen                                                                                                                                                   \\Narrativo} & \scriptsize \thead{Indicadores Objetivamente\\Verificables} \\ \hline

		\multirow{2}{*}[1.2mm]{\cellcolor[gray]{.8} \scriptsize \thead{Propósito General(Fin)}} & \tiny \multirow{2}{*}{\makecell{Reducir la contaminación presente en las cuencas de la ciudad de                                                                             \\Popayán, aumentando el índice de actividades de limpieza y el uso de\\tecnologías adecuadas con el fin de obtener un mejoramiento del $50\%$\\para el $2023$.}} & \cellcolor[gray]{.8} \scriptsize \textbf{Indicadores de Impacto} \\ \cline{3-3}
		\cellcolor[gray]{.8}                                                                    &                                                                                                       & \tiny \makecell{Disminución de la contaminación hídrica en la ciudad \\ y con ello una disminución considerable de los malos olores,\\ color de aguas negras y presencia de plagas en las\\ zonas colindantes.} \\ \hline

		\multirow{2}{*}[1.2mm]{\cellcolor[gray]{.8} \scriptsize \thead{Objetivo del Proyecto}}  & \tiny \multirow{2}{*}{\makecell{Proponer el uso de la tecnología para el monitoreo y el control de la                                                                        \\contaminación en las fuentes hídricas que circulan por la ciudad.}} & \cellcolor[gray]{.8} \scriptsize \textbf{Indicadores de Efectividad} \\ \cline{3-3}
		\cellcolor[gray]{.8}                                                                    &                                                                                                       & \tiny \makecell{Metodos de evealuación que permita comparar          \\los estados antes y después de haber\\introducido el sistema, y observar que cambios\\se han producido en un determinado tiempo\\(lapso de tiempo corto).} \\ \hline

		\multirow{2}{*}[1.2mm]{\cellcolor[gray]{.8} \scriptsize \thead{Resultado (Productos)}}  & \tiny \multirow{2}{*}{\makecell{Se reducirán los desechos en los ríos y se fortalecerán las zonas más                                                                        \\débiles, como también extraer las basuras del fondo de los ríos para\\prevenir el aumento del nivel del agua.}} & \cellcolor[gray]{.8} \scriptsize \textbf{Indicadores de Gestión} \\ \cline{3-3}
		\cellcolor[gray]{.8}                                                                    &                                                                                                       & \tiny \makecell{Campañas informativas acerca del manejo de           \\basuras y para generar conciencia en las\\personas sobre el cuidado del medio ambiente,\\acompañados por resultados de monitores\\realizados recientemente.} \\ \hline

		\multirow{2}{*}[1.2mm]{\cellcolor[gray]{.8} \scriptsize \thead{Actividades}}            & \tiny \multirow{2}{*}{\makecell{Programa ``Base de datos'' tener un visto bueno por alguna empresa                                                                           \\relacionada con el tema de bases de datos, contratar personal capacitado\\con experiencia en la creación de``Base de datos''. Diseñar un sistema\\que se actualice a diario para tener una información numeraria y útil/\\programa ``análisis de datos'' contratar un analista con experiencia en\\``Bases de datos'', implementar una disciplina de trabajo diariamente,\\para sacar datos más precisos y acertados/crear un programa donde se\\pueda visualizar el entorno del ``análisis de datos'' ejercido por un\\personal especializado con experiencias de usuario, para brindar\\un ambiente agradable entre lo proyectado y el usuario.}} & \cellcolor[gray]{.8} \scriptsize \textbf{Insumo/Bienes y servicios requeridos} \\ \cline{3-3}
		\cellcolor[gray]{.8}                                                                    &                                                                                                       & \tiny \makecell{*Un máster en Big Data Analytics de                  \\tiempo completo por 2 años. *Un estudiante con\\doctorado en DBMS por 3 meses. *Un experto en\\desarrollo de software por 4 meses. *Una oficina\\espaciosa ** e computadores. *Conexión a\\internet. *Licencias de software de análisis\\estadístico. *Un servidor de host para la base de\\datos. *Papelería.} \\ \hline
	\end{tabular}
	\caption{Matriz de marco lógico}
	\label{matrizML}
\end{table}
