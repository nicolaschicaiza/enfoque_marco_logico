\section{Problema Identificado}
  Por la ciudad de Popayán pasan varios riachuelos, los cuales atraviesan distintas zonas urbanas que tienen contacto directo con dichos afluentes, por lo tanto, se aprecia que la contaminación de estos mismos se ha visto incrementada debido a la alta tasa de residuos que son arrojadas en ellos de forma directa e indirecta. Además la contaminación proveniente de las aguas residuales tanto industriales, domesticas y urbanas presentan ser amenazas para incrementar los problemas sanitarios y la transparencia del agua.

  Muchos de los riachuelos no están en una situación de total atención por parte de los residentes de la ciudad, estos sufren graves daños como la contaminación caudal generada por desechos o residuos abandonados en las avenidas circundantes. Es muy conocido que en los sectores comerciales cercanos la tasa de acumulación de desechos es bastante alta, por lo que en ocasiones las corrientes de aguas lluvias se mezclan con las sustancias que se producen de los residuos de diferentes materiales, terminando estos siendo parte de las corrientes de aguas pluviales de la ciudad.
  
  Teniendo en cuanta que el apoyo que brinda entes como la Alcaldía, la Gobernación, la CRC, la Secretaria de Salud y el Acueducto de la ciudad, son insuficientes para maximizar la realización de proyectos de limpieza y control de calidad del agua de los riachuelos, se puede identificar cierto descontento por las comunidades que de cierta manera se ven perjudicadas por los malos olores, la preocupación del aumento del nivel del agua, presencia de plagas en la zona, entre otros. 

