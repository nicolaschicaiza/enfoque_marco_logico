\section{Introducción}
\label{sec:introduccion}
La contaminación plástica es una de los grandes retos de la gestión medioambiental en nuestro tiempo. Los desechos plásticos son una combinación de alta persistencia, baja densidad y distribución de tamaño extremadamente amplia. Esto hace que el comportamiento de los desechos plásticos muestre una variedad mucho más amplia que la de la mayoría de los otros materiales, como los sedimentos finos en suspensión \parencite{Wu2018}. Las partículas de plástico causan graves daños a los ecosistemas marinos y de agua dulce \parencite{AnthonyBrowne2011}. Solo en los océanos, el daño económico debido la contaminación plástica se estima en 21 mil millones de euros \parencite{Beaumont2019}. A pesar de un gran esfuerzo científico para abordar este problema en todo el mundo, el estado de nuestro conocimiento es todavía deficiente por diferentes razones. En primer lugar, a pesar de los amplios esfuerzos de investigación que investigan la contaminación plástica en los océanos, se ha prestado mucha menos atención a los sistemas de agua dulce \parencite{Blettler2017}. Sin embargo, este desequilibrio parece estar revirtiéndose en los últimos años (por ejemplo,~\cite{GUNDOGDU2018342};~\cite{Battulga2019};~\cite{VANWIJNEN2019392}). En segundo lugar, la investigación sobre la contaminación plástica del agua dulce se ha llevado a cabo principalmente en países industrializados (el Norte Global;~\cite{Rochman2015};~\cite{Blettler2017}). Esto no es sorprendente debido al sesgo en la producción científica entre el Norte global y el Sur global \parencite{Guterl2012}. Sin embargo, esta disparidad es motivo de preocupación, ya que el aumento de los niveles de población, la rápida urbanización, los asentamientos informales y el aumento de los niveles de consumo han acelerado enormemente la tasa de generación de desechos sólidos en el Sur Global, donde la recolección, el procesamiento y la disposición final de desechos aún es deficiente (~\cite{MINGHUA20091227};~\cite{UnitedNationsHumanSettlementsProgramme2016}).

En tercer lugar, existe un claro predominio de los microplásticos sobre los estudios de microplásticos en entornos de agua dulce en todo el mundo (menos del $20\%$ de los estudios totales en sistemas de agua dulce se han forzado en macroplásticos;~\cite{Blettler2017}). En consecuencia, se requieren con urgencia más estudios de macroplásticos en agua dulce ya que:
\begin{enumerate*}[label = \(\roman*)]
	\item los estudios que estiman la cantidad de plástico exportado de los ríos al océano son limitados debido a la escasez de datos de campo en los ríos (~\cite{Lebreton2017};~\cite{Schmidt2017});
	\item los estudios globales que estiman la cantidad de plástico exportado de los ríos al océano han evidenciado una entrada significativamente mayor ($\>100$ veces) en términos de peso de macroplásticos (en comparación con microplásticos,~\cite{Schmidt2017});
	\item la eliminación de los macroplásticos en los ríos (por ejemplo, utilizando barreras de auge artesanales) es una acción efectiva/bajo costo para evitar que los plásticos lleguen al océano pero, por el contrario, la misma acción sobre los microplásticos es prácticamente imposible.
\end{enumerate*}
Los microplásticos se pueden clasificar según su origen. Los microplásticos primarios se fabrican a propósito para tener ese tamaño (por ejemplo, microperlas utilizadas en cosméticos y productos de cuidado personal, gránulos de resina virgen utilizados en los procesos de fabricación de plástico), mientas que los microplásticos secundarios son el resultado de la descomposición de elementos de plástico más grandes en partículas más pequeñas \parencite{Weinstein2016}. Los estudios indicaron que las plantas de tratamiento de aguas residuales (PTAR) desempeñan un papel importante en la liberación de microplásticos primarios al medio ambiente (~\cite{OU2018317};~\cite{GUNDOGDU2018342}).

En cuarto lugar, los ríos más grandes del mundo (también llamados mega-ríos) se encuentran en países en desarrollo (ver~\cite{LATRUBESSE2008130}). Las grandes descargas, el tamaño de las cuencas y las malas condiciones sanitarias de las personas que viven en estas cuencas, aumentan potencialmente la cantidad de desechos plásticos que fluyen a través de los mega-ríos hacia el océano. Sin embargo, la información sobre la contaminación plástica en los mega-ríos de los países en desarrollo es todavía muy escasa (~\cite{PAZOS201785};~\cite{Blettler2017}), a pesar de que toda la entrada de plástico transportada por los ríos finalmente se libera en los océanos \parencite{MORRITT2014196} o acumulados en los estuarios \parencite{VERMEIREN20167}.

En quinto lugar, la ingestión de microplásticos por los peces y los riesgos asociados para la salud humana siguen siendo importantes lagunas de conocimiento \parencite{SILVACAVALCANTI2017218}, aunque las principales pesquerías continentales se encuentran precisamente en los ríos más contaminados por plásticos \parencite{Lebreton2017} del Sur Global \parencite{Waterland2013}. Lo anterior sugiere una urgente necesidad de enfocar los esfuerzos de monitoreo en los ríos más contaminados, especialmente donde la pesca continental es crucial para el consumo y las economías locales, como es el caso del río Paraná.

Teniendo en cuenta el fundamento expuesto anteriormente, los objetivos de este estudio fueron determinar:
\begin{enumerate*}[label = \( \roman*)]
	\item la cantidad, origen y composición de los desechos plásticos depositados en los sedimentos de un mega-río (río Paraná);
	\item el insumo plástico transportado por un arroyo que se une al río Paraná;
	\item relación cuantitativa entre macro, meso y microplásticos en sedimentos;
	\item ingestión de microplásticos por Prochilodus lineatus, un pez iliófago (que se alimenta de lodo que contiene detritos y organismos asociados)
\end{enumerate*}
