\section{Planteamiento del Problema}
\subsection{Definición del Problema}
% Parte Nicolás
En todo el mundo la contaminación del agua esta presente y las organizaciones mundiales encargadas de la seguridad del medio ambiente se concentra en realizar acciones en su mayoría para la contaminación que padece los océanos del planeta, sin embargo, se ha prestado mucha menos atención a los sistemas de agua dulce~\cite{Blettler2017}. Los ríos urbanos presentan una cierta contaminación que es causa de problemas en las comunidades como lo son la mala gestión de los residuos locales y la falta de conciencia en la sociedad, por lo que esto a provocado que las aguas de los ríos tomen cambios físicos como son las aguas negras y malos olores \cite{Gastezzi-Arias2016}. Aunque la contaminación de los residuos presentes en los ríos genera ciertos cambios en el estado natural del agua, también se conoce casos en los que el crecimiento de árboles, plantas y arbustos dentro del lecho fluvial provoca que este se reduzca hasta generar una clase de pozos donde el agua se represa y su nivel de turbidez se eleva considerablemente \cite{Stepien2019}. A pesar de estos problemas que su solución puede ser ligera, también se encuentran problemas con soluciones muy complejas.

% Parte Andrés
En un estudio que se llevo acabo en un caso particular, como es el del río Daqing (China), en un periodo largo se hizo el debido procedimiento y se encontró con unos factores principales, los cuales eran participes de la contaminación del agua que pasa por esta zona, estos fueron; nitrógeno, fósforo, demanda de oxígeno químico e índice de permanganato \cite{He2021}. Otro factor que influye en esta contaminación de estos ríos reside en los basureros y en los sanitarios, un análisis fisicoquímico o geofísico que tomo el país de Brasil para poder determinar una solución que permita o demuestre el bajo nivel de contaminación, la idea que plantean en esta región es incluir la regulación e inspección obligatoria con relación al post-cierre de los basureros y los rellenos sanitarios \cite{Morita2021}. Otra problemática que va adherente a esta contaminación es el bajo control del alcantarillado, el país de Ecuador implemento un sistema o red de alcantarillado donde su tarea es de recolección y transporte de residuos domiciliarios, sin embargo, debido a que la planta de tratamiento se encuentra en un área ubicada por encima del nivel de inundación, se planteó un punto de bombeo compuesto por dos bombas centrífugas de 20 HP \cite{MERCHAN2021}.

% Parte Brayan
Por otro lado, hay un efecto de consideración ya que se hace mención muchas veces en los artículos en los cuales se busco información y es que la presencia de bacterias es crucial en el estudio de la contaminación del agua debido a los inconvenientes que estos generan en el ambiente, la flora y fauna, la misma salud de las personas que viven cerca de estos mismos~\cite{MendozaCarino2014} y es que esto es producto de los diferentes factores que se identificaron. Debido a lo anterior en muchos países se ha venido estudiando rigurosamente la calidad del agua ya que este problema es común en todas las partes del mundo tanto en países desarrollados como los países en vía de desarrollo, gracias a estos estudios se ha podido identificar numerosas bacterias como coliformes totales, coliformes fecales, enterococos fecales, Pseudomonas, aeruginosa y Salmonella~\cite{}. Todo esto aparece en gran medida por el inadecuado manejo de sus desechos, las actividades antrópicas, el vertimiento de residuos sólidos y líquidos que en el peor de los casos se han visto que son químicos farmacéuticos~\cite{Garcia-Morales2021} y desechos de aguas residuales producto de las actividades humanas~\cite {Ahumada-Santos2014} que se generan en las zonas urbanas aledañas a estos afluentes.
\subsection{Justificación}
% Nicolás
La contaminación hídrica de los ríos y arroyos de la ciudad de Popayán es generada a partir de una serie de problemas que se identifican como sus principales causas. Las instituciones como la Alcaldía, la Secretaria de Salud y la empresa prestadora del servicio de acueducto y alcantarillado de la ciudad, se encargan de realizar actividades de inspección, vigilancia, limpieza y control de riesgos asociados a las condiciones de la calidad del agua y de las cuencas~\cite{SecretariadeSalud2019}. Estas actividades son enfocadas para mantener el agua potable por debajo de los niveles permitidos de la calidad del agua (Índice de Riesgo de la Calidad del Agua -- IRCA) y son ejecutadas en los tramos antes del proceso de captación del agua. Sin embargo, estas actividades a pesar de cumplir con su propósito se presencia un grave descuido del estado del agua y de las cuencas al transitar estas por la ciudad. Inicialmente se identifica el tipo de contaminantes que alberga estos cuerpos hídricos entre los cuales se conoce la presencia de contaminación por residuos físicos y contaminación de sustancias químicas, que son dañinas para el estado natural de estas aguas~\cite{Aquae}.
