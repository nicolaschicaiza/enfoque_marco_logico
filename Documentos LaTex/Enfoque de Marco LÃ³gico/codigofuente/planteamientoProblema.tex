\section{Planteamiento del Problema}
\subsection{Definición del Problema}
El problema que se comprende en este documento trata principalmente en la contaminación hídrica de los ríos y arroyos de la ciudad de Popayán y sus principales causas. En la ciudad existen instituciones como la Alcaldía, la Secretaria de Salud, empresa prestadora del servicio de acueducto y alcantarillado y la empresa encargada del aseo de la ciudad, las cuales se encargan de realizar actividades de inspección, vigilancia, limpieza y control de riesgos asociados a las condiciones de la calidad del agua y de las cuencas \cite{SecretariadeSalud2019}.

Las actividades que realizan estas instituciones son enfocadas para mantener el agua potable por debajo de los niveles permitidos de la calidad del agua(Indice de Riesgo de la Calidad del Agua -- IRCA) y son ejecutadas en los tramos antes del proceso de captación de agua de las plantas de tratamiento. Sin embargo, estas actividades a pesar de cumplir con su propósito se presencia un grave descuido del estado del agua y de las cuencas al atravesar estas por la ciudad, después de la captación de aguas para las plantas de tratamiento.

Por ello, esta investigación pretende explicar cuáles son las causas de la contaminación presente en los ríos y arroyos de la ciudad después del proceso de captación del agua de las platas de tratamiento. Inicialmente se identifica el tipo de contaminantes que alberga estos cuerpos hídricos entre los cuales se conoce la presencia de contaminación por residuos físicos y contaminación de sustancias químicas, que son dañinas para el estado natural de estas aguas. La contaminación física hace referencia a las basuras  a los residuos que son arrojados al agua de manera directa o que llegan por otras vías, como las aguas residuales, y la contaminación química son los contaminantes que no se aprecian, pero que se encuentran presentes en el agua, como por ejemplo, los pesticidas químicos, bacterias fecales o productos procedentes de la industria \cite{tipocontaminacion}.



Por otro lado en la sección del efecto en el cual se analiza la presencia de bacterias producto de diferentes factores que son causas directas e indirectas en la aparición de dichas bacterias el cual pueden generar malestar en el ambiente y en la salud de las personas produciendo un daño en la población aledaña a estos cauces tanto en animales como en las mismas personas. Se observa también que a nivel internacional se ha venido realizando durante mucho tiempo el estudio del agua en los ríos ya que en países desarrollados es de mucha importancia este recurso natural, que en ocasiones escasea en muchas partes del mundo por lo cual se han venido desarrollando tecnologías para detectar y evaluar la calidad del agua, determinando la presencia de bacterias que habitan en los ríos.

Cuando se habla de evaluar la calidad del agua en los ríos identificando la existencia de organismos bacteriológicos se tiene como primera instancia la necesidad de extraer muestras para un determinado estudio en algunos casos el análisis de dicho recurso se realiza en el mismo lugar pero en otros casos como hicieron los científicos en el departamento del cesar, en Colombia, que tomaron muestras hídricas en frascos para que posteriormente se le realizara un estudio de dicha sustancia en laboratorios especializados para dicha labor, siendo ambas formas válidas para llevar a cabo la identificación de bacterias. Es importante precisar que dichas muestras deben ser refrigeradas y transportadas en un tiempo menor a 6 horas para ser procesadas inmediatamente debido a que algunos organismos pueden asentarse, diluirse, morir o desaparecer por lo cual podrían pasar desapercibidos y así mismo estropear el análisis.

Luego de este proceso se determina que organismos bacteriológicos están presentes en el recurso hídrico a estudiar, se comparan los resultados con otras sustancias haciendo uso también de las normativas que rigen en su sector acerca de la evaluación de la calidad del agua, que para el caso de los científicos del cesar que tomaron y analizaron las muestras de los ríos Manaure y Casacara se rigieron bajo las normativas de Colombia (Resolución 2115 del 2007 y Decretos 1594 del 1984 y 3930 de 2010). Que permite saber que valores de bacterias son en cierta medida aceptados y cuáles son los que ya generan un riesgo para el ambiente y las personas.

Se encontraron concentraciones de coliformes totales, coliformes fecales, enterococos fecales, Pseudomonas, aeruginosa y Salmonella a través del método del sustrato definido, todo esto se debe en gran medida al inadecuado manejo de sus cuencas y las actividades antrópicas como la tala, quema, invasión de riveras, erosión y vertimiento de residuos sólidos y líquidos. Estos ríos en algunos sectores son utilizados como puntos de descargas de desechos sólidos, aguas residuales domésticas y agroindustriales, provocando su degradación y contaminación. Así mismo, propician el desequilibrio de los ecosistemas y del medio ambiente, además de las implicaciones que esto puede tener en la salud de los asentamientos humanos aledañas a lo largo de toda su cuenca.

-Debido a que hay ríos que son fuentes hídricas las cuales suministran agua a sectores poblados surge de ahí la necesidad de establecer modelos de evaluación y gestión integral que garantice su calidad. Entonces es importante tener presente los principales indicadores microbiológicos que se emplean para evaluar el agua potable, esto es clave para poder proponer un nuevo esquema para el monitoreo en Colombia. Los resultados permiten considerar como bioindicadores, además de las bacterias y protozoos establecidos en la norma, algunos agentes microbianos como virus u otras bacterias y parásitos. Por otro lado indican la necesidad de establecer valores de referencia y definir los microorganismos a emplear con base en evaluaciones específicas de la situación microbiana del agua en monitoreos de validación, operación y verificación. Es por ello que el agua que es utilizada para uso doméstico en la población debe estar libre de microorganismos causantes de enfermedades por lo que las posibles consecuencias de la contaminación microbiana para la salud son tales que su control debe ser objetivo primordial y nunca debe comprometerse.

Algunos de los principales factores que provocan el aumento de bacterias, parásitos, virus y hongos en el agua son por los cambios en el ambiente o en la misma población por desarrollar actividades de urbanización no controlada, crecimiento industrial, pobreza y la disposición inadecuada de excretas humanas y animales según la investigación realizada por la Doctora en Ciencias Básicas Biomédicas, Microbióloga de la universidad pontifica bolivariana. Con base en los criterios mencionados los indicadores microbiológicos de  contaminación del agua generalmente han sido bacterias de la flora saprófita intestinal, entre las que se encuentran Bacteroides fragilis, bacterias mesófilas, coliformes totales, y fecales [termotolerantes], Escherichia coli y estreptococos fecales. Algunas de estas, de origen animal [generalmente de explotaciones pecuarias], representan un alto potencial zoonótico, siendo abundantes estreptococos fecales y parásitos como Giardia intestinalis y Cryptosporidium, que tienen una mayor resistencia a los procesos de tratamiento y desinfección del agua para consumo humano

-También se encontró que la contaminación del agua es producto de sustancias químicas tales como los que producen los residuos farmacéuticos, desechos de medicamentos caducos, contaminación en rellenos sanitarios por residuos farmacéuticos que contaminan directa e indirectamente el agua, se determinó en una investigación realizada en la ciudad de México que el agua potable de algunas zonas de su ciudad contenían microcontaminantes de origen farmacéutico en concentraciones relativamente bajas, sustancias tales como estrógenos, ibuprofeno, gemfibrozil, ketoprofeno, ácido salicílico, diclofenaco, di-2-etilhexilftalato (DEHP), butilbencilftalato (BBP), triclosán, bisfenol A (BPA) y 4-nonilfenol (4-NP). Con esto se da a entender que muchas veces las industrias farmacéuticas y los pequeñas farmacias del sector comercial en la ciudad no tienen una gestión adecuada de sus desechos, puede que dichas empresas no están obligadas a eliminar los residuos farmacológicos de manera amigable con el ambiente gracias a la carencia de leyes que promuevan un buen manejo de estos desechos.

Los resultados del estudio realizado en el río Tlalnepantla del Estado de México, México, indicaron que los músculos de C. carpio tenían residuos de diclofenaco, ibuprofeno y naproxeno, a concentraciones de 0.08 a 0.21 ppm, lo que les ocasionó estrés oxidativo y alteraciones genéticas. No hay, hasta este momento, un procedimiento para la correcta eliminación de residuos farmacológicos por parte de las empresas farmacéuticas, hospitales o casas-habitación de nuestro país (Gracia-Vásquez et al. 2014). La presencia de residuos farmacológicos en diferentes reservorios del suelo y acuíferos es un peligro latente para la salud tanto de los seres humanos como de la flora y fauna de todo el planeta cuya presencia en el agua en grandes cantidades pueden causar graves daños en los ecosistemas acuáticos, reduciendo la biodiversidad. Provienen de los vertidos domésticos, agrícolas e industriales, que pueden contener distintos compuestos químicos. En ocasiones, son liberados directamente a la atmósfera e incorporados por la lluvia. Los esfuerzos de varios países no sólo desarrollados sino también aquellos en vías de desarrollo por establecer regulaciones para este tipo de problema como lo son los residuos farmacéuticos y que cada vez la demanda de este tipo de productos es mayor.
\subsection{Justificación}
