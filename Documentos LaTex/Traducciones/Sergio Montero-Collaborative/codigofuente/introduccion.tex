\section{Introducción}
    La degradación de las cuencas hidrográficas se ha convertido en uno de los más importantes problemas ambientales, sociales y económicos en todo el mundo. México no es una excepción. Similar a muchas cuencas hidrográficas en todo el país, la cuenca del río Ayaquila presenta una compleja gama de problemas derivados del cambio de uso de la tierra, incendios forestales, erosión del suelo, contaminación, agotamiento de las aguas subterráneas, disminución de los caudales de arroyos y ríos, y el uso ineficiente del agua para el abastecimiento de agua y el riego urbano. Además, cuando se combina con la creciente demanda de agua debido al crecimiento de la población en los centros urbanos y el aumento de la producción agrícola, la necesidad de perforar más profundo para extraer agua subterránea o transportar agua de largas distancias pueden, a su vez, producir conflictos locales y regionales.

    En México, la actual organización de las instituciones responsables de aspectos de la gestión de los recursos hídricos sobre una base sectorial no se corresponde con la naturaleza multifuncional del agua. Además, dad la escala y complejidad de la degradación de las cuencas hidrográficas, los niveles de gobierno federal y estatal a menudo carecen de la capacidad operativa para abordar este tipo de problema ambiental. Aquí es donde los gobiernos municipales pueden jugar un papel protagónico, porque es precisamente a nivel de cuenca donde el gobierno local responde de manera más directa a las demandas e iniciativas locales. Cuando varios gobiernos municipales enfrentan problemas comunes en relación con el manejo de la tierra, debido a la intersección de procesos ecológicos y socioeconómicos dentro de las cuencas hidrográficas que trascienden los límites administrativos, es de fundamental importancia aumentar la capacidad institucional para el manejo de los recursos hídricos a través de arreglos intermunicipales, dentro del marco de gestión integrada de cuencas hidrográficas y/o gestión integrada de recursos hídricos.(3)

    La gestión integrada de los recursos hídricos (IWRM) ha surgido precisamente en respuesta a la observación de que la infraestructura y la gestión de los recursos hídricos se han desarrollado tradicionalmente para casa sector relacionado con el agua (como el riego, el suministro de agua urbana, la industria) de forma independiente, con poca o ninguna coordinación entre los sectores. Por lo tanto, la IWRM se refiere a la necesidad de considerar el agua de una manera más holística, teniendo en cuenta todos los aspectos del desarrollo, la gestión y el uso de los recursos hídricos, y los efectos de estos entre sí, sociales y beneficios ambientales del uso del agua. (4)

    \setlength{\leftskip}{1cm}
        La Asociación Mundial del Agua define la IWRM como:\\
        ``Un proceso que promueva el desarrollo coordinado y gestión coordinados del agua, la tierra y los recursos relacionados con el fin de maximizar el bienestar económico y social resultante de forma equitativa sin comprometer la sostenibilidad de ecosistemas vitales''.(5)
    \setlength{\leftskip}{0cm}

    De esta manera, el IWRM promueve la integración de la gestión de la tierra y el agua, la consideración y gestión conjunta de todas las fuentes / cuerpos de agua y ambientes acuáticos, y considera en conjunto los diferentes usos y usuarios del agua. Esto, a su vez, requiere un enfoque particular en la dinámica río arriba-río abajo, así como la adopción de límites físicos, temporales y administrativos más extensos que los utilizados en la gestión de proyectos hídricos convencionales: límites de cuencas fluviales en lugar de divisiones políticas; marcos de tiempo a más largo plazo para que coincidan mejor con el ciclo hidrológico y los procesos ecológicos, en lugar de los términos electorales; y estructuras de gobernanza más amplias para abarcar una gama más amplia de actores que incluyen tanto a los usuarios del agua como a los no usuarios.

    El objetivo general de la IWRM  es fortalecer los marcos de gobernanza del agua y, al hacerlo, mejorar el desarrollo, la gestión y el uso del agua. También se pone un fuerte énfasis en la participación pública, especialmente de mujeres y grupos de bajos ingresos. Un marco de gobernanza del agua más integrado no implica necesariamente la necesidad de un ministerio de recursos hídricos centralizado, sino más bien, la capacidad de planificar, gestionar y utilizar el agua en conjunto y en sinergia cuando sea posible, y minimizar los conflictos entre usos y usuarios en competencia.

    Se ha propuesto una gama de herramientas diferentes para ayudar a lograr los objetivos de la IWRM. Estos incluyen diferentes instituciones (por ejemplo, comités de cuencas hidrográficas), regulaciones (por ejemplo, normas de contaminación) y mecanismos (por ejemplo, mercados). Sin embargo, otros aspectos también son importantes, incluida la escala en la que se estructura la toma de decisiones, los marcos de gobernanza y la práctica de implementación. El enfoque aquí no es únicamente si se implementa la IWRM y con qué mecanismos, sino si los mecanismos elegidos se implementan de manera efectiva y compatible con los objetivos de la IWRM. Por ejemplo, es preferible la toma de decisiones a la escala más pequeña apropiada, y la descentralización a menudo se ha implementado para este propósito, pero esto solo será efectivo cuando esté acompañado de recursos financieros adecuados, una sólida capacidad local y un marco de gobernanza más amplio y apropiado. Asimismo, la creación de un comité de cuenca hidrográfica probablemente no conducirá a una mejor gestión de la cuenca si no cuenta con personal capacitado, o si no incluye la participación de todo tipo de actores sociales en la cuenca, corriendo el riesgo de ser monopolizado por más grupos poderosos.
    
    En la práctica, es importante considerar cómo se puede traducir en la práctica este pensamiento internacional actual sobre la gestión de los recursos hídricos, incluida la forma en que se puede financiar de manera sostenible y cómo se pueden medir sus impactos y efectividad. 6) En México, los gobiernos municipales han iniciado y consolidado cambios importantes que han fortalecido su capacidad para formular políticas compatibles con el desarrollo externo. De esta forma, han acometido una reestructuración interna que les ha permitido asumir nuevas responsabilidades como la gestión ambiental, han adoptado nuevos procesos para una organización más eficiente y han desarrollado sus recursos humanos. Todas estas mejoras han propiciado nuevas formas de cooperación con los niveles de gobierno estatal y federal, así como con la población local.(7)
    
    El objetivo de este trabajo es presentar la experiencia de 10 municipios de la parte central de la cuenca del río Ayuquila en occidente de México, que formó una asociación de colaboración para intentar mejorar la calidad de vida de sus ciudadanos y promover la gestión más sostenible del agua y otros recursos naturales dentro y fuera de sus fronteras. Dada la amplia gama de problemas ambientales que enfrenta la cuenca del río Ayuquila, y considerando el papel central de los recursos hídricos, este documento adopta una conceptualización amplia de la IWRM y describe una estrategia que se basa en los principios de que la gobernanza de los recursos hídricos debe asegurar la acceso de todos los ciudadanos de la cuenca al agua potable y adecuada, y que, reconociendo que el agua también puede ser un bien económico, el estado debe regular los mercados basados en el agua para prevenir las inequidades e injusticias generadas por las fuerzas descontroladas del mercado.
    
    A continuación de esta introducción, la Sección II presenta los antecedentes de la región y sus problemas sociales y ambientales, centrándose en la contaminación del río Ayuquila. Estos problemas llevaron a la creación de la Iniciativa Intermunicipal para el Manejo Integrado de la Cuenca del Río Ayuquila, que se describe en la Sección III. A continuación, las secciones IV y V describen las lecciones, los desafíos y las limitaciones, respectivamente, de la iniciativa. La sección VI termina con una sección de conclusión que reflexiona sobre las perspectivas futuras de la iniciativa.
