\section{Objetivos}
Determinar según las investigaciones realizadas y con los datos obtenidos si existe o no, algún grado de contaminación en los ríos circundantes a nuestra ciudad, que pueda afectar directa o indirectamente la salud de las personas y el bienestar de los ecosistemas que ahí se desarrollan. En el caso de que exista algún riesgo considerable, es importante tomar acciones para desarrollar un plan estratégico acorde a los problemas que se presenten en aquella zona para buscar y poner a prueba un sistema que mejore la calidad del agua.
\subsection{Objetivo General}
Buscar el plan más coherente para la descontaminación de los Ríos según las necesidades presentadas en la zona y que son identificadas gracias a los análisis realizados en dichos ríos.
\subsection{Objetivos Específicos}
\begin{itemize}
	\item Aumentar los niveles de limpieza en los cauces.
	\item Ejercer un tipo de inspección obligatoria alrededor de los depósitos de basura y los rellenos sanitarios.
	\item Incentivar una expansión optimizada de los puntos de muestreo y el monitoreo continuo, para tener una mejor verificación de estado de estos ríos.
	\item Reestructurar el sistema de alcantarillado, de manera que genere una mejor calidad de vida para sus habitantes.
	\item Gestionar campañas informativas para dar una mayor conciencia y explicar la situación crítica que se genera a raíz de la contaminación.
	\item Dar un mejor control con los abonos que se utilizan cercano o próximos al rio.
	\item Disminuir el porcentaje de concentración bacteriológica en el agua.
	\item Monitorear la circulación natural de los ríos para evitar estancamientos y posteriores desbordes.
	\item Incrementar señalizaciones que indiquen la importancia del cuidado de estas zonas.
	\item Proliferar planes de gestión de basuras a la población.
	\item Estructurar un grupo encargado del monitoreo regular de dichas zonas.
	\item Reconstruir canales deteriorados por los desbordamientos y demás fuerzas que generan los ríos.
	\item Inspeccionar muestras de agua cada determinado tiempo del año para evaluar la calidad del agua.
	\item Medir el comportamiento que tienen las personas en el cuidado de los arroyos.
	\item Emplear la ley para sancionar a los que contaminen el agua.
	\item Construir un sistema que pueda filtrar los desechos sólidos que contaminan el agua.
	\item Establecer lugares adecuados donde la población aledaña agrupe su basura para evitar que estas paren en los ríos.
\end{itemize}
