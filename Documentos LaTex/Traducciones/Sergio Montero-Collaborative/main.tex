%Preamble
\documentclass[10pt,letterpaper,onecolumn]{article}
\usepackage[spanish,es-tabla]{babel} % idioma: Español, no coloque nombre tablas como cuadro
\usepackage[T1]{fontenc} \usepackage[utf8]{inputenc} % símbolos especiales del idioma
\usepackage{times}
\usepackage[calc,showdow,spanish]{datetime2}
\parindent = 0cm % configuración de sangría
\usepackage[backend=biber,style=ieee]{biblatex}
\usepackage{tabularx} % extra features for tabular environment
%\usepackage{amsmath} \usepackage{amssymb,amsfonts,latexsym,cancel}  % símbolos matemáticos 
\usepackage{array}
%\usepackage{bm}
%\usepackage{epstopdf} % figuras en formato eps a pdf
\usepackage{hyperref} % agregar hiper enlaces dentro del archivo PDF generado
\usepackage{longtable} % habilitar tablas largas
%\setcounter{MaxMatrixCols}{40} % configuración de limite columnas: 40
%\usepackage{multicol} % varias columnas al documento
\usepackage{subfigure} % varias figuras
\usepackage[small,compact]{titlesec} \usepackage{titling} %cambiar el formato del titulo
%\newcolumntype{E}{>{$}c<{$}} % información de tablas en formato matemático
\usepackage{graphicx} % takes care of graphic including machinery
\usepackage{geometry} % configuración del dimensiones de la margen del documento 
\usepackage{booktabs}
%\usepackage{subcaption}
\usepackage{tcolorbox}
\usepackage{fancyhdr} % configuración del formato del documento
\usepackage{authblk}
\usepackage[font=footnotesize]{caption}
\usepackage[toc,page]{appendix}
\usepackage{parskip}
\usepackage{amssymb, amsmath} % Paquetes matemáticos de la American Mathematical Society
\usepackage{float}
\usepackage{setspace}
\usepackage{parskip}
\usepackage{multirow}
\usepackage[all]{xy}
\usepackage{tikz}
\usepackage{tikz}
\usepackage{circuitikz}
\usetikzlibrary{positioning,circuits.ee.IEC}
\usetikzlibrary{matrix}
\usetikzlibrary{calc}
\usetikzlibrary{fit}
\usepackage{fourier}
\usepackage{makecell,cellspace,caption}
\usepackage{color,colortbl}
\usepackage[first=0,last=9]{lcg}
\usepackage{hhline} 
%\usepackage{showframe}
    
%\brokenpenalty=10000 
%\hyphenpenalty=5000 
\raggedbottom
%---------------------------------------------
% configuración formato de fecha
\DTMnewdatestyle{mydateformat}{%
  \renewcommand{\DTMdisplaydate}[4]{%
    %\DTMshortweekdayname{##4},\space% short weekday,
    %\DTMmonthname{##2}\nobreakspace%  (full) Month
    %\number##3,\space%                day,
    %\number##1%                       year
  }%
  \renewcommand{\DTMDisplaydate}{\DTMdisplaydate}%
}
%---------------------------------------------
% configuración de margen
\geometry{
    papersize = {216mm, 279.4mm},
    width = 18cm,
    height = 25cm,
    headsep = 5mm,
    head = 2cm,
    marginpar = 2mm,
    includeall,
}
%---------------------------------------------
% configuración de hiper vínculos
\hypersetup{
	colorlinks=true,       % false: boxed links; true: colored links
	linkcolor=black,        % color of internal links
	citecolor=black,        % color of links to bibliography
	filecolor=magenta,     % color of file links
	urlcolor=black         
}
%---------------------------------------------
% configuración de estilo de la margen
\fancyhf{}
\renewcommand{\headrulewidth}{0pt}
\fancyhead[LO,LE]{
    \includegraphics[scale=0.15]{img/escudo.jpg}
}
\fancyhead[RO,RE]{
    \fontsize{9}{9}
    \textsf{
        Universidad del Cauca\\
        Facultad de Ingeniería Electrónica y Telecomunicaciones\\
        \vspace{-1.3mm}
        \today   
        \DTMsetdatestyle{mydateformat}
        \today
    }
}
\fancyfoot[C]{\thepage}
\pagestyle{fancy}
%---------------------------------------------
\spanishdecimal{.}
\tcbuselibrary{theorems}
%---------------------------------------------
% configuración para sección de archivos e información adicional al documento
\addto\captionsspanish{
    \renewcommand\appendixname{Anexo}
    \renewcommand\appendixpagename{Anexos}
    }
%---------------------------------------------
% algunas configuraciones del cuerpo del documento
\renewcommand{\tablename}{Tabla}
\renewcommand{\baselinestretch}{0.8} % Para indicar el tamaño del entrelineado
%\titleformat{\subsection}[wrap]
%    {\large\normalfont\fontseries{b}\selectfont\filright}
%    {\thesubsection.}{.5em}{}
%\titlespacing{\subsection}
%    {12pc}{1.5ex plus .1ex minus .2ex}{1pc}
%            
%\titleformat{\section}[wrap]
%    {\large\normalfont\fontseries{b}\selectfont\filright}
%    {\thesection.}{.5em}{}
%\titlespacing{\section}
%    {12pc}{1.5ex plus .1ex minus .2ex}{1pc}
\setlength{\parskip}{1.5mm} % Modificar espacio entre párrafos
\renewcommand*{\bibfont}{\footnotesize} % Cambiar tamaño bibliografía
%---------------------------------------------
% configuración de estilo para autores y afiliaciones
\renewcommand*{\Authsep}{, }
\renewcommand*{\Authand}{ y }
\renewcommand*{\Authands}{ y }
\renewcommand*{\Affilfont}{\normalsize}
%\renewcommand*{\Authfont}{\bfseries}    % make author names boldface    
\setlength{\affilsep}{-2mm}   % set the space between author and affiliation
\renewcommand\Authfont{\fontsize{12}{12}\selectfont} % Cambiar tamaño de letra autores
\renewcommand\Affilfont{\fontsize{9}{9}\itshape} % Cambiar tamaño de letra afiliaciones de autores
%---------------------------------------------
% Simbolo fuente AC
\tikzset{circuit declare symbol = ac source}
\tikzset{set ac source graphic = ac source IEC graphic}
\tikzset{
    ac source IEC graphic/.style={
        transform shape,
        circuit symbol lines,
        circuit symbol size = width 3 height 3,
        shape=generic circle IEC,
        /pgf/generic circle IEC/before background={
            \pgfpathmoveto{\pgfpoint{-0.8pt}{0pt}}
            \pgfpathsine{\pgfpoint{0.4pt}{0.4pt}}
            \pgfpathcosine{\pgfpoint{0.4pt}{-0.4pt}}
            \pgfpathsine{\pgfpoint{0.4pt}{-0.4pt}}
            \pgfpathcosine{\pgfpoint{0.4pt}{0.4pt}}
            \pgfusepathqstroke
        }
    }
}
%---------------------------------------------
% Cambiar titulo de keyword a Palabras claves
% Keywords command
\providecommand{\keywords}[1]
{
  \textbf{\textit{Palabras clave: }} #1
}
%---------------------------------------------
% Salto de linea dentro de una celda de tabla
\renewcommand\theadalign{bc}
\renewcommand\theadfont{\bfseries}
\renewcommand\theadgape{\Gape[4pt]}
\renewcommand\cellgape{\Gape[4pt]}
%---------------------------------------------
% Definición de colores
\newcommand{\ra}{\rand0.\arabic{rand}}
\definecolor{Gray}{RGB}{192,192,192}
\definecolor{LightCyan}{rgb}{0.88,1,1}
\newcolumntype{g}{>{\centering \columncolor{Gray}}p{3.5cm}}
%---------------------------------------------
%% Agregar color a celdas de las tablas
%\newcommand*{\arraycolor}[1]{\protect\leavevmode\color{#1}}
%\newcolumntype{A}{>{\columncolor{Gray!42}}Sc}
%%---------------------------------------------
% Limitación de espacios top and bottom en celdas de tablas
\setlength\cellspacetoplimit{3pt}
\setlength\cellspacebottomlimit{3pt}
%---------------------------------------------
% Titulo, autores e información afiliada a los autores
\title{
    \fontsize{16}{16}\selectfont 
    \textbf{Traducción: Collaborative governance for sustainable water resources management: the experience of the Inter-municipal Initiative for the Integrated Management of the Ayuquila River Basin, Mexico
    \vspace{-1mm}}}
\author[1]{Jefry Nicolas Chicaiza Carrasquilla}
\affil[1]{\url{jefryn@unicauca.edu.co}
    \vspace{-2mm}}
\date{}
%---------------------------------------------
% vincular archivo de bibliografías
\bibliography{codigofuente/bibliografia}
%---------------------------------------------
% cuerpo del documento
\begin{document}
%---------------------------------------------
% retornar configuraciones realizadas en preámbulo al cuerpo del documento
\maketitle
\vspace{-5mm}
\thispagestyle{fancy}
%---------------------------------------------
\begin{abstract}
    Este documento describe cómo diez municipios dentro de la Ayuquila cuenta fluvial en México formó una asociación colaborativa para reducir la contaminación de los ríos y, en general, trabajar juntos para mejorar las condiciones de vida y promover una gestión más sostenible de los recursos naturales dentro y a través de su límites administrativos. Describe las mejorar logradas y analiza los acuerdos institucionales para esta asociación, incluyendo la participación de universidades locales e institucionales a nivel estatal y federal y la provisión para participación. También analiza las medidas adoptadas para garantizar la eficacia continua, especialmente garantizando el apoyo de los políticos municipales (que sólo tienen mandatos de tres años), fortaleciendo la capacidad y lo que se necesita para garantizar la eficacia de la asociación en el futuro.
\end{abstract}
\\\\
%---------------------------------------------
\keywords{Gobernanza, México, nivel municipal, contaminación, cuenca fluvial, agua.}
%---------------------------------------------
\section{Introducción}
    La degradación de las cuencas hidrográficas se ha convertido en uno de los más importantes problemas ambientales, sociales y económicos en todo el mundo. México no es una excepción. Similar a muchas cuencas hidrográficas en todo el país, la cuenca del río Ayaquila presenta una compleja gama de problemas derivados del cambio de uso de la tierra, incendios forestales, erosión del suelo, contaminación, agotamiento de las aguas subterráneas, disminución de los caudales de arroyos y ríos, y el uso ineficiente del agua para el abastecimiento de agua y el riego urbano. Además, cuando se combina con la creciente demanda de agua debido al crecimiento de la población en los centros urbanos y el aumento de la producción agrícola, la necesidad de perforar más profundo para extraer agua subterránea o transportar agua de largas distancias pueden, a su vez, producir conflictos locales y regionales.

    En México, la actual organización de las instituciones responsables de aspectos de la gestión de los recursos hídricos sobre una base sectorial no se corresponde con la naturaleza multifuncional del agua. Además, dad la escala y complejidad de la degradación de las cuencas hidrográficas, los niveles de gobierno federal y estatal a menudo carecen de la capacidad operativa para abordar este tipo de problema ambiental. Aquí es donde los gobiernos municipales pueden jugar un papel protagónico, porque es precisamente a nivel de cuenca donde el gobierno local responde de manera más directa a las demandas e iniciativas locales. Cuando varios gobiernos municipales enfrentan problemas comunes en relación con el manejo de la tierra, debido a la intersección de procesos ecológicos y socioeconómicos dentro de las cuencas hidrográficas que trascienden los límites administrativos, es de fundamental importancia aumentar la capacidad institucional para el manejo de los recursos hídricos a través de arreglos intermunicipales, dentro del marco de gestión integrada de cuencas hidrográficas y/o gestión integrada de recursos hídricos.(3)

    La gestión integrada de los recursos hídricos (IWRM) ha surgido precisamente en respuesta a la observación de que la infraestructura y la gestión de los recursos hídricos se han desarrollado tradicionalmente para casa sector relacionado con el agua (como el riego, el suministro de agua urbana, la industria) de forma independiente, con poca o ninguna coordinación entre los sectores. Por lo tanto, la IWRM se refiere a la necesidad de considerar el agua de una manera más holística, teniendo en cuenta todos los aspectos del desarrollo, la gestión y el uso de los recursos hídricos, y los efectos de estos entre sí, sociales y beneficios ambientales del uso del agua. (4)

    \setlength{\leftskip}{1cm}
        La Asociación Mundial del Agua define la IWRM como:\\
        ``Un proceso que promueva el desarrollo coordinado y gestión coordinados del agua, la tierra y los recursos relacionados con el fin de maximizar el bienestar económico y social resultante de forma equitativa sin comprometer la sostenibilidad de ecosistemas vitales''.(5)
    \setlength{\leftskip}{0cm}

    De esta manera, el IWRM promueve la integración de la gestión de la tierra y el agua, la consideración y gestión conjunta de todas las fuentes / cuerpos de agua y ambientes acuáticos, y considera en conjunto los diferentes usos y usuarios del agua. Esto, a su vez, requiere un enfoque particular en la dinámica río arriba-río abajo, así como la adopción de límites físicos, temporales y administrativos más extensos que los utilizados en la gestión de proyectos hídricos convencionales: límites de cuencas fluviales en lugar de divisiones políticas; marcos de tiempo a más largo plazo para que coincidan mejor con el ciclo hidrológico y los procesos ecológicos, en lugar de los términos electorales; y estructuras de gobernanza más amplias para abarcar una gama más amplia de actores que incluyen tanto a los usuarios del agua como a los no usuarios.

    El objetivo general de la IWRM  es fortalecer los marcos de gobernanza del agua y, al hacerlo, mejorar el desarrollo, la gestión y el uso del agua. También se pone un fuerte énfasis en la participación pública, especialmente de mujeres y grupos de bajos ingresos. Un marco de gobernanza del agua más integrado no implica necesariamente la necesidad de un ministerio de recursos hídricos centralizado, sino más bien, la capacidad de planificar, gestionar y utilizar el agua en conjunto y en sinergia cuando sea posible, y minimizar los conflictos entre usos y usuarios en competencia.

    Se ha propuesto una gama de herramientas diferentes para ayudar a lograr los objetivos de la IWRM. Estos incluyen diferentes instituciones (por ejemplo, comités de cuencas hidrográficas), regulaciones (por ejemplo, normas de contaminación) y mecanismos (por ejemplo, mercados). Sin embargo, otros aspectos también son importantes, incluida la escala en la que se estructura la toma de decisiones, los marcos de gobernanza y la práctica de implementación. El enfoque aquí no es únicamente si se implementa la IWRM y con qué mecanismos, sino si los mecanismos elegidos se implementan de manera efectiva y compatible con los objetivos de la IWRM. Por ejemplo, es preferible la toma de decisiones a la escala más pequeña apropiada, y la descentralización a menudo se ha implementado para este propósito, pero esto solo será efectivo cuando esté acompañado de recursos financieros adecuados, una sólida capacidad local y un marco de gobernanza más amplio y apropiado. Asimismo, la creación de un comité de cuenca hidrográfica probablemente no conducirá a una mejor gestión de la cuenca si no cuenta con personal capacitado, o si no incluye la participación de todo tipo de actores sociales en la cuenca, corriendo el riesgo de ser monopolizado por más grupos poderosos.
    
    En la práctica, es importante considerar cómo se puede traducir en la práctica este pensamiento internacional actual sobre la gestión de los recursos hídricos, incluida la forma en que se puede financiar de manera sostenible y cómo se pueden medir sus impactos y efectividad. 6) En México, los gobiernos municipales han iniciado y consolidado cambios importantes que han fortalecido su capacidad para formular políticas compatibles con el desarrollo externo. De esta forma, han acometido una reestructuración interna que les ha permitido asumir nuevas responsabilidades como la gestión ambiental, han adoptado nuevos procesos para una organización más eficiente y han desarrollado sus recursos humanos. Todas estas mejoras han propiciado nuevas formas de cooperación con los niveles de gobierno estatal y federal, así como con la población local.(7)
    
    El objetivo de este trabajo es presentar la experiencia de 10 municipios de la parte central de la cuenca del río Ayuquila en occidente de México, que formó una asociación de colaboración para intentar mejorar la calidad de vida de sus ciudadanos y promover la gestión más sostenible del agua y otros recursos naturales dentro y fuera de sus fronteras. Dada la amplia gama de problemas ambientales que enfrenta la cuenca del río Ayuquila, y considerando el papel central de los recursos hídricos, este documento adopta una conceptualización amplia de la IWRM y describe una estrategia que se basa en los principios de que la gobernanza de los recursos hídricos debe asegurar la acceso de todos los ciudadanos de la cuenca al agua potable y adecuada, y que, reconociendo que el agua también puede ser un bien económico, el estado debe regular los mercados basados en el agua para prevenir las inequidades e injusticias generadas por las fuerzas descontroladas del mercado.
    
    A continuación de esta introducción, la Sección II presenta los antecedentes de la región y sus problemas sociales y ambientales, centrándose en la contaminación del río Ayuquila. Estos problemas llevaron a la creación de la Iniciativa Intermunicipal para el Manejo Integrado de la Cuenca del Río Ayuquila, que se describe en la Sección III. A continuación, las secciones IV y V describen las lecciones, los desafíos y las limitaciones, respectivamente, de la iniciativa. La sección VI termina con una sección de conclusión que reflexiona sobre las perspectivas futuras de la iniciativa.

\section{Antecedentes}
    El río Ayuquila-Armería es uno de los ríos más importantes del occidente de México y su cuenca cubre un área de $9.803$ kilómetros cuadrados en los estados de Jalisco y Colima en el occidente de México (Figura 1). El río Ayuquila nace en la cuenca alta y se fusiona con el río Tuxcacuexco al este para formar el río Armería. Sigue un rumbo sur durante $294$ kilómetros a través de Jalisco y luego Colima antes de descargar en el Océano Pacífico. (8) 

    A nivel nacional, los recursos hídricos dentro de la cuenca del río Ayuquila-Armería tiene alta prioridad, debido a que la cuenca contienen cinco áreas naturales protegidas, una alta diversidad de especies nativas y amenazadas y tres grandes presas que brindan para regar $54,000$ hectáreas de tierras de cultivo de Jalisco y Colima. Por un tramo de $71$ kilómetros, el río Ayuquila forma el límite noreste de la Reserva de la Biosfera Sierra de Manantlán (Figura 2), y también constituye su ecosistema acuático más importante. Esta área protegida forma parte de la red internacional de reservas dentro del programa Hombre y Biosfera de la UNESCO. Desde una perspectiva de conservación, el río Ayuquila-Armería alberga la mayor biodiversidad de colima y la segunda más alta de Jalisco. Contiene $29$ especies de peces, dos de las cuales son endémicas de la región; nueve especies de crustáceos acuáticos, una de las cuales es endémica; y la nutria neotropical (de río)(Lontra longicaudis),(9) que figura en la Lista Roja de Especies Amenazadas de 2006 de la Unión Mundial para la Naturaleza (UICN).(10)

    La enmienda inicial al artículo $115$ de la contitución mexicana en 1983 marcó el primer intento serio de descentralizar el poder al nivel local, al transferir la responsabilidad del agua potable, alcantarillado y tratamiento de aguas residuales, así como la gestión de residuos sólidos, entre otras funciones, a las funciones municipales.(11) Sin embargo, no fue hasta $1992$ que Jalisco delegó la gestión de los servicios de agua, saneamiento y tratamiento de aguas residuales al nivel municipal.

    Al igual que en muchas partes de México, el problema clave con el suministro de agua potable y la eliminación de aguas residuales en la cuenca del río Ayuquila es la prestación de servicios inadecuados. La cobertura de abastecimiento de agua potable en las zonas urbanas de la cuenca está ligeramente por encima del promedio nacional, con un $94\%$, en comparación con un $89\%$, a nivel nacional; mientras que la cobertura rural esa ligeramente por debajo, de solo el $64\%$, en comparación con el $70\%$ a nivel nacional (en la cuenca, el $66\%$ de la población vive en asentamientos urbanos y el resto en áreas rurales). (12) Estadísticamente, por lo tanto, beber la cobertura de abastecimiento de agua parece ser buena en comparación con la situación en México en su conjunto. Sin embargo. en la práctica, estas cigras con engañosas, porque la oferta interna es muy intermitente. En los municipios de la cuenca, es común que el agua corriente solo esté disponible unos pocos días a la semana, en cuto caso la mayoría de los hogares dependen del almacenamiento de agua en tanques elevados. Desafortunadamente, no existe información local detallada y confiable para determinar el verdadero alcance de estas deficiencias en el suministro de agua.

    En relación al saneamiento, según una evaluación realizada por la Comisión de Agua y Saneamiento del Estado de Jalisco (13) en 2001, ni un solo municipio cumplió con los estándates ambientales para la disposición del tratamiento de aguas residuales. En $1996$, las regulaciones federales dictaron los plazos dentro de los cuales los centros urbanos de diferentes tamaños deben instalar plantas de tratamiento. Se fijó un plazo de $2000$ para los asentamientos de más de $50.000$ habitantes. Sin embargo, en la cuenca de Ayuquila, algunas localidades de ese tamaño no cumplieron con los requisitos, como Autlán, que tiene poco más de $50.000$ habitantes pero recién instaló su planta de tratamiento en $2003$; y la ciudad de Villa de Álvarez en el estado de Colima, con una población de más de $250.000$ habitantes, pero que solo iniciará la construcción de su planta en $2007$. Para asentamientos de entre $20.000$ y $50.000$ habitantes, como El Grullo y Unión de Tula, la fecha límite era el $2005$, pero ninguna ha construido una planta de tratamiento de aguas residuales. Los pueblos pequeños con entre $10.000$ y $20.000$ habitantes tienen hasta $2010$ para instalar una planta de tratamiento.

    Sin embargo, la construcción de una planta de tratamiento es solo un paso hacia la mejora de los estándares de tratamiento de aguas residuales. Los pueblos de la cuenca que ya cuentan con algún tipo de sistema de tratamiento de aguas residuales enfrentan problemas que son comunes en todo México: recursos insuficientes para la operación, cobertura limitada de tuberías de alcantarillado y sistemas de recolección, tecnologías inadecuadas, personal no capacitado, falta de control sobre la descarga de aguas residuales municipales, tarifas que no reflejen los costos y que no intenten reutilizar las aguas residuales tratadas. (14)

    A pesar de lo anterior, el control de la contaminación del agua en el río Ayuquila, tanto de efluentes industriales como de aguas residuales urbanas, fue un importante motor que impulsó a los pueblos a considerar la calidad ambiental y tomar acciones para abordar la contaminación del agua. Esto indica que los gobiernos municipales están avanzando gradualmente hacia el abordaje del problema del tratamiento de aguas residuales. Sin embargo, la disponibilidad de información detallada y confiable sobre los recursos hídricos en la región permitiría mejor a los municipios definir acciones a más largo plazo.
    
    Durante las últimas dos décadas, la contaminación del agua ha sido el principal factor que ha contribuido a la degradación del río en la parte central de la cuenca, principalmente por la descarga de efluentes de la industria azucarera local y aguas residuales no tratadas de los grandes centros urbanos de la cuenca. (15) cada año, durante la temporada de cosecha de la caña azúcar, el efluente de la refinería de azúcar Ingenio Melchor Ocampo solía matar una gran cantidad de peces y crustáceos, lo que comprometía las fuentes de alimentos y tenía graves impactos en la salud de la zona ribereña, comunidades de los municipios de Tuxcacuesco, Tolimán y Zapotitlán de Vadillo, que se encuentran entre las más pobres de Jalisco. Además del efluente de la refinería de azúcar, los pueblos de El Grullo y Autlán también solían descargar sus aguas residuales sin tratar al río, lo que afectó de manera similar tanto la biodiversidad del río como las comunidades río abajo.(16)

    Dada la importancia económica de la refinería de azúcar Ingenio Ocampo como principal fuente de empleo de la región, y dado que no soportan los impactos directos de la contaminación que fluía aguas abajo, ni los municipios de Autlán y El Grullo ni los federales y estatales las autoridades gubernamentales respondieron a las quejas de las comunidades campesinas locales sobre la contaminación. Además, la legislación vigente contenía muchas inconsistencias que, en la década de 1980, dificultaron su aplicación para prevenir la contaminación. La situación a lo largo del río Ayuquila fue, por lo tanto, un caso clásico de injusticia ambiental, en el que las comunidades más pobres que vivían aguas abajo de la refinería de azúcar soportaban los costos de la contaminación, pero recibían pocos beneficios económicos de la industria y la agricultura desarrollada aguas arriba. Además, la refinería de azúcar Ingenio Melchor Ocampo se negó a reconocer que era el origen de la contaminación que tan negativamente afectaba a las comunidades río abajo en cada temporada de cosecha.

    En respuesta a la preocupaciones de las comunidades ribereñas que viven junto a la Reserva de la Biosfera Sierra de Manantlán, que fueron afecatadas por la contaminación, en 1989 el instito Manantlán de Econogía y Conservación de la Biodiversidad (IMECBIO) en la Universidad de Guadalajara (17)(también ubicada dentro de la cuenca del Ayuquila), realizó una evaluación ambiental de la situación y elaboró un conjunto de lineamientos para el saneamiento del río.(18) Posteriormente, en 1993, el Gobierno Federal creó la Dirección de la Reserva de la Biosfera sierra de Manantlán (Dirección de la Reserva de la Biosfera Sierra de Manantlán)(DRBSM) para administrar el área natural portegida desde la sede local. Luego, la DRBSM creó nuevas estrcturas institucinales en forma de `juntas de protección del río Ayuquila'' que permitieron a las comunidades ribereñas expresar sus preocupaciones.(19) En estas comunidades, el Ministro Federal de Desarrollo Social, (20) el gobierno del estado de Jalisco, el DRBSM y el además, varios gobiernos municipales colaboraron para desarrollar un proceso de planificación participativa con el fin de definir prioridades de acción para la reducción de la pobreza. Como parte de este proceso, las comunidades identificaron la contaminación del río como el principal obstáculo para el desarrollo local. (21) En el marco de la Ley Nacional de Aguas de 1992, en 1995 se creó el nuevo Ministerio de Medio Ambiente y Recursos Naturales, (22) que absorbió la Comisión Nacional de Agua, (23) la agencia nacional responsable de la gestión del agua, y el antiguo ministerio de Pesca. (24) Esto permitió la creación de un enfoque más integrado para la gestión del agua en el país. Este mismo año, el partido político que ocupaba el cargo en el gobierno del estado Jalisco cambió por primera vez en la historia del estado y se privatizó la refinería de azúcar Ingenio Melchor Ocampo (anteriormente había sido una empresa paraestatal) como resultado de cambios de política económica en el nivel federal. La combinación de estos eventos formó un nuevo contexto institucional, legal y social que facilitó la creación de nuevos canales para abordar el problema de la contaminación de los ríos.(25)

    Al mismo tiempo, se llevó a cabo una campaña pública contra la contaminación de los ríos a través de los medios de comunicación local, estatales y nacionales. IMECBIO desarrolló un programa de investigación para recolectar y documentar evidencias de la contaminación que respaldarían las quejas del campesino

\printbibliography[title={Bibliografía}]
\end{document}
%---------------------------------------------
