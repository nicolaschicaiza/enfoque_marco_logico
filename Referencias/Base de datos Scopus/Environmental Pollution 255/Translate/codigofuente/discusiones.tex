\section{Discusiones}
\label{sec:discusiones}

\subsection{Concentraciones masivas de plástico: cuestiones geopolíticas y sociales}%
Los materiales macroplásticos son la forma más visible de contaminación plástica.~\cite{Blettler2017} reportaron un promedio de 172.5 ítems macroplásticos por transecto de $150 m^{2}$ ($\sim 1.15$ ítems $m^{2}$) en un lago de llanura aluvial del río Paraná, ubicado a solo 18 km de nuestra área de muestreo. En el presente estudio, encontramos casi el doble de esa cantidad: $340.8$ macroplásticos por $150 m^{2}$ ($\sim 2.27 m^{2}$).

Si bien se han realizado varios estudios sobre macroplásticos en la superficie del agua de ríos (~\cite{GASPERI2014163};~\cite{faure2015};~\cite{Baldwin2016};~\cite{LAHENS2018661}) y lagos \parencite{faure2015}, los estudios macroplásticos en sedimentos fluviales aún son escasos, especialmente para las playas. Algunos ejemplos incluyen~\cite{imhof2013contamination} en el lago de Garda (Italia) y~\cite{faure2015}en 6 lagos de Suiza. Sin embargo, la comparación directa con el presente estudio es inviable ya que estos autores consideraron macroplásticos como partículas superiores a $5 mm$ (incluido el tamaño mesoplástico).

La gran cantidad de detritos macroplásticos registrados en la playa Thompson e isla Curupí, así como el origen de los mismos (residuos domiciliarios, Tabla \ref{tab:macroplasticos}), sugieren un deficiente procesamiento de recolección y disposición final d residuos en la ciudad de Paraná. La gestión de residuos es uno de los problemas ambientales clave relacionados con los hidrosistemas urbanos a escala global, sin embargo, en el Sur Global todavía se basa fuertemente en el vertido incontrolado y/o la basura \parencite{GUERRERO2013220}. Como resultado, ocurren serios problemas ambientales \parencite{ALKHATIB20101131} y una creciente contaminación plástica \parencite{Battulga2019}, particularmente en los sistemas de agua dulce. Los municipios de los países de bajos ingresos están gastando una proporción menor de sus presupuestos en la gestión de desechos y, sin embargo, más del $90\%$  de los desechos en los países de bajos ingresos todavía se arrojan abiertamente \parencite{Kaza2018}. Además, el aumento de la población y el aumento de los niveles de consumo han acelerado enormemente la tasa de generación de residuos sólidos en Argentina (tasas de generación de residuos: $1.14 kg$ / cápita / día;~\cite{Kaza2018}). El presente estudio muestra, en parte, esta tendencia global.

La mayoría de los macroplásticos registrados en la presente investigación fueron bolsas de la compra, seguidos de envoltorios de alimentos y envases de espuma (casi el $80\%$; Tabla \ref{tab:macroplasticos}). Las primeras comunidades en adoptar la norma de las bolsas antiplásticas fueron las del Sur Global, mientras que las del Norte Global lo hicieron mucho más recientemente \parencite{Clapp2009}. Sin embargo, no se adoptó una ordenanza municipal contra las bolsas de plástico en la ciudad de Paraná antes de 2017.

Los resultados de los estudios de microplásticos disponibles en sistemas de agua dulce son extremadamente variables de acuerdo con la metodología utilizada (por ejemplo, $m^{2}$, $m^{3}$, $l$, $kg$), medio ambiente (río, lago, embalse, estuario, alcantarillado, etc.), y compartimento de muestreo (superficie o columna de agua, sedimento del fondo o de la playa, etc.). Como resultado, las comparaciones entre estudios mundiales son muy difíciles. Encontramos un promedio de 5239 microplásticos $m^{2}$ (rango de tamaño: $0.35-5 mm$) en los sedimentos del banco del río Paraná, que van desde solo 75 hasta un máximo de $34443$ microplásticos $m^{2}$ (Tabla \ref{tab:microplasticos}).~\cite{doi:10.1139/cjfas-2014-0281} encontraron alrededor de 13832 $m^{2}$ de microperlas de polietileno, retenidas por un tamiz de $0.5mm$, de efluentes industriales en los sedimentos del río San Lorenzo (Canadá).~\cite{Klein2015} tiene un registro de aproximadamente 228-3763 micropartículas $kg^{-1}$ en los sedimentos costeros de los ríos Rin y Main en Alemania (tamaño microplástico: $0.2-5mm$). Además,~\cite{SU2016711}han informado de un rango de $15-1600$ microplásticos $l^{-1}$ ($>0.3 mm$) en el río Yangtze medio-bajo (China),~\cite{WANG20171369} registraron $178-544$ microplásticos $l^{-1}$ ($<5 mm$) en los sedimentos de río Beijiang, y~\cite{PENG2017283} encontraron $410-1600$ microplásticos $kg^{-1}$ ($0.05-5 mm$) en algunos ríos de Shaghai, la mayoría de los fragmentos de dobladillo, esferas y fibras.

~\cite{Blettler2017}, utilizando la misma metodología que el presente estudio, han registrado un promedio mucho más bajo de $704$ microplásticos $m^{-2}$ (rango de tamaño: $0.35-5mm$) en sedimentos de playa de ambientes lénticos del río Paraná (un lago de llanura aluvial ubicado a $18 km$ de distancia del área de muestreo del presente estudio).~\cite{XIONG2018899} reportaron $50-1292$ microplásticos $m^{-2}$ ($>0.1mm$) en el lago Qinghai (China); la mayoría eran películas, fibras y espumas.

A pesar de las limitaciones y debilidades de las comparaciones anteriores (es decir, diferentes rangos de tamaño, unidades, ambientes), la información disponible sugiere una contaminación microplástica significativa presente en los sedimentos del río Paraná.

La variación de la abundancia y el tipo de microplásticos entre los sitios de muestreo fue estadísticamente significativa (Fig. \ref{fig:particulas_microplasticas}), mostrando una clara diferenciación por playa de muestreo. La playa de Thompson mostró la mayor concentración de microplásticos, mientras que Escondida reveló la distribución más heterogénea (las estaciones de muestreo variaron de baja a alta concentración de microplásticos).

El microplástico puede presentarse en forma primaria (perlas) o secundaria (originada por la descomposición de artículos plásticos más grandes;~\cite{COLE20112588}). Aún se desconoce la importancia relativa de las fuentes primarias y secundarias de microplásticos. Encontramos ambos, pero los secundarios fueron considerablemente más abundantes (Tabla \ref{tab:microplasticos})

Se debe prestar especial atención a la ropa sintética, que es una fuente importante de fibras a través del lavado \parencite{Conkle2018}. En nuestro estudio, la fibra fue el único microplástico primario registrado \parencite{COLLE20112588}. Sin embargo, cabe señalar que algunos autores consideran la fibra como secundaria (por ejemplo:~\cite{dris2015microplastic}). Otros microplásticos primarios como microperlas, cápsulas o gránulos (utilizados en cosméticos y productos de cuidado personal, depuradores industriales utilizados para limpieza con chorro abrasivo y gránulos vírgenes utilizados en procesos de fabricación de plástico, respectivamente) estaban ausentes. Se observó una falta similar de microperlas en el río Yangtze \parencite{ZHANG2015117} y el embalse de las Tres Gargantas \parencite{Zhang2017} en China, el río Saigón en Vietnam \parencite{LAHENS2018661} y el Estuario del río Paraná en Argentina \parencite{PAZOS2018134}. No obstante, se observó una gran presencia de microperlas en los ríos Rin y San Lorenzo (~\cite{Mani2015} y~\cite{doi:10.1139/cjfas-2014-0281}, respectivamente) y en los grandes Lagos Laurentinos \parencite{ERIKSEN2013177}. En algunos países que se benefician de instalaciones avanzadas de tratamiento de residuos (principalmente en Europa y América del Norte), las emisiones de microplásticos secundarios son incluso más bajas que las de los microplásticos primarios \parencite{gouin2015}. Las pérdidas de microplásticos primarios pueden ocurrir durante las etapas de producción, transporte o reciclaje de plásticos, o durante la fase de uso de productos que contienen microplásticos (por ejemplo, microperlas originadas de limpiadores faciales ampliamente utilizados en países desarrollados;~\cite{NAPPER2015178};~\cite{gouin2015}). Esto contrasta con los microplásticos secundarios que se originan principalmente a partir de residuos mal gestionados durante la eliminación de productos que contienen plásticos \parencite{Boucher2017}. La ausencia de microperlas en el sistema del río Paraná podría explicarse por estas diferencias en los hábitos de consumo y la gestión de residuos entre sociedades y países. Aquí, casi el $50\%$ de los microplásticos registrados fueron partículas de película (como producto secundario del proceso avanzado de descomposición de bolsas), $33.1\%$ de fibras (utilizadas en textiles) y $18.7\%$ resultantes de partículas más grandes de plástico de origen incierto que se descomponen en elementos más pequeños (probablemente botella de bebida, envoltorio de comida y espumas)(Tabla \ref{tab:microplasticos}). Por el contrario, otros estudios en ríos de países en desarrollo han informado de un predominio de fibras microplásticas (~\cite{ZHANG2015117};~\cite{LAHENS2018661}), incluso en el estuario del río Paraná \parencite{PAZOS2018134}.

Las proporciones variables entre macro o mesoplásticos en nuestro estudi han demostrado que estos datos no pueden servir como sustitutos para el monitoreo de microplásticos (Tabla \ref{tab:correlaciones}). Esto es importante ya que los voluntarios pueden realizar fácilmente estudios de desechos de macroplásticos, que han desempeñado un papel importatne en muchos programas de monitoreo de desexhos \parencite{RIBIC2012994}.

\subsection{Papel de las corrientes urbanas en la difusión del plástico}

Los ríos y arroyos urbanos sufren de múltiples factores estresantes interactivos, especiamente en el Sur Global ()

