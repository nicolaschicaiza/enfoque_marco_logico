\section{Antecedentes}
    El río Ayuquila-Armería es uno de los ríos más importantes del occidente de México y su cuenca cubre un área de $9.803$ kilómetros cuadrados en los estados de Jalisco y Colima en el occidente de México (Figura 1). El río Ayuquila nace en la cuenca alta y se fusiona con el río Tuxcacuexco al este para formar el río Armería. Sigue un rumbo sur durante $294$ kilómetros a través de Jalisco y luego Colima antes de descargar en el Océano Pacífico. (8) 

    A nivel nacional, los recursos hídricos dentro de la cuenca del río Ayuquila-Armería tiene alta prioridad, debido a que la cuenca contienen cinco áreas naturales protegidas, una alta diversidad de especies nativas y amenazadas y tres grandes presas que brindan para regar $54,000$ hectáreas de tierras de cultivo de Jalisco y Colima. Por un tramo de $71$ kilómetros, el río Ayuquila forma el límite noreste de la Reserva de la Biosfera Sierra de Manantlán (Figura 2), y también constituye su ecosistema acuático más importante. Esta área protegida forma parte de la red internacional de reservas dentro del programa Hombre y Biosfera de la UNESCO. Desde una perspectiva de conservación, el río Ayuquila-Armería alberga la mayor biodiversidad de colima y la segunda más alta de Jalisco. Contiene $29$ especies de peces, dos de las cuales son endémicas de la región; nueve especies de crustáceos acuáticos, una de las cuales es endémica; y la nutria neotropical (de río)(Lontra longicaudis),(9) que figura en la Lista Roja de Especies Amenazadas de 2006 de la Unión Mundial para la Naturaleza (UICN).(10)

    La enmienda inicial al artículo $115$ de la contitución mexicana en 1983 marcó el primer intento serio de descentralizar el poder al nivel local, al transferir la responsabilidad del agua potable, alcantarillado y tratamiento de aguas residuales, así como la gestión de residuos sólidos, entre otras funciones, a las funciones municipales.(11) Sin embargo, no fue hasta $1992$ que Jalisco delegó la gestión de los servicios de agua, saneamiento y tratamiento de aguas residuales al nivel municipal.

    Al igual que en muchas partes de México, el problema clave con el suministro de agua potable y la eliminación de aguas residuales en la cuenca del río Ayuquila es la prestación de servicios inadecuados. La cobertura de abastecimiento de agua potable en las zonas urbanas de la cuenca está ligeramente por encima del promedio nacional, con un $94\%$, en comparación con un $89\%$, a nivel nacional; mientras que la cobertura rural esa ligeramente por debajo, de solo el $64\%$, en comparación con el $70\%$ a nivel nacional (en la cuenca, el $66\%$ de la población vive en asentamientos urbanos y el resto en áreas rurales). (12) Estadísticamente, por lo tanto, beber la cobertura de abastecimiento de agua parece ser buena en comparación con la situación en México en su conjunto. Sin embargo. en la práctica, estas cigras con engañosas, porque la oferta interna es muy intermitente. En los municipios de la cuenca, es común que el agua corriente solo esté disponible unos pocos días a la semana, en cuto caso la mayoría de los hogares dependen del almacenamiento de agua en tanques elevados. Desafortunadamente, no existe información local detallada y confiable para determinar el verdadero alcance de estas deficiencias en el suministro de agua.

    En relación al saneamiento, según una evaluación realizada por la Comisión de Agua y Saneamiento del Estado de Jalisco (13) en 2001, ni un solo municipio cumplió con los estándates ambientales para la disposición del tratamiento de aguas residuales. En $1996$, las regulaciones federales dictaron los plazos dentro de los cuales los centros urbanos de diferentes tamaños deben instalar plantas de tratamiento. Se fijó un plazo de $2000$ para los asentamientos de más de $50.000$ habitantes. Sin embargo, en la cuenca de Ayuquila, algunas localidades de ese tamaño no cumplieron con los requisitos, como Autlán, que tiene poco más de $50.000$ habitantes pero recién instaló su planta de tratamiento en $2003$; y la ciudad de Villa de Álvarez en el estado de Colima, con una población de más de $250.000$ habitantes, pero que solo iniciará la construcción de su planta en $2007$. Para asentamientos de entre $20.000$ y $50.000$ habitantes, como El Grullo y Unión de Tula, la fecha límite era el $2005$, pero ninguna ha construido una planta de tratamiento de aguas residuales. Los pueblos pequeños con entre $10.000$ y $20.000$ habitantes tienen hasta $2010$ para instalar una planta de tratamiento.

    Sin embargo, la construcción de una planta de tratamiento es solo un paso hacia la mejora de los estándares de tratamiento de aguas residuales. Los pueblos de la cuenca que ya cuentan con algún tipo de sistema de tratamiento de aguas residuales enfrentan problemas que son comunes en todo México: recursos insuficientes para la operación, cobertura limitada de tuberías de alcantarillado y sistemas de recolección, tecnologías inadecuadas, personal no capacitado, falta de control sobre la descarga de aguas residuales municipales, tarifas que no reflejen los costos y que no intenten reutilizar las aguas residuales tratadas. (14)

    A pesar de lo anterior, el control de la contaminación del agua en el río Ayuquila, tanto de efluentes industriales como de aguas residuales urbanas, fue un importante motor que impulsó a los pueblos a considerar la calidad ambiental y tomar acciones para abordar la contaminación del agua. Esto indica que los gobiernos municipales están avanzando gradualmente hacia el abordaje del problema del tratamiento de aguas residuales. Sin embargo, la disponibilidad de información detallada y confiable sobre los recursos hídricos en la región permitiría mejor a los municipios definir acciones a más largo plazo.
    
    Durante las últimas dos décadas, la contaminación del agua ha sido el principal factor que ha contribuido a la degradación del río en la parte central de la cuenca, principalmente por la descarga de efluentes de la industria azucarera local y aguas residuales no tratadas de los grandes centros urbanos de la cuenca. (15) cada año, durante la temporada de cosecha de la caña azúcar, el efluente de la refinería de azúcar Ingenio Melchor Ocampo solía matar una gran cantidad de peces y crustáceos, lo que comprometía las fuentes de alimentos y tenía graves impactos en la salud de la zona ribereña, comunidades de los municipios de Tuxcacuesco, Tolimán y Zapotitlán de Vadillo, que se encuentran entre las más pobres de Jalisco. Además del efluente de la refinería de azúcar, los pueblos de El Grullo y Autlán también solían descargar sus aguas residuales sin tratar al río, lo que afectó de manera similar tanto la biodiversidad del río como las comunidades río abajo.(16)

    Dada la importancia económica de la refinería de azúcar Ingenio Ocampo como principal fuente de empleo de la región, y dado que no soportan los impactos directos de la contaminación que fluía aguas abajo, ni los municipios de Autlán y El Grullo ni los federales y estatales las autoridades gubernamentales respondieron a las quejas de las comunidades campesinas locales sobre la contaminación. Además, la legislación vigente contenía muchas inconsistencias que, en la década de 1980, dificultaron su aplicación para prevenir la contaminación. La situación a lo largo del río Ayuquila fue, por lo tanto, un caso clásico de injusticia ambiental, en el que las comunidades más pobres que vivían aguas abajo de la refinería de azúcar soportaban los costos de la contaminación, pero recibían pocos beneficios económicos de la industria y la agricultura desarrollada aguas arriba. Además, la refinería de azúcar Ingenio Melchor Ocampo se negó a reconocer que era el origen de la contaminación que tan negativamente afectaba a las comunidades río abajo en cada temporada de cosecha.

    En respuesta a la preocupaciones de las comunidades ribereñas que viven junto a la Reserva de la Biosfera Sierra de Manantlán, que fueron afecatadas por la contaminación, en 1989 el instito Manantlán de Econogía y Conservación de la Biodiversidad (IMECBIO) en la Universidad de Guadalajara (17)(también ubicada dentro de la cuenca del Ayuquila), realizó una evaluación ambiental de la situación y elaboró un conjunto de lineamientos para el saneamiento del río.(18) Posteriormente, en 1993, el Gobierno Federal creó la Dirección de la Reserva de la Biosfera sierra de Manantlán (Dirección de la Reserva de la Biosfera Sierra de Manantlán)(DRBSM) para administrar el área natural portegida desde la sede local. Luego, la DRBSM creó nuevas estrcturas institucinales en forma de `juntas de protección del río Ayuquila'' que permitieron a las comunidades ribereñas expresar sus preocupaciones.(19) En estas comunidades, el Ministro Federal de Desarrollo Social, (20) el gobierno del estado de Jalisco, el DRBSM y el además, varios gobiernos municipales colaboraron para desarrollar un proceso de planificación participativa con el fin de definir prioridades de acción para la reducción de la pobreza. Como parte de este proceso, las comunidades identificaron la contaminación del río como el principal obstáculo para el desarrollo local. (21) En el marco de la Ley Nacional de Aguas de 1992, en 1995 se creó el nuevo Ministerio de Medio Ambiente y Recursos Naturales, (22) que absorbió la Comisión Nacional de Agua, (23) la agencia nacional responsable de la gestión del agua, y el antiguo ministerio de Pesca. (24) Esto permitió la creación de un enfoque más integrado para la gestión del agua en el país. Este mismo año, el partido político que ocupaba el cargo en el gobierno del estado Jalisco cambió por primera vez en la historia del estado y se privatizó la refinería de azúcar Ingenio Melchor Ocampo (anteriormente había sido una empresa paraestatal) como resultado de cambios de política económica en el nivel federal. La combinación de estos eventos formó un nuevo contexto institucional, legal y social que facilitó la creación de nuevos canales para abordar el problema de la contaminación de los ríos.(25)

    Al mismo tiempo, se llevó a cabo una campaña pública contra la contaminación de los ríos a través de los medios de comunicación local, estatales y nacionales. IMECBIO desarrolló un programa de investigación para recolectar y documentar evidencias de la contaminación que respaldarían las quejas del campesino
