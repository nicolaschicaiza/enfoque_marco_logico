\begin{table}[H]
	\centering
	\footnotesize\sf
	\begin{tabular}{| g | Sc | Sc |} \hline
		\rowcolor[gray]{.8}                                                                     & \scriptsize \thead{Resumen                                                                                                                                                                                  \\Narrativo} & \scriptsize \thead{Indicadores Objetivamente\\Verificables} \\ \hline

		\multirow{2}{*}[1.2mm]{\cellcolor[gray]{.8} \scriptsize \thead{Propósito General(Fin)}} & \tiny \multirow{2}{*}{\makecell{Reducir la contaminación presente en las cuencas de la ciudad de                                                                                                            \\Popayán, aumentando el índice de actividades de limpieza y el uso de\\tecnologías adecuadas con el fin de obtener un mejoramiento del $50\%$\\para el $2023$.}} & \cellcolor[gray]{.8} \scriptsize \textbf{Indicadores de Impacto} \\ \cline{3-3}
		\cellcolor[gray]{.8}                                                                    &                                                                                                                    & \tiny \makecell{Disminución de la contaminación hídrica en la ciudad                   \\ y con ello una disminución considerable de los malos olores,\\ color de aguas negras y presencia de plagas en las\\ zonas colindantes.} \\ \hline

		\multirow{2}{*}[1.2mm]{\cellcolor[gray]{.8} \scriptsize \thead{Objetivo del Proyecto}}  & \tiny \multirow{2}{*}{\makecell{Proponer el uso de la tecnología para el monitoreo y el control de la                                                                                                       \\contaminación en las fuentes hídricas que circulan por la ciudad.}} & \cellcolor[gray]{.8} \scriptsize \textbf{Indicadores de Efectividad} \\ \cline{3-3}
		\cellcolor[gray]{.8}                                                                    &                                                                                                                    & \tiny \makecell{Métodos de evaluación que permita comparar                             \\los estados antes y después de haber\\introducido el sistema, y observar que cambios\\se han producido en un determinado tiempo\\(lapso de tiempo corto).} \\ \hline

		\multirow{2}{*}[1.2mm]{\cellcolor[gray]{.8} \scriptsize \thead{Resultado (Productos)}}  & \tiny \multirow{2}{*}{\makecell{Se reducirán los desechos en los ríos y se fortalecerán las zonas más                                                                                                       \\débiles, como también extraer las basuras del fondo de los ríos para\\prevenir el aumento del nivel del agua.}} & \cellcolor[gray]{.8} \scriptsize \textbf{Indicadores de Gestión} \\ \cline{3-3}
		\cellcolor[gray]{.8}                                                                    &                                                                                                                    & \tiny \makecell{Campañas informativas acerca del manejo de                             \\basuras y para generar conciencia en las\\personas sobre el cuidado del medio ambiente,\\acompañados por resultados de monitores\\realizados recientemente.} \\ \hline

		\multirow{2}{*}[1.2mm]{\cellcolor[gray]{.8} \scriptsize \thead{Actividades}}            & \tiny \multirow{2}{*}{\makecell{\textit{(I)} Realizar vigilancia de presencia de basuras y flora en los sedimentos                                                                                          \\ de los ríos para evitar todo tipo de bloqueo del flujo hídrico;\\ \textit{(II)} Canalización con estructura de concreto para proteger los ríos y sus\\ canales de todo tipo de acto contaminante; \textit{(III)} Monitoreo constante de\\ variables importantes para mantener un estado saludable de las\\ aguas de los ríos por medio de muestras en diferentes puntos\\ estratégicos en su paso por la ciudad; \textit{(IV)} Realizar limpiezas de los \\canales de desagüe de las aguas residuales de todo tipo\\ incluyendo las aguas residuales para las lluvias y capacitar a\\ voluntarios para que se realicen en todos los sectores de\\ la ciudad.}}                                                                    & \cellcolor[gray]{.8} \scriptsize \textbf{Insumo/Bienes y servicios requeridos} \\ \cline{3-3}
		\cellcolor[gray]{.8}                                                                    &                                                                                                                    & \tiny \makecell{\textit{(I)} Equipo tecnológico para realizar         monitoreo de las \\ variables importantes que permiten establecer la\\ calidad del agua; \textit{(II)} Aumento de la contratación de\\ expertos en el cuidado ambiental de cuerpos hídricos\\ para realizar capacitaciones a voluntarios; \textit{(III)} Disponer\\ de la maquinaria perteneciente a las instalaciones\\ institucionales para realizar un plan de canalización\\ de los ríos.} \\ \hline
	\end{tabular}
	\caption{Matriz de marco lógico}
	\label{matrizML}
\end{table}
