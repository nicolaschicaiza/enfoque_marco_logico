%Preamble
\documentclass[11pt,letterpaper,onecolumn]{article}
\usepackage[spanish,es-tabla]{babel} % idioma: Español, no coloque nombre tablas como cuadro
\usepackage[T1]{fontenc} \usepackage[utf8]{inputenc} % símbolos especiales del idioma
\usepackage{times}
\usepackage[calc,showdow,spanish]{datetime2}
\parindent = 0cm % configuración de sangría
\usepackage[backend=biber,style=ieee]{biblatex}
\usepackage{tabularx} % extra features for tabular environment
%\usepackage{amsmath} \usepackage{amssymb,amsfonts,latexsym,cancel}  % símbolos matemáticos 
\usepackage{array}
%\usepackage{bm}
%\usepackage{epstopdf} % figuras en formato eps a pdf
\usepackage{hyperref} % agregar hiper enlaces dentro del archivo PDF generado
\usepackage{longtable} % habilitar tablas largas
%\setcounter{MaxMatrixCols}{40} % configuración de limite columnas: 40
%\usepackage{multicol} % varias columnas al documento
\usepackage{subfigure} % varias figuras
\usepackage[small,compact]{titlesec} \usepackage{titling} %cambiar el formato del titulo
%\newcolumntype{E}{>{$}c<{$}} % información de tablas en formato matemático
\usepackage{graphicx} % takes care of graphic including machinery
\usepackage{geometry} % configuración del dimensiones de la margen del documento 
\usepackage{booktabs}
%\usepackage{subcaption}
\usepackage{tcolorbox}
\usepackage{fancyhdr} % configuración del formato del documento
\usepackage{authblk}
\usepackage[font=footnotesize]{caption}
\usepackage[toc,page]{appendix}
\usepackage{parskip}
\usepackage{amssymb, amsmath} % Paquetes matemáticos de la American Mathematical Society
\usepackage{float}
\usepackage{setspace}
\usepackage{parskip}
\usepackage{multirow}
\usepackage[all]{xy}
\usepackage{tikz}
\usepackage{tikz}
\usepackage{circuitikz}
\usetikzlibrary{positioning,circuits.ee.IEC}
\usetikzlibrary{matrix}
\usetikzlibrary{calc}
\usetikzlibrary{fit}
\usepackage{fourier}
\usepackage{makecell,cellspace,caption}
\usepackage{color,colortbl}
\usepackage[first=0,last=9]{lcg}
\usepackage{hhline} 
%\usepackage{showframe}
    
%\brokenpenalty=10000 
%\hyphenpenalty=5000 
\raggedbottom
%---------------------------------------------
% configuración formato de fecha
\DTMnewdatestyle{mydateformat}{%
	\renewcommand{\DTMdisplaydate}[4]{%
		%\DTMshortweekdayname{##4},\space% short weekday,
		%\DTMmonthname{##2}\nobreakspace%  (full) Month
		%\number##3,\space%                day,
		%\number##1%                       year
	}%
	\renewcommand{\DTMDisplaydate}{\DTMdisplaydate}%
}
%---------------------------------------------
% configuración de margen
\geometry{
	papersize = {216mm, 279.4mm},
	width = 18cm,
	height = 25cm,
	headsep = 5mm,
	head = 2cm,
	marginpar = 2mm,
	includeall,
}
%---------------------------------------------
% configuración de hiper vínculos
\hypersetup{
	colorlinks=true,       % false: boxed links; true: colored links
	linkcolor=black,        % color of internal links
	citecolor=black,        % color of links to bibliography
	filecolor=magenta,     % color of file links
	urlcolor=black
}
%---------------------------------------------
% configuración de estilo de la margen
\fancyhf{}
\renewcommand{\headrulewidth}{0pt}
\fancyhead[LO,LE]{
	\includegraphics[scale=0.15]{img/escudo.jpg}
}
\fancyhead[RO,RE]{
	\fontsize{9}{9}
	\textsf{
		Universidad del Cauca\\
		Facultad de Ingeniería Electrónica y Telecomunicaciones\\
		\vspace{-1.3mm}
		\today
		\DTMsetdatestyle{mydateformat}
		\today
	}
}
\fancyfoot[C]{\thepage}
\pagestyle{fancy}
%---------------------------------------------
\spanishdecimal{.}
\tcbuselibrary{theorems}
%---------------------------------------------
% configuración para sección de archivos e información adicional al documento
\addto\captionsspanish{
	\renewcommand\appendixname{Anexo}
	\renewcommand\appendixpagename{Anexos}
}
%---------------------------------------------
% algunas configuraciones del cuerpo del documento
\renewcommand{\tablename}{Tabla}
\renewcommand{\baselinestretch}{0.8} % Para indicar el tamaño del entrelineado
%\titleformat{\subsection}[wrap]
%    {\large\normalfont\fontseries{b}\selectfont\filright}
%    {\thesubsection.}{.5em}{}
%\titlespacing{\subsection}
%    {12pc}{1.5ex plus .1ex minus .2ex}{1pc}
%            
%\titleformat{\section}[wrap]
%    {\large\normalfont\fontseries{b}\selectfont\filright}
%    {\thesection.}{.5em}{}
%\titlespacing{\section}
%    {12pc}{1.5ex plus .1ex minus .2ex}{1pc}
\setlength{\parskip}{1.5mm} % Modificar espacio entre párrafos
\renewcommand*{\bibfont}{\footnotesize} % Cambiar tamaño bibliografía
%---------------------------------------------
% configuración de estilo para autores y afiliaciones
\renewcommand*{\Authsep}{, }
\renewcommand*{\Authand}{ y }
\renewcommand*{\Authands}{ y }
\renewcommand*{\Affilfont}{\normalsize}
%\renewcommand*{\Authfont}{\bfseries}    % make author names boldface    
\setlength{\affilsep}{-2mm}   % set the space between author and affiliation
\renewcommand\Authfont{\fontsize{12}{12}\selectfont} % Cambiar tamaño de letra autores
\renewcommand\Affilfont{\fontsize{9}{9}\itshape} % Cambiar tamaño de letra afiliaciones de autores
%---------------------------------------------
% Simbolo fuente AC
\tikzset{circuit declare symbol = ac source}
\tikzset{set ac source graphic = ac source IEC graphic}
\tikzset{
	ac source IEC graphic/.style={
			transform shape,
			circuit symbol lines,
			circuit symbol size = width 3 height 3,
			shape=generic circle IEC,
			/pgf/generic circle IEC/before background={
					\pgfpathmoveto{\pgfpoint{-0.8pt}{0pt}}
					\pgfpathsine{\pgfpoint{0.4pt}{0.4pt}}
					\pgfpathcosine{\pgfpoint{0.4pt}{-0.4pt}}
					\pgfpathsine{\pgfpoint{0.4pt}{-0.4pt}}
					\pgfpathcosine{\pgfpoint{0.4pt}{0.4pt}}
					\pgfusepathqstroke
				}
		}
}
%---------------------------------------------
% Salto de linea dentro de una celda de tabla
\renewcommand\theadalign{bc}
\renewcommand\theadfont{\bfseries}
\renewcommand\theadgape{\Gape[4pt]}
\renewcommand\cellgape{\Gape[4pt]}
%---------------------------------------------
% Definición de colores
\newcommand{\ra}{\rand0.\arabic{rand}}
\definecolor{Gray}{RGB}{192,192,192}
\definecolor{LightCyan}{rgb}{0.88,1,1}
\newcolumntype{g}{>{\centering \columncolor{Gray}}p{3.5cm}}
%---------------------------------------------
%% Agregar color a celdas de las tablas
%\newcommand*{\arraycolor}[1]{\protect\leavevmode\color{#1}}
%\newcolumntype{A}{>{\columncolor{Gray!42}}Sc}
%%---------------------------------------------
% Limitación de espacios top and bottom en celdas de tablas
\setlength\cellspacetoplimit{3pt}
\setlength\cellspacebottomlimit{3pt}
%---------------------------------------------
% Titulo, autores e información afiliada a los autores
\title{
	\fontsize{22}{22}\selectfont
	\vspace{-9mm}
	\textbf{Enfoque de Marco Lógico
		\vspace{-1mm}}}
\author[1]{Jefry Nicolás Chicaiza}
\author[2]{Gustavo Hernan Paz}
\author[3]{Brayan David Ruiz Dorado}
\author[4]{Andres Felipe Zuñiga}
\affil[1]{\url{jefryn@unicauca.edu.co}
	\vspace{-2mm}}
\affil[2]{\url{ghpaz@unicauca.edu.co}
	\vspace{-2mm}}
\affil[3]{\url{brayanrdorado@unicauca.edu.co}
	\vspace{-2mm}}
\affil[4]{\url{felipezl767@unicauca.edu.co}
	\vspace{-5mm}}
\date{}
%---------------------------------------------
% vincular archivo de bibliografías
\bibliography{codigofuente/bibliografia}
%---------------------------------------------
% cuerpo del documento
\begin{document}
%---------------------------------------------
% retornar configuraciones realizadas en preámbulo al cuerpo del documento
\maketitle
\vspace{-5mm}
\thispagestyle{fancy}
%---------------------------------------------
\section{Problema Identificado}
Por la ciudad de Popayán pasan varios riachuelos, los cuales atraviesan distintas zonas urbanas que tienen contacto directo con dichos afluentes, por lo tanto, se aprecia que la contaminación de estos mismos se ha visto incrementada debido a la alta tasa de residuos que son arrojadas en ellos de forma directa e indirecta. Además la contaminación proveniente de las aguas residuales tanto industriales, domesticas y comerciales resultan ser amenazas que incrementan los problemas sanitarios y la transparencia del agua.

Muchos de los riachuelos no están en total atención por parte de los residentes de la ciudad, estos sufren graves daños como la contaminación generada por desechos o residuos abandonados en las avenidas circundantes. Es muy conocido que en los sectores comerciales cercanos la tasa de acumulación de desechos es bastante alta, por lo que en ocasiones las corrientes de aguas lluvias arrastran sustancias que estos suelen derramar en las calles, terminando siendo parte de las corrientes de aguas pluviales de la ciudad.

Teniendo en cuanta que el apoyo que brinda entes como la Alcaldía, la Gobernación, la CRC, la Secretaria de Salud y el Acueducto de la ciudad, son insuficientes para maximizar la realización de proyectos de limpieza y control de calidad del agua de los riachuelos, se puede identificar cierto descontento por las comunidades que de cierta manera se ven perjudicadas por los malos olores, la preocupación del aumento del nivel del agua, presencia de plagas en la zona, entre otros.

\section{Análisis de Participación}
  % Graphic for TeX using PGF
% Title: /home/jnicolaschc/GitHub/Metodología de la Investigación/enfoque_marco_logico/Esquemas/Árbol de Participación/arbolParticipacion.dia
% Creator: Dia v0.97+git
% CreationDate: Sun Jun 27 19:22:05 2021
% For: jnicolaschc
% \usepackage{tikz}
% The following commands are not supported in PSTricks at present
% We define them conditionally, so when they are implemented,
% this pgf file will use them.
\begin{figure}[H]
    \centering
    \ifx\du\undefined
      \newlength{\du}
    \fi
    \setlength{\du}{15\unitlength}
    \begin{tikzpicture}[even odd rule]
    \pgftransformxscale{1.000000}
    \pgftransformyscale{-1.000000}
    \definecolor{dialinecolor}{rgb}{0.000000, 0.000000, 0.000000}
    \pgfsetstrokecolor{dialinecolor}
    \pgfsetstrokeopacity{1.000000}
    \definecolor{diafillcolor}{rgb}{1.000000, 1.000000, 1.000000}
    \pgfsetfillcolor{diafillcolor}
    \pgfsetfillopacity{1.000000}
    \pgfsetlinewidth{0.000000\du}
    \pgfsetdash{}{0pt}
    \pgfsetroundjoin
    \pgfsetbuttcap
    {\pgfsetcornersarced{\pgfpoint{0.300000\du}{0.300000\du}}\definecolor{diafillcolor}{rgb}{0.172549, 0.533333, 0.850980}
    \pgfsetfillcolor{diafillcolor}
    \pgfsetfillopacity{1.000000}
    \fill (49.907800\du,8.337130\du)--(49.907800\du,9.087734\du)--(52.579713\du,9.087734\du)--(52.579713\du,8.337130\du)--cycle;
    }{\pgfsetcornersarced{\pgfpoint{0.300000\du}{0.300000\du}}\definecolor{dialinecolor}{rgb}{0.172549, 0.533333, 0.850980}
    \pgfsetstrokecolor{dialinecolor}
    \pgfsetstrokeopacity{1.000000}
    \draw (49.907800\du,8.337130\du)--(49.907800\du,9.087734\du)--(52.579713\du,9.087734\du)--(52.579713\du,8.337130\du)--cycle;
    }% setfont left to latex
    \definecolor{dialinecolor}{rgb}{1.000000, 1.000000, 1.000000}
    \pgfsetstrokecolor{dialinecolor}
    \pgfsetstrokeopacity{1.000000}
    \definecolor{diafillcolor}{rgb}{1.000000, 1.000000, 1.000000}
    \pgfsetfillcolor{diafillcolor}
    \pgfsetfillopacity{1.000000}
    \node[anchor=base,inner sep=0pt, outer sep=0pt,color=dialinecolor] at (51.243800\du,8.651505\du){Stakeholders};
    \pgfsetlinewidth{0.000000\du}
    \pgfsetdash{}{0pt}
    \pgfsetroundjoin
    \pgfsetbuttcap
    {\pgfsetcornersarced{\pgfpoint{0.300000\du}{0.300000\du}}\definecolor{diafillcolor}{rgb}{0.450980, 0.058824, 0.764706}
    \pgfsetfillcolor{diafillcolor}
    \pgfsetfillopacity{1.000000}
    \fill (53.735000\du,10.140900\du)--(53.735000\du,10.923575\du)--(56.285803\du,10.923575\du)--(56.285803\du,10.140900\du)--cycle;
    }{\pgfsetcornersarced{\pgfpoint{0.300000\du}{0.300000\du}}\definecolor{dialinecolor}{rgb}{0.450980, 0.058824, 0.764706}
    \pgfsetstrokecolor{dialinecolor}
    \pgfsetstrokeopacity{1.000000}
    \draw (53.735000\du,10.140900\du)--(53.735000\du,10.923575\du)--(56.285803\du,10.923575\du)--(56.285803\du,10.140900\du)--cycle;
    }\pgfsetlinewidth{0.050000\du}
    \pgfsetdash{}{0pt}
    \pgfsetmiterjoin
    \pgfsetbuttcap
    {
    \definecolor{diafillcolor}{rgb}{0.294118, 0.360784, 0.419608}
    \pgfsetfillcolor{diafillcolor}
    \pgfsetfillopacity{1.000000}
    % was here!!!
    \pgfsetarrowsend{stealth}
    {\pgfsetcornersarced{\pgfpoint{0.500000\du}{0.500000\du}}\definecolor{dialinecolor}{rgb}{0.294118, 0.360784, 0.419608}
    \pgfsetstrokecolor{dialinecolor}
    \pgfsetstrokeopacity{1.000000}
    \draw (51.243756\du,9.087734\du)--(51.243756\du,9.614317\du)--(55.010402\du,9.614317\du)--(55.010402\du,10.140900\du);
    }}
    % setfont left to latex
    \definecolor{dialinecolor}{rgb}{1.000000, 1.000000, 1.000000}
    \pgfsetstrokecolor{dialinecolor}
    \pgfsetstrokeopacity{1.000000}
    \definecolor{diafillcolor}{rgb}{1.000000, 1.000000, 1.000000}
    \pgfsetfillcolor{diafillcolor}
    \pgfsetfillopacity{1.000000}
    \node[anchor=base,inner sep=0pt, outer sep=0pt,color=dialinecolor] at (55.010400\du,10.402853\du){Sin participación};
    % setfont left to latex
    \definecolor{dialinecolor}{rgb}{1.000000, 1.000000, 1.000000}
    \pgfsetstrokecolor{dialinecolor}
    \pgfsetstrokeopacity{1.000000}
    \definecolor{diafillcolor}{rgb}{1.000000, 1.000000, 1.000000}
    \pgfsetfillcolor{diafillcolor}
    \pgfsetfillopacity{1.000000}
    \node[anchor=base,inner sep=0pt, outer sep=0pt,color=dialinecolor] at (55.010400\du,10.755631\du){directa};
    \pgfsetlinewidth{0.000000\du}
    \pgfsetdash{}{0pt}
    \pgfsetroundjoin
    \pgfsetbuttcap
    {\pgfsetcornersarced{\pgfpoint{0.300000\du}{0.300000\du}}\definecolor{diafillcolor}{rgb}{0.101961, 0.682353, 0.623529}
    \pgfsetfillcolor{diafillcolor}
    \pgfsetfillopacity{1.000000}
    \fill (46.713200\du,10.140900\du)--(46.713200\du,10.951096\du)--(49.293356\du,10.951096\du)--(49.293356\du,10.140900\du)--cycle;
    }{\pgfsetcornersarced{\pgfpoint{0.300000\du}{0.300000\du}}\definecolor{dialinecolor}{rgb}{0.101961, 0.682353, 0.623529}
    \pgfsetstrokecolor{dialinecolor}
    \pgfsetstrokeopacity{1.000000}
    \draw (46.713200\du,10.140900\du)--(46.713200\du,10.951096\du)--(49.293356\du,10.951096\du)--(49.293356\du,10.140900\du)--cycle;
    }% setfont left to latex
    \definecolor{dialinecolor}{rgb}{1.000000, 1.000000, 1.000000}
    \pgfsetstrokecolor{dialinecolor}
    \pgfsetstrokeopacity{1.000000}
    \definecolor{diafillcolor}{rgb}{1.000000, 1.000000, 1.000000}
    \pgfsetfillcolor{diafillcolor}
    \pgfsetfillopacity{1.000000}
    \node[anchor=base,inner sep=0pt, outer sep=0pt,color=dialinecolor] at (48.003300\du,10.402853\du){Con participación};
    % setfont left to latex
    \definecolor{dialinecolor}{rgb}{1.000000, 1.000000, 1.000000}
    \pgfsetstrokecolor{dialinecolor}
    \pgfsetstrokeopacity{1.000000}
    \definecolor{diafillcolor}{rgb}{1.000000, 1.000000, 1.000000}
    \pgfsetfillcolor{diafillcolor}
    \pgfsetfillopacity{1.000000}
    \node[anchor=base,inner sep=0pt, outer sep=0pt,color=dialinecolor] at (48.003300\du,10.755631\du){directa};
    \pgfsetlinewidth{0.050000\du}
    \pgfsetdash{}{0pt}
    \pgfsetmiterjoin
    \pgfsetbuttcap
    {
    \definecolor{diafillcolor}{rgb}{0.294118, 0.360784, 0.419608}
    \pgfsetfillcolor{diafillcolor}
    \pgfsetfillopacity{1.000000}
    % was here!!!
    \pgfsetarrowsend{stealth}
    {\pgfsetcornersarced{\pgfpoint{0.500000\du}{0.500000\du}}\definecolor{dialinecolor}{rgb}{0.294118, 0.360784, 0.419608}
    \pgfsetstrokecolor{dialinecolor}
    \pgfsetstrokeopacity{1.000000}
    \draw (51.243756\du,9.087734\du)--(51.243756\du,9.614317\du)--(48.003278\du,9.614317\du)--(48.003278\du,10.140900\du);
    }}
    \pgfsetlinewidth{0.050000\du}
    \pgfsetdash{}{0pt}
    \pgfsetmiterjoin
    \pgfsetbuttcap
    {
    \definecolor{diafillcolor}{rgb}{0.294118, 0.360784, 0.419608}
    \pgfsetfillcolor{diafillcolor}
    \pgfsetfillopacity{1.000000}
    % was here!!!
    \pgfsetarrowsend{stealth}
    {\pgfsetcornersarced{\pgfpoint{0.500000\du}{0.500000\du}}\definecolor{dialinecolor}{rgb}{0.294118, 0.360784, 0.419608}
    \pgfsetstrokecolor{dialinecolor}
    \pgfsetstrokeopacity{1.000000}
    \draw (48.003278\du,10.951096\du)--(48.003278\du,11.478898\du)--(49.505825\du,11.478898\du)--(49.505825\du,12.006700\du);
    }}
    \pgfsetlinewidth{0.050000\du}
    \pgfsetdash{}{0pt}
    \pgfsetmiterjoin
    \pgfsetbuttcap
    {
    \definecolor{diafillcolor}{rgb}{0.294118, 0.360784, 0.419608}
    \pgfsetfillcolor{diafillcolor}
    \pgfsetfillopacity{1.000000}
    % was here!!!
    \pgfsetarrowsend{stealth}
    {\pgfsetcornersarced{\pgfpoint{0.500000\du}{0.500000\du}}\definecolor{dialinecolor}{rgb}{0.294118, 0.360784, 0.419608}
    \pgfsetstrokecolor{dialinecolor}
    \pgfsetstrokeopacity{1.000000}
    \draw (48.003278\du,10.951096\du)--(48.003278\du,11.477048\du)--(46.586887\du,11.477048\du)--(46.586887\du,12.003000\du);
    }}
    \pgfsetlinewidth{0.000000\du}
    \pgfsetdash{}{0pt}
    \pgfsetroundjoin
    \pgfsetbuttcap
    {\pgfsetcornersarced{\pgfpoint{0.300000\du}{0.300000\du}}\definecolor{diafillcolor}{rgb}{0.909804, 0.513726, 0.227451}
    \pgfsetfillcolor{diafillcolor}
    \pgfsetfillopacity{1.000000}
    \fill (45.677300\du,12.003000\du)--(45.677300\du,12.499266\du)--(47.496475\du,12.499266\du)--(47.496475\du,12.003000\du)--cycle;
    }{\pgfsetcornersarced{\pgfpoint{0.300000\du}{0.300000\du}}\definecolor{dialinecolor}{rgb}{0.909804, 0.513726, 0.227451}
    \pgfsetstrokecolor{dialinecolor}
    \pgfsetstrokeopacity{1.000000}
    \draw (45.677300\du,12.003000\du)--(45.677300\du,12.499266\du)--(47.496475\du,12.499266\du)--(47.496475\du,12.003000\du)--cycle;
    }\pgfsetlinewidth{0.000000\du}
    \pgfsetdash{}{0pt}
    \pgfsetroundjoin
    \pgfsetbuttcap
    {\pgfsetcornersarced{\pgfpoint{0.300000\du}{0.300000\du}}\definecolor{diafillcolor}{rgb}{0.909804, 0.513726, 0.227451}
    \pgfsetfillcolor{diafillcolor}
    \pgfsetfillopacity{1.000000}
    \fill (48.639800\du,12.006700\du)--(48.639800\du,12.506698\du)--(50.371850\du,12.506698\du)--(50.371850\du,12.006700\du)--cycle;
    }{\pgfsetcornersarced{\pgfpoint{0.300000\du}{0.300000\du}}\definecolor{dialinecolor}{rgb}{0.909804, 0.513726, 0.227451}
    \pgfsetstrokecolor{dialinecolor}
    \pgfsetstrokeopacity{1.000000}
    \draw (48.639800\du,12.006700\du)--(48.639800\du,12.506698\du)--(50.371850\du,12.506698\du)--(50.371850\du,12.006700\du)--cycle;
    }% setfont left to latex
    \definecolor{dialinecolor}{rgb}{1.000000, 1.000000, 1.000000}
    \pgfsetstrokecolor{dialinecolor}
    \pgfsetstrokeopacity{1.000000}
    \definecolor{diafillcolor}{rgb}{1.000000, 1.000000, 1.000000}
    \pgfsetfillcolor{diafillcolor}
    \pgfsetfillopacity{1.000000}
    \node[anchor=base,inner sep=0pt, outer sep=0pt,color=dialinecolor] at (46.586900\du,12.272353\du){Participantes};
    % setfont left to latex
    \definecolor{dialinecolor}{rgb}{1.000000, 1.000000, 1.000000}
    \pgfsetstrokecolor{dialinecolor}
    \pgfsetstrokeopacity{1.000000}
    \definecolor{diafillcolor}{rgb}{1.000000, 1.000000, 1.000000}
    \pgfsetfillcolor{diafillcolor}
    \pgfsetfillopacity{1.000000}
    \node[anchor=base,inner sep=0pt, outer sep=0pt,color=dialinecolor] at (49.505800\du,12.268653\du){Beneficiarios};
    \pgfsetlinewidth{0.050000\du}
    \pgfsetdash{}{0pt}
    \pgfsetmiterjoin
    \pgfsetbuttcap
    {
    \definecolor{diafillcolor}{rgb}{0.294118, 0.360784, 0.419608}
    \pgfsetfillcolor{diafillcolor}
    \pgfsetfillopacity{1.000000}
    % was here!!!
    \pgfsetarrowsend{stealth}
    {\pgfsetcornersarced{\pgfpoint{0.500000\du}{0.500000\du}}\definecolor{dialinecolor}{rgb}{0.294118, 0.360784, 0.419608}
    \pgfsetstrokecolor{dialinecolor}
    \pgfsetstrokeopacity{1.000000}
    \draw (55.010402\du,10.923575\du)--(55.010402\du,11.449837\du)--(58.005318\du,11.449837\du)--(58.005318\du,11.976100\du);
    }}
    \pgfsetlinewidth{0.050000\du}
    \pgfsetdash{}{0pt}
    \pgfsetmiterjoin
    \pgfsetbuttcap
    {
    \definecolor{diafillcolor}{rgb}{0.294118, 0.360784, 0.419608}
    \pgfsetfillcolor{diafillcolor}
    \pgfsetfillopacity{1.000000}
    % was here!!!
    \pgfsetarrowsend{stealth}
    {\pgfsetcornersarced{\pgfpoint{0.500000\du}{0.500000\du}}\definecolor{dialinecolor}{rgb}{0.294118, 0.360784, 0.419608}
    \pgfsetstrokecolor{dialinecolor}
    \pgfsetstrokeopacity{1.000000}
    \draw (55.010402\du,10.923575\du)--(55.010402\du,11.448587\du)--(52.003848\du,11.448587\du)--(52.003848\du,11.973600\du);
    }}
    \pgfsetlinewidth{0.050000\du}
    \pgfsetdash{}{0pt}
    \pgfsetroundjoin
    \pgfsetbuttcap
    {\pgfsetcornersarced{\pgfpoint{0.300000\du}{0.300000\du}}\definecolor{diafillcolor}{rgb}{0.396078, 0.345098, 0.960784}
    \pgfsetfillcolor{diafillcolor}
    \pgfsetfillopacity{1.000000}
    \fill (51.214800\du,11.973600\du)--(51.214800\du,12.840258\du)--(52.792896\du,12.840258\du)--(52.792896\du,11.973600\du)--cycle;
    }{\pgfsetcornersarced{\pgfpoint{0.300000\du}{0.300000\du}}\definecolor{dialinecolor}{rgb}{0.396078, 0.345098, 0.960784}
    \pgfsetstrokecolor{dialinecolor}
    \pgfsetstrokeopacity{1.000000}
    \draw (51.214800\du,11.973600\du)--(51.214800\du,12.840258\du)--(52.792896\du,12.840258\du)--(52.792896\du,11.973600\du)--cycle;
    }\pgfsetlinewidth{0.050000\du}
    \pgfsetdash{}{0pt}
    \pgfsetroundjoin
    \pgfsetbuttcap
    {\pgfsetcornersarced{\pgfpoint{0.300000\du}{0.300000\du}}\definecolor{diafillcolor}{rgb}{0.396078, 0.345098, 0.960784}
    \pgfsetfillcolor{diafillcolor}
    \pgfsetfillopacity{1.000000}
    \fill (56.896000\du,11.976100\du)--(56.896000\du,12.848566\du)--(59.114637\du,12.848566\du)--(59.114637\du,11.976100\du)--cycle;
    }{\pgfsetcornersarced{\pgfpoint{0.300000\du}{0.300000\du}}\definecolor{dialinecolor}{rgb}{0.396078, 0.345098, 0.960784}
    \pgfsetstrokecolor{dialinecolor}
    \pgfsetstrokeopacity{1.000000}
    \draw (56.896000\du,11.976100\du)--(56.896000\du,12.848566\du)--(59.114637\du,12.848566\du)--(59.114637\du,11.976100\du)--cycle;
    }\pgfsetlinewidth{0.050000\du}
    \pgfsetdash{}{0pt}
    \pgfsetroundjoin
    \pgfsetbuttcap
    {\pgfsetcornersarced{\pgfpoint{0.300000\du}{0.300000\du}}\definecolor{diafillcolor}{rgb}{0.396078, 0.345098, 0.960784}
    \pgfsetfillcolor{diafillcolor}
    \pgfsetfillopacity{1.000000}
    \fill (54.077300\du,11.938300\du)--(54.077300\du,12.856026\du)--(55.926869\du,12.856026\du)--(55.926869\du,11.938300\du)--cycle;
    }{\pgfsetcornersarced{\pgfpoint{0.300000\du}{0.300000\du}}\definecolor{dialinecolor}{rgb}{0.396078, 0.345098, 0.960784}
    \pgfsetstrokecolor{dialinecolor}
    \pgfsetstrokeopacity{1.000000}
    \draw (54.077300\du,11.938300\du)--(54.077300\du,12.856026\du)--(55.926869\du,12.856026\du)--(55.926869\du,11.938300\du)--cycle;
    }\pgfsetlinewidth{0.050000\du}
    \pgfsetdash{}{0pt}
    \pgfsetbuttcap
    {
    \definecolor{diafillcolor}{rgb}{0.294118, 0.360784, 0.419608}
    \pgfsetfillcolor{diafillcolor}
    \pgfsetfillopacity{1.000000}
    % was here!!!
    \pgfsetarrowsend{stealth}
    \definecolor{dialinecolor}{rgb}{0.294118, 0.360784, 0.419608}
    \pgfsetstrokecolor{dialinecolor}
    \pgfsetstrokeopacity{1.000000}
    \draw (55.010400\du,10.923600\du)--(55.002100\du,11.938300\du);
    }
    % setfont left to latex
    \definecolor{dialinecolor}{rgb}{1.000000, 1.000000, 1.000000}
    \pgfsetstrokecolor{dialinecolor}
    \pgfsetstrokeopacity{1.000000}
    \definecolor{diafillcolor}{rgb}{1.000000, 1.000000, 1.000000}
    \pgfsetfillcolor{diafillcolor}
    \pgfsetfillopacity{1.000000}
    \node[anchor=base,inner sep=0pt, outer sep=0pt,color=dialinecolor] at (52.003800\du,12.235553\du){Neutrales};
    % setfont left to latex
    \definecolor{dialinecolor}{rgb}{1.000000, 1.000000, 1.000000}
    \pgfsetstrokecolor{dialinecolor}
    \pgfsetstrokeopacity{1.000000}
    \definecolor{diafillcolor}{rgb}{1.000000, 1.000000, 1.000000}
    \pgfsetfillcolor{diafillcolor}
    \pgfsetfillopacity{1.000000}
    \node[anchor=base,inner sep=0pt, outer sep=0pt,color=dialinecolor] at (52.003800\du,12.588331\du){(excluídos)};
    % setfont left to latex
    \definecolor{dialinecolor}{rgb}{1.000000, 1.000000, 1.000000}
    \pgfsetstrokecolor{dialinecolor}
    \pgfsetstrokeopacity{1.000000}
    \definecolor{diafillcolor}{rgb}{1.000000, 1.000000, 1.000000}
    \pgfsetfillcolor{diafillcolor}
    \pgfsetfillopacity{1.000000}
    \node[anchor=base,inner sep=0pt, outer sep=0pt,color=dialinecolor] at (55.002100\du,12.200253\du){Beneficiarios};
    % setfont left to latex
    \definecolor{dialinecolor}{rgb}{1.000000, 1.000000, 1.000000}
    \pgfsetstrokecolor{dialinecolor}
    \pgfsetstrokeopacity{1.000000}
    \definecolor{diafillcolor}{rgb}{1.000000, 1.000000, 1.000000}
    \pgfsetfillcolor{diafillcolor}
    \pgfsetfillopacity{1.000000}
    \node[anchor=base,inner sep=0pt, outer sep=0pt,color=dialinecolor] at (55.002100\du,12.553031\du){indirectos};
    % setfont left to latex
    \definecolor{dialinecolor}{rgb}{1.000000, 1.000000, 1.000000}
    \pgfsetstrokecolor{dialinecolor}
    \pgfsetstrokeopacity{1.000000}
    \definecolor{diafillcolor}{rgb}{1.000000, 1.000000, 1.000000}
    \pgfsetfillcolor{diafillcolor}
    \pgfsetfillopacity{1.000000}
    \node[anchor=base,inner sep=0pt, outer sep=0pt,color=dialinecolor] at (58.005300\du,12.238053\du){Con posibilidad};
    % setfont left to latex
    \definecolor{dialinecolor}{rgb}{1.000000, 1.000000, 1.000000}
    \pgfsetstrokecolor{dialinecolor}
    \pgfsetstrokeopacity{1.000000}
    \definecolor{diafillcolor}{rgb}{1.000000, 1.000000, 1.000000}
    \pgfsetfillcolor{diafillcolor}
    \pgfsetfillopacity{1.000000}
    \node[anchor=base,inner sep=0pt, outer sep=0pt,color=dialinecolor] at (58.005300\du,12.590831\du){ de oposición};
    \pgfsetlinewidth{0.050000\du}
    \pgfsetdash{}{0pt}
    \pgfsetbuttcap
    {
    \definecolor{diafillcolor}{rgb}{0.294118, 0.360784, 0.419608}
    \pgfsetfillcolor{diafillcolor}
    \pgfsetfillopacity{1.000000}
    % was here!!!
    \pgfsetarrowsend{stealth}
    \definecolor{dialinecolor}{rgb}{0.294118, 0.360784, 0.419608}
    \pgfsetstrokecolor{dialinecolor}
    \pgfsetstrokeopacity{1.000000}
    \draw (46.586900\du,12.499300\du)--(46.595300\du,13.500100\du);
    }
    \pgfsetlinewidth{0.000000\du}
    \pgfsetdash{}{0pt}
    \pgfsetmiterjoin
    \pgfsetbuttcap
    {\pgfsetcornersarced{\pgfpoint{0.000000\du}{0.000000\du}}\definecolor{diafillcolor}{rgb}{0.968627, 0.764706, 0.145098}
    \pgfsetfillcolor{diafillcolor}
    \pgfsetfillopacity{1.000000}
    \fill (45.727300\du,13.500100\du)--(45.727300\du,15.295029\du)--(47.463306\du,15.295029\du)--(47.463306\du,13.500100\du)--cycle;
    }{\pgfsetcornersarced{\pgfpoint{0.000000\du}{0.000000\du}}\definecolor{dialinecolor}{rgb}{0.968627, 0.764706, 0.145098}
    \pgfsetstrokecolor{dialinecolor}
    \pgfsetstrokeopacity{1.000000}
    \draw (45.727300\du,13.500100\du)--(45.727300\du,15.295029\du)--(47.463306\du,15.295029\du)--(47.463306\du,13.500100\du)--cycle;
    }% setfont left to latex
    \definecolor{dialinecolor}{rgb}{0.176471, 0.231373, 0.270588}
    \pgfsetstrokecolor{dialinecolor}
    \pgfsetstrokeopacity{1.000000}
    \definecolor{diafillcolor}{rgb}{0.176471, 0.231373, 0.270588}
    \pgfsetfillcolor{diafillcolor}
    \pgfsetfillopacity{1.000000}
    \node[anchor=base,inner sep=0pt, outer sep=0pt,color=dialinecolor] at (46.595300\du,13.662215\du){1. Alcaldía};
    % setfont left to latex
    \definecolor{dialinecolor}{rgb}{0.176471, 0.231373, 0.270588}
    \pgfsetstrokecolor{dialinecolor}
    \pgfsetstrokeopacity{1.000000}
    \definecolor{diafillcolor}{rgb}{0.176471, 0.231373, 0.270588}
    \pgfsetfillcolor{diafillcolor}
    \pgfsetfillopacity{1.000000}
    \node[anchor=base,inner sep=0pt, outer sep=0pt,color=dialinecolor] at (46.595300\du,13.873882\du){2. Urbaser};
    % setfont left to latex
    \definecolor{dialinecolor}{rgb}{0.176471, 0.231373, 0.270588}
    \pgfsetstrokecolor{dialinecolor}
    \pgfsetstrokeopacity{1.000000}
    \definecolor{diafillcolor}{rgb}{0.176471, 0.231373, 0.270588}
    \pgfsetfillcolor{diafillcolor}
    \pgfsetfillopacity{1.000000}
    \node[anchor=base,inner sep=0pt, outer sep=0pt,color=dialinecolor] at (46.595300\du,14.085549\du){3. Secretaria de};
    % setfont left to latex
    \definecolor{dialinecolor}{rgb}{0.176471, 0.231373, 0.270588}
    \pgfsetstrokecolor{dialinecolor}
    \pgfsetstrokeopacity{1.000000}
    \definecolor{diafillcolor}{rgb}{0.176471, 0.231373, 0.270588}
    \pgfsetfillcolor{diafillcolor}
    \pgfsetfillopacity{1.000000}
    \node[anchor=base,inner sep=0pt, outer sep=0pt,color=dialinecolor] at (46.595300\du,14.297215\du){Salud};
    % setfont left to latex
    \definecolor{dialinecolor}{rgb}{0.176471, 0.231373, 0.270588}
    \pgfsetstrokecolor{dialinecolor}
    \pgfsetstrokeopacity{1.000000}
    \definecolor{diafillcolor}{rgb}{0.176471, 0.231373, 0.270588}
    \pgfsetfillcolor{diafillcolor}
    \pgfsetfillopacity{1.000000}
    \node[anchor=base,inner sep=0pt, outer sep=0pt,color=dialinecolor] at (46.595300\du,14.508882\du){4. CRC};
    % setfont left to latex
    \definecolor{dialinecolor}{rgb}{0.176471, 0.231373, 0.270588}
    \pgfsetstrokecolor{dialinecolor}
    \pgfsetstrokeopacity{1.000000}
    \definecolor{diafillcolor}{rgb}{0.176471, 0.231373, 0.270588}
    \pgfsetfillcolor{diafillcolor}
    \pgfsetfillopacity{1.000000}
    \node[anchor=base,inner sep=0pt, outer sep=0pt,color=dialinecolor] at (46.595300\du,14.720549\du){5 Acueducto y};
    % setfont left to latex
    \definecolor{dialinecolor}{rgb}{0.176471, 0.231373, 0.270588}
    \pgfsetstrokecolor{dialinecolor}
    \pgfsetstrokeopacity{1.000000}
    \definecolor{diafillcolor}{rgb}{0.176471, 0.231373, 0.270588}
    \pgfsetfillcolor{diafillcolor}
    \pgfsetfillopacity{1.000000}
    \node[anchor=base,inner sep=0pt, outer sep=0pt,color=dialinecolor] at (46.595300\du,14.932215\du){alcantarillado};
    \pgfsetlinewidth{0.050000\du}
    \pgfsetdash{}{0pt}
    \pgfsetbuttcap
    {
    \definecolor{diafillcolor}{rgb}{0.294118, 0.360784, 0.419608}
    \pgfsetfillcolor{diafillcolor}
    \pgfsetfillopacity{1.000000}
    % was here!!!
    \pgfsetarrowsend{stealth}
    \definecolor{dialinecolor}{rgb}{0.294118, 0.360784, 0.419608}
    \pgfsetstrokecolor{dialinecolor}
    \pgfsetstrokeopacity{1.000000}
    \draw (49.505800\du,12.506700\du)--(49.515100\du,13.506400\du);
    }
    \pgfsetlinewidth{0.000000\du}
    \pgfsetdash{}{0pt}
    \pgfsetmiterjoin
    \pgfsetbuttcap
    {\pgfsetcornersarced{\pgfpoint{0.000000\du}{0.000000\du}}\definecolor{diafillcolor}{rgb}{0.968627, 0.764706, 0.145098}
    \pgfsetfillcolor{diafillcolor}
    \pgfsetfillopacity{1.000000}
    \fill (48.939800\du,13.506400\du)--(48.939800\du,14.263161\du)--(50.090347\du,14.263161\du)--(50.090347\du,13.506400\du)--cycle;
    }{\pgfsetcornersarced{\pgfpoint{0.000000\du}{0.000000\du}}\definecolor{dialinecolor}{rgb}{0.968627, 0.764706, 0.145098}
    \pgfsetstrokecolor{dialinecolor}
    \pgfsetstrokeopacity{1.000000}
    \draw (48.939800\du,13.506400\du)--(48.939800\du,14.263161\du)--(50.090347\du,14.263161\du)--(50.090347\du,13.506400\du)--cycle;
    }% setfont left to latex
    \definecolor{dialinecolor}{rgb}{0.176471, 0.231373, 0.270588}
    \pgfsetstrokecolor{dialinecolor}
    \pgfsetstrokeopacity{1.000000}
    \definecolor{diafillcolor}{rgb}{0.176471, 0.231373, 0.270588}
    \pgfsetfillcolor{diafillcolor}
    \pgfsetfillopacity{1.000000}
    \node[anchor=base,inner sep=0pt, outer sep=0pt,color=dialinecolor] at (49.515100\du,13.663587\du){1. Población};
    % setfont left to latex
    \definecolor{dialinecolor}{rgb}{0.176471, 0.231373, 0.270588}
    \pgfsetstrokecolor{dialinecolor}
    \pgfsetstrokeopacity{1.000000}
    \definecolor{diafillcolor}{rgb}{0.176471, 0.231373, 0.270588}
    \pgfsetfillcolor{diafillcolor}
    \pgfsetfillopacity{1.000000}
    \node[anchor=base,inner sep=0pt, outer sep=0pt,color=dialinecolor] at (49.515100\du,13.875254\du){aledaña};
    % setfont left to latex
    \definecolor{dialinecolor}{rgb}{0.176471, 0.231373, 0.270588}
    \pgfsetstrokecolor{dialinecolor}
    \pgfsetstrokeopacity{1.000000}
    \definecolor{diafillcolor}{rgb}{0.176471, 0.231373, 0.270588}
    \pgfsetfillcolor{diafillcolor}
    \pgfsetfillopacity{1.000000}
    \node[anchor=base,inner sep=0pt, outer sep=0pt,color=dialinecolor] at (49.515100\du,14.086921\du){2. Animales};
    \pgfsetlinewidth{0.050000\du}
    \pgfsetdash{}{0pt}
    \pgfsetbuttcap
    {
    \definecolor{diafillcolor}{rgb}{0.294118, 0.360784, 0.419608}
    \pgfsetfillcolor{diafillcolor}
    \pgfsetfillopacity{1.000000}
    % was here!!!
    \pgfsetarrowsend{stealth}
    \definecolor{dialinecolor}{rgb}{0.294118, 0.360784, 0.419608}
    \pgfsetstrokecolor{dialinecolor}
    \pgfsetstrokeopacity{1.000000}
    \draw (52.003800\du,12.840300\du)--(52.002400\du,13.509200\du);
    }
    \pgfsetlinewidth{0.000000\du}
    \pgfsetdash{}{0pt}
    \pgfsetmiterjoin
    \pgfsetbuttcap
    {\pgfsetcornersarced{\pgfpoint{0.000000\du}{0.000000\du}}\definecolor{diafillcolor}{rgb}{0.968627, 0.764706, 0.145098}
    \pgfsetfillcolor{diafillcolor}
    \pgfsetfillopacity{1.000000}
    \fill (50.918000\du,13.509200\du)--(50.918000\du,14.741530\du)--(53.086735\du,14.741530\du)--(53.086735\du,13.509200\du)--cycle;
    }{\pgfsetcornersarced{\pgfpoint{0.000000\du}{0.000000\du}}\definecolor{dialinecolor}{rgb}{0.968627, 0.764706, 0.145098}
    \pgfsetstrokecolor{dialinecolor}
    \pgfsetstrokeopacity{1.000000}
    \draw (50.918000\du,13.509200\du)--(50.918000\du,14.741530\du)--(53.086735\du,14.741530\du)--(53.086735\du,13.509200\du)--cycle;
    }% setfont left to latex
    \definecolor{dialinecolor}{rgb}{0.176471, 0.231373, 0.270588}
    \pgfsetstrokecolor{dialinecolor}
    \pgfsetstrokeopacity{1.000000}
    \definecolor{diafillcolor}{rgb}{0.176471, 0.231373, 0.270588}
    \pgfsetfillcolor{diafillcolor}
    \pgfsetfillopacity{1.000000}
    \node[anchor=base,inner sep=0pt, outer sep=0pt,color=dialinecolor] at (52.002400\du,13.666387\du){1. Habitantes sin};
    % setfont left to latex
    \definecolor{dialinecolor}{rgb}{0.176471, 0.231373, 0.270588}
    \pgfsetstrokecolor{dialinecolor}
    \pgfsetstrokeopacity{1.000000}
    \definecolor{diafillcolor}{rgb}{0.176471, 0.231373, 0.270588}
    \pgfsetfillcolor{diafillcolor}
    \pgfsetfillopacity{1.000000}
    \node[anchor=base,inner sep=0pt, outer sep=0pt,color=dialinecolor] at (52.002400\du,13.878054\du){intervención a los ríos};
    % setfont left to latex
    \definecolor{dialinecolor}{rgb}{0.176471, 0.231373, 0.270588}
    \pgfsetstrokecolor{dialinecolor}
    \pgfsetstrokeopacity{1.000000}
    \definecolor{diafillcolor}{rgb}{0.176471, 0.231373, 0.270588}
    \pgfsetfillcolor{diafillcolor}
    \pgfsetfillopacity{1.000000}
    \node[anchor=base,inner sep=0pt, outer sep=0pt,color=dialinecolor] at (52.002400\du,14.089721\du){2. Comunas sin };
    % setfont left to latex
    \definecolor{dialinecolor}{rgb}{0.176471, 0.231373, 0.270588}
    \pgfsetstrokecolor{dialinecolor}
    \pgfsetstrokeopacity{1.000000}
    \definecolor{diafillcolor}{rgb}{0.176471, 0.231373, 0.270588}
    \pgfsetfillcolor{diafillcolor}
    \pgfsetfillopacity{1.000000}
    \node[anchor=base,inner sep=0pt, outer sep=0pt,color=dialinecolor] at (52.002400\du,14.301388\du){interacción directa };
    % setfont left to latex
    \definecolor{dialinecolor}{rgb}{0.176471, 0.231373, 0.270588}
    \pgfsetstrokecolor{dialinecolor}
    \pgfsetstrokeopacity{1.000000}
    \definecolor{diafillcolor}{rgb}{0.176471, 0.231373, 0.270588}
    \pgfsetfillcolor{diafillcolor}
    \pgfsetfillopacity{1.000000}
    \node[anchor=base,inner sep=0pt, outer sep=0pt,color=dialinecolor] at (52.002400\du,14.513054\du){con los ríos};
    \pgfsetlinewidth{0.050000\du}
    \pgfsetdash{}{0pt}
    \pgfsetbuttcap
    {
    \definecolor{diafillcolor}{rgb}{0.294118, 0.360784, 0.419608}
    \pgfsetfillcolor{diafillcolor}
    \pgfsetfillopacity{1.000000}
    % was here!!!
    \pgfsetarrowsend{stealth}
    \definecolor{dialinecolor}{rgb}{0.294118, 0.360784, 0.419608}
    \pgfsetstrokecolor{dialinecolor}
    \pgfsetstrokeopacity{1.000000}
    \draw (55.002100\du,12.856000\du)--(55.008300\du,13.504800\du);
    }
    \pgfsetlinewidth{0.000000\du}
    \pgfsetdash{}{0pt}
    \pgfsetmiterjoin
    \pgfsetbuttcap
    {\pgfsetcornersarced{\pgfpoint{0.000000\du}{0.000000\du}}\definecolor{diafillcolor}{rgb}{0.968627, 0.764706, 0.145098}
    \pgfsetfillcolor{diafillcolor}
    \pgfsetfillopacity{1.000000}
    \fill (54.161100\du,13.504800\du)--(54.161100\du,14.268839\du)--(55.855443\du,14.268839\du)--(55.855443\du,13.504800\du)--cycle;
    }{\pgfsetcornersarced{\pgfpoint{0.000000\du}{0.000000\du}}\definecolor{dialinecolor}{rgb}{0.968627, 0.764706, 0.145098}
    \pgfsetstrokecolor{dialinecolor}
    \pgfsetstrokeopacity{1.000000}
    \draw (54.161100\du,13.504800\du)--(54.161100\du,14.268839\du)--(55.855443\du,14.268839\du)--(55.855443\du,13.504800\du)--cycle;
    }% setfont left to latex
    \definecolor{dialinecolor}{rgb}{0.176471, 0.231373, 0.270588}
    \pgfsetstrokecolor{dialinecolor}
    \pgfsetstrokeopacity{1.000000}
    \definecolor{diafillcolor}{rgb}{0.176471, 0.231373, 0.270588}
    \pgfsetfillcolor{diafillcolor}
    \pgfsetfillopacity{1.000000}
    \node[anchor=base,inner sep=0pt, outer sep=0pt,color=dialinecolor] at (55.008300\du,13.661987\du){1. Turistas};
    % setfont left to latex
    \definecolor{dialinecolor}{rgb}{0.176471, 0.231373, 0.270588}
    \pgfsetstrokecolor{dialinecolor}
    \pgfsetstrokeopacity{1.000000}
    \definecolor{diafillcolor}{rgb}{0.176471, 0.231373, 0.270588}
    \pgfsetfillcolor{diafillcolor}
    \pgfsetfillopacity{1.000000}
    \node[anchor=base,inner sep=0pt, outer sep=0pt,color=dialinecolor] at (55.008300\du,13.873654\du){2. Campesinos};
    % setfont left to latex
    \definecolor{dialinecolor}{rgb}{0.176471, 0.231373, 0.270588}
    \pgfsetstrokecolor{dialinecolor}
    \pgfsetstrokeopacity{1.000000}
    \definecolor{diafillcolor}{rgb}{0.176471, 0.231373, 0.270588}
    \pgfsetfillcolor{diafillcolor}
    \pgfsetfillopacity{1.000000}
    \node[anchor=base,inner sep=0pt, outer sep=0pt,color=dialinecolor] at (55.008300\du,14.085321\du){3. Flora y fauna};
    \pgfsetlinewidth{0.050000\du}
    \pgfsetdash{}{0pt}
    \pgfsetbuttcap
    {
    \definecolor{diafillcolor}{rgb}{0.294118, 0.360784, 0.419608}
    \pgfsetfillcolor{diafillcolor}
    \pgfsetfillopacity{1.000000}
    % was here!!!
    \pgfsetarrowsend{stealth}
    \definecolor{dialinecolor}{rgb}{0.294118, 0.360784, 0.419608}
    \pgfsetstrokecolor{dialinecolor}
    \pgfsetstrokeopacity{1.000000}
    \draw (58.005300\du,12.848600\du)--(58.010200\du,13.506400\du);
    }
    \pgfsetlinewidth{0.000000\du}
    \pgfsetdash{}{0pt}
    \pgfsetmiterjoin
    \pgfsetbuttcap
    {\pgfsetcornersarced{\pgfpoint{0.000000\du}{0.000000\du}}\definecolor{diafillcolor}{rgb}{0.968627, 0.764706, 0.145098}
    \pgfsetfillcolor{diafillcolor}
    \pgfsetfillopacity{1.000000}
    \fill (57.227300\du,13.506400\du)--(57.227300\du,14.087106\du)--(58.793197\du,14.087106\du)--(58.793197\du,13.506400\du)--cycle;
    }{\pgfsetcornersarced{\pgfpoint{0.000000\du}{0.000000\du}}\definecolor{dialinecolor}{rgb}{0.968627, 0.764706, 0.145098}
    \pgfsetstrokecolor{dialinecolor}
    \pgfsetstrokeopacity{1.000000}
    \draw (57.227300\du,13.506400\du)--(57.227300\du,14.087106\du)--(58.793197\du,14.087106\du)--(58.793197\du,13.506400\du)--cycle;
    }% setfont left to latex
    \definecolor{dialinecolor}{rgb}{0.176471, 0.231373, 0.270588}
    \pgfsetstrokecolor{dialinecolor}
    \pgfsetstrokeopacity{1.000000}
    \definecolor{diafillcolor}{rgb}{0.176471, 0.231373, 0.270588}
    \pgfsetfillcolor{diafillcolor}
    \pgfsetfillopacity{1.000000}
    \node[anchor=base,inner sep=0pt, outer sep=0pt,color=dialinecolor] at (58.010200\du,13.663587\du){1. Microempresas};
    % setfont left to latex
    \definecolor{dialinecolor}{rgb}{0.176471, 0.231373, 0.270588}
    \pgfsetstrokecolor{dialinecolor}
    \pgfsetstrokeopacity{1.000000}
    \definecolor{diafillcolor}{rgb}{0.176471, 0.231373, 0.270588}
    \pgfsetfillcolor{diafillcolor}
    \pgfsetfillopacity{1.000000}
    \node[anchor=base,inner sep=0pt, outer sep=0pt,color=dialinecolor] at (58.010200\du,13.875254\du){2. Indigentes};
    \end{tikzpicture}
\end{figure}

  El árbol de participación que se realizó se tuvo en cuenta la información recolectada en los sitios web de las diferentes instituciones del municipio de Popayán y el departamento del Cauca, los cuales se manifiesta que la Alcaldía junto con la Secretaria de Salud ya sea Municipal o Departamental trabajan en articulación junto con la empresa prestadoras de servicios públicos de acueducto y alcantarillado de dicho municipio para realizar actividades como actualizaciones de mapas de riesgo de calidad del agua, inspección, vigilancia y control de riesgos asociados a las condiciones de calidad de las cuencas. 

  Sin embargo, estas actividades son orientadas para ser efectuadas en los tramos de las cuencas antes de la captación del agua para el acueducto del municipio o en este caso para la ciudad de Popayán, normalmente este proceso de captación se realiza antes del paso de la cuenca por la ciudad. Esto implica que los procesos de inspección, vigilancia y control de riesgos asociados a las condiciones de calidad de las cuencas son ignorados por alguna de estas entidades. Como la ciudad de Popayán cuenta con un sistema de acueducto las aguas residuales de tipo domestico, comercial, industrial y de aguas lluvias son canalizadas y tratadas para su respectiva descontaminación, esto presenta problemas mínimos de contaminación concentrando la preocupación ambiental en otro aspecto.

  La contaminación de mayor magnitud que presencian las cuencas que atraviesan la ciudad es causada por los mismos habitantes del sector y habitantes que lo transitan, aunque realizar una actividad para concientizar a las personas del cuidado del medio ambiente y de las fuentes hídricas es muy complejo cuando se trata de controlar a un gran numero de personas. Para eso existen organizaciones con o sin animo de lucro para realizar actividades de limpieza junto con la colaboración de las personas que se ven beneficiadas de ellas, organizaciones como La Corporación Autónoma Regional del Cauca (CRC), la empresa encargada de los servicios púbicos del aseo de la ciudad (Urbaser) y la Secretaria de Salud trabajan en conjunto para planificar, administrar, hacer seguimiento y monitoreo de las fuentes hídricas con el fin de al máximo evitar los conflictos que se presentan asociados al recurso.

  Todas las cuencas, microcuencas y macrocuencas presentes en el municipio tienen como desembocadura en el río Cauca cuya fuente hídrica es una de las más importantes de Colombia. Realizar limpiezas desde los recursos minoritarios es una buena practica para el cuidado de ríos grandes como este, así muchos grupos se van a ver beneficiados, como por ejemplo campesinos, pueblos, entro otros. Pero no solamente los daños ambientales afectan a personas, sino que también afectan a todo ser vivo que dependan de estos como animales y plantas, por ello las autoridades del cuidado ambiental de los países decretan y regulan sanciones para aquellas personas que son encontradas en fragancia realizando daños a los recursos hídricos y exigir la reparación de los daños causados.

  La Corporación Autónoma del Cauca es la entidad encargada de realizar estas acciones de regulación y sanción en el departamento, pero las actividades que se realizan para llevar un monitoreo o vigilancia para estos escenarios son mínimos ya que puede llegar a afectar a organizaciones pequeñas que no cuentan con estrategias para la manipulación de sus residuos y adicionalmente habitantes de las calles que optan por usar las orillas o puentes como su vivienda.

  Prueba de cambios.

\section{Análisis de Problemas}
    % Graphic for TeX using PGF
% Title: C:\Users\Nicolas Chicaiza\Dia\arbolProblemas.dia
% Creator: Dia v0.97.2
% CreationDate: Sun May 02 21:05:28 2021
% For: Nicolas Chicaiza
% \usepackage{tikz}
% The following commands are not supported in PSTricks at present
% We define them conditionally, so when they are implemented,
% this pgf file will use them.
\definecolor{titulo}{RGB}{41,56,69}
\ifx\du\undefined
\newlength{\du}
\fi
\setlength{\du}{15\unitlength}
\begin{tikzpicture}
    \pgftransformxscale{1.000000}
    \pgftransformyscale{-1.000000}
    \definecolor{dialinecolor}{rgb}{0.000000, 0.000000, 0.000000}
    \pgfsetstrokecolor{dialinecolor}
    \definecolor{dialinecolor}{rgb}{1.000000, 1.000000, 1.000000}
    \pgfsetfillcolor{dialinecolor}
    \pgfsetlinewidth{0.100000\du}
    \pgfsetdash{}{0pt}
    \pgfsetdash{}{0pt}
    \pgfsetmiterjoin
    \pgfsetbuttcap
    {
    \definecolor{dialinecolor}{rgb}{0.101961, 0.682353, 0.623529}
    \pgfsetfillcolor{dialinecolor}
    % was here!!!
    \definecolor{dialinecolor}{rgb}{0.101961, 0.682353, 0.623529}
    \pgfsetstrokecolor{dialinecolor}
    \pgfpathmoveto{\pgfpoint{37.405868\du}{11.384334\du}}
    \pgfpathcurveto{\pgfpoint{37.394405\du}{12.662399\du}}{\pgfpoint{41.001828\du}{12.864948\du}}{\pgfpoint{41.001828\du}{14.364948\du}}
    \pgfusepath{stroke}
    }
    \pgfsetlinewidth{0.100000\du}
    \pgfsetdash{}{0pt}
    \pgfsetdash{}{0pt}
    \pgfsetmiterjoin
    \pgfsetbuttcap
    {
    \definecolor{dialinecolor}{rgb}{0.741176, 0.203922, 0.819608}
    \pgfsetfillcolor{dialinecolor}
    % was here!!!
    \definecolor{dialinecolor}{rgb}{0.741176, 0.203922, 0.819608}
    \pgfsetstrokecolor{dialinecolor}
    \pgfpathmoveto{\pgfpoint{36.592704\du}{11.384334\du}}
    \pgfpathcurveto{\pgfpoint{36.593868\du}{12.642708\du}}{\pgfpoint{32.973106\du}{12.903483\du}}{\pgfpoint{32.973106\du}{14.364948\du}}
    \pgfusepath{stroke}
    }
    \pgfsetlinewidth{0.100000\du}
    \pgfsetdash{}{0pt}
    \pgfsetdash{}{0pt}
    \pgfsetmiterjoin
    \pgfsetbuttcap
    {
    \definecolor{dialinecolor}{rgb}{0.047059, 0.568627, 0.772549}
    \pgfsetfillcolor{dialinecolor}
    % was here!!!
    \definecolor{dialinecolor}{rgb}{0.047059, 0.568627, 0.772549}
    \pgfsetstrokecolor{dialinecolor}
    \pgfpathmoveto{\pgfpoint{35.797217\du}{11.331302\du}}
    \pgfpathcurveto{\pgfpoint{35.785755\du}{12.804229\du}}{\pgfpoint{25.993979\du}{13.359000\du}}{\pgfpoint{25.993979\du}{14.378579\du}}
    \pgfusepath{stroke}
    }
    % setfont left to latex
    \definecolor{dialinecolor}{rgb}{0.000000, 0.000000, 0.000000}
    \pgfsetstrokecolor{dialinecolor}
    \node at (25.942974\du,14.815404\du){\scriptsize Represamiento del};
    % setfont left to latex
    \definecolor{dialinecolor}{rgb}{0.000000, 0.000000, 0.000000}
    \pgfsetstrokecolor{dialinecolor}
    \node at (25.942974\du,15.344571\du){\scriptsize flujo hidrico};
    % setfont left to latex
    \definecolor{dialinecolor}{rgb}{0.000000, 0.000000, 0.000000}
    \pgfsetstrokecolor{dialinecolor}
    \node[anchor=west] at (24.709536\du,16.205230\du){};
    % setfont left to latex
    \definecolor{dialinecolor}{rgb}{0.000000, 0.000000, 0.000000}
    \pgfsetstrokecolor{dialinecolor}
    \node[anchor=west] at (37.027529\du,10.825328\du){};
    \pgfsetlinewidth{0.100000\du}
    \pgfsetdash{}{0pt}
    \pgfsetdash{}{0pt}
    \pgfsetmiterjoin
    \pgfsetbuttcap
    {
    \definecolor{dialinecolor}{rgb}{0.047059, 0.568627, 0.772549}
    \pgfsetfillcolor{dialinecolor}
    % was here!!!
    \definecolor{dialinecolor}{rgb}{0.047059, 0.568627, 0.772549}
    \pgfsetstrokecolor{dialinecolor}
    \pgfpathmoveto{\pgfpoint{26.009356\du}{15.642044\du}}
    \pgfpathcurveto{\pgfpoint{26.008024\du}{16.967970\du}}{\pgfpoint{28.110242\du}{16.127083\du}}{\pgfpoint{28.109356\du}{17.042044\du}}
    \pgfusepath{stroke}
    }
    \pgfsetlinewidth{0.100000\du}
    \pgfsetdash{}{0pt}
    \pgfsetdash{}{0pt}
    \pgfsetmiterjoin
    \pgfsetbuttcap
    {
    \definecolor{dialinecolor}{rgb}{0.047059, 0.568627, 0.772549}
    \pgfsetfillcolor{dialinecolor}
    % was here!!!
    \definecolor{dialinecolor}{rgb}{0.047059, 0.568627, 0.772549}
    \pgfsetstrokecolor{dialinecolor}
    \pgfpathmoveto{\pgfpoint{26.007805\du}{15.603965\du}}
    \pgfpathcurveto{\pgfpoint{25.997513\du}{16.894392\du}}{\pgfpoint{23.905806\du}{16.085038\du}}{\pgfpoint{23.909356\du}{17.042044\du}}
    \pgfusepath{stroke}
    }
    % setfont left to latex
    \definecolor{dialinecolor}{rgb}{0.000000, 0.000000, 0.000000}
    \pgfsetstrokecolor{dialinecolor}
    \node at (23.893734\du,17.351322\du){\tiny Desechos de};
    % setfont left to latex
    \definecolor{dialinecolor}{rgb}{0.000000, 0.000000, 0.000000}
    \pgfsetstrokecolor{dialinecolor}
    \node at (23.893734\du,17.774655\du){\tiny residuos sólidos};
    % setfont left to latex
    \definecolor{dialinecolor}{rgb}{0.000000, 0.000000, 0.000000}
    \pgfsetstrokecolor{dialinecolor}
    \node at (28.109356\du,17.346702\du){\tiny Malos};
    % setfont left to latex
    \definecolor{dialinecolor}{rgb}{0.000000, 0.000000, 0.000000}
    \pgfsetstrokecolor{dialinecolor}
    \node at (28.109356\du,17.770035\du){\tiny olores};
    % setfont left to latex
    \definecolor{dialinecolor}{rgb}{0.000000, 0.000000, 0.000000}
    \pgfsetstrokecolor{dialinecolor}
    \node at (33.015835\du,14.833847\du){\scriptsize Presencia de};
    % setfont left to latex
    \definecolor{dialinecolor}{rgb}{0.000000, 0.000000, 0.000000}
    \pgfsetstrokecolor{dialinecolor}
    \node at (33.015835\du,15.363014\du){\scriptsize bacterias};
    \pgfsetlinewidth{0.100000\du}
    \pgfsetdash{}{0pt}
    \pgfsetdash{}{0pt}
    \pgfsetmiterjoin
    \pgfsetbuttcap
    {
    \definecolor{dialinecolor}{rgb}{0.741176, 0.203922, 0.819608}
    \pgfsetfillcolor{dialinecolor}
    % was here!!!
    \definecolor{dialinecolor}{rgb}{0.741176, 0.203922, 0.819608}
    \pgfsetstrokecolor{dialinecolor}
    \pgfpathmoveto{\pgfpoint{33.014808\du}{15.638079\du}}
    \pgfpathcurveto{\pgfpoint{33.008410\du}{16.936437\du}}{\pgfpoint{35.089606\du}{16.106061\du}}{\pgfpoint{35.113256\du}{17.000000\du}}
    \pgfusepath{stroke}
    }
    \pgfsetlinewidth{0.100000\du}
    \pgfsetdash{}{0pt}
    \pgfsetdash{}{0pt}
    \pgfsetmiterjoin
    \pgfsetbuttcap
    {
    \definecolor{dialinecolor}{rgb}{0.741176, 0.203922, 0.819608}
    \pgfsetfillcolor{dialinecolor}
    % was here!!!
    \definecolor{dialinecolor}{rgb}{0.741176, 0.203922, 0.819608}
    \pgfsetstrokecolor{dialinecolor}
    \pgfpathmoveto{\pgfpoint{33.013256\du}{15.600000\du}}
    \pgfpathcurveto{\pgfpoint{32.997899\du}{16.936437\du}}{\pgfpoint{31.011303\du}{16.095550\du}}{\pgfpoint{31.013256\du}{17.000000\du}}
    \pgfusepath{stroke}
    }
    % setfont left to latex
    \definecolor{dialinecolor}{rgb}{0.000000, 0.000000, 0.000000}
    \pgfsetstrokecolor{dialinecolor}
    \node[anchor=west] at (31.800000\du,14.800000\du){};
    % setfont left to latex
    \definecolor{dialinecolor}{rgb}{0.000000, 0.000000, 0.000000}
    \pgfsetstrokecolor{dialinecolor}
    \node[anchor=west] at (30.800000\du,15.000000\du){};
    % setfont left to latex
    \definecolor{dialinecolor}{rgb}{0.000000, 0.000000, 0.000000}
    \pgfsetstrokecolor{dialinecolor}
    \node[anchor=west] at (32.000000\du,15.000000\du){};
    % setfont left to latex
    \definecolor{dialinecolor}{rgb}{0.000000, 0.000000, 0.000000}
    \pgfsetstrokecolor{dialinecolor}
    \node[anchor=west] at (32.600000\du,15.200000\du){};
    % setfont left to latex
    \definecolor{dialinecolor}{rgb}{0.000000, 0.000000, 0.000000}
    \pgfsetstrokecolor{dialinecolor}
    \node[anchor=west] at (31.400000\du,14.600000\du){};
    % setfont left to latex
    \definecolor{dialinecolor}{rgb}{0.000000, 0.000000, 0.000000}
    \pgfsetstrokecolor{dialinecolor}
    \node at (31.013256\du,17.309847\du){\tiny Químicos};
    % setfont left to latex
    \definecolor{dialinecolor}{rgb}{0.000000, 0.000000, 0.000000}
    \pgfsetstrokecolor{dialinecolor}
    \node at (31.013256\du,17.733181\du){\tiny dañinos};
    % setfont left to latex
    \definecolor{dialinecolor}{rgb}{0.000000, 0.000000, 0.000000}
    \pgfsetstrokecolor{dialinecolor}
    \node at (35.107016\du,17.319207\du){\tiny Destrucción de};
    % setfont left to latex
    \definecolor{dialinecolor}{rgb}{0.000000, 0.000000, 0.000000}
    \pgfsetstrokecolor{dialinecolor}
    \node at (35.107016\du,17.742541\du){\tiny flora y fauna};
    \pgfsetlinewidth{0.100000\du}
    \pgfsetdash{}{0pt}
    \pgfsetdash{}{0pt}
    \pgfsetmiterjoin
    \pgfsetbuttcap
    {
    \definecolor{dialinecolor}{rgb}{0.396078, 0.345098, 0.960784}
    \pgfsetfillcolor{dialinecolor}
    % was here!!!
    \definecolor{dialinecolor}{rgb}{0.396078, 0.345098, 0.960784}
    \pgfsetstrokecolor{dialinecolor}
    \pgfpathmoveto{\pgfpoint{38.236709\du}{11.348979\du}}
    \pgfpathcurveto{\pgfpoint{38.156472\du}{12.896409\du}}{\pgfpoint{47.953693\du}{13.345369\du}}{\pgfpoint{47.953693\du}{14.364948\du}}
    \pgfusepath{stroke}
    }
    % setfont left to latex
    \definecolor{dialinecolor}{rgb}{0.000000, 0.000000, 0.000000}
    \pgfsetstrokecolor{dialinecolor}
    \node at (48.074411\du,14.857680\du){\scriptsize Mala gestión};
    % setfont left to latex
    \definecolor{dialinecolor}{rgb}{0.000000, 0.000000, 0.000000}
    \pgfsetstrokecolor{dialinecolor}
    \node at (48.074411\du,15.386846\du){\scriptsize de la basura};
    % setfont left to latex
    \definecolor{dialinecolor}{rgb}{0.000000, 0.000000, 0.000000}
    \pgfsetstrokecolor{dialinecolor}
    \node[anchor=west] at (46.647799\du,16.368816\du){};
    \pgfsetlinewidth{0.100000\du}
    \pgfsetdash{}{0pt}
    \pgfsetdash{}{0pt}
    \pgfsetmiterjoin
    \pgfsetbuttcap
    {
    \definecolor{dialinecolor}{rgb}{0.396078, 0.345098, 0.960784}
    \pgfsetfillcolor{dialinecolor}
    % was here!!!
    \definecolor{dialinecolor}{rgb}{0.396078, 0.345098, 0.960784}
    \pgfsetstrokecolor{dialinecolor}
    \pgfpathmoveto{\pgfpoint{48.015090\du}{15.670689\du}}
    \pgfpathcurveto{\pgfpoint{47.985360\du}{16.993676\du}}{\pgfpoint{50.032213\du}{16.151419\du}}{\pgfpoint{50.100988\du}{17.011102\du}}
    \pgfusepath{stroke}
    }
    \pgfsetlinewidth{0.100000\du}
    \pgfsetdash{}{0pt}
    \pgfsetdash{}{0pt}
    \pgfsetmiterjoin
    \pgfsetbuttcap
    {
    \definecolor{dialinecolor}{rgb}{0.396078, 0.345098, 0.960784}
    \pgfsetfillcolor{dialinecolor}
    % was here!!!
    \definecolor{dialinecolor}{rgb}{0.396078, 0.345098, 0.960784}
    \pgfsetstrokecolor{dialinecolor}
    \pgfpathmoveto{\pgfpoint{48.013539\du}{15.632609\du}}
    \pgfpathcurveto{\pgfpoint{47.991899\du}{16.965252\du}}{\pgfpoint{45.905734\du}{16.048257\du}}{\pgfpoint{45.894272\du}{16.988177\du}}
    \pgfusepath{stroke}
    }
    % setfont left to latex
    \definecolor{dialinecolor}{rgb}{0.000000, 0.000000, 0.000000}
    \pgfsetstrokecolor{dialinecolor}
    \node at (45.888757\du,17.371821\du){\tiny Apoyo};
    % setfont left to latex
    \definecolor{dialinecolor}{rgb}{0.000000, 0.000000, 0.000000}
    \pgfsetstrokecolor{dialinecolor}
    \node at (45.888757\du,17.795155\du){\tiny insuficiente};
    % setfont left to latex
    \definecolor{dialinecolor}{rgb}{0.000000, 0.000000, 0.000000}
    \pgfsetstrokecolor{dialinecolor}
    \node at (50.112057\du,17.399769\du){\tiny Problemas};
    % setfont left to latex
    \definecolor{dialinecolor}{rgb}{0.000000, 0.000000, 0.000000}
    \pgfsetstrokecolor{dialinecolor}
    \node at (50.112057\du,17.823102\du){\tiny sanitarios};
    % setfont left to latex
    \definecolor{dialinecolor}{rgb}{0.000000, 0.000000, 0.000000}
    \pgfsetstrokecolor{dialinecolor}
    \node at (41.052593\du,14.807772\du){\scriptsize Aguas};
    % setfont left to latex
    \definecolor{dialinecolor}{rgb}{0.000000, 0.000000, 0.000000}
    \pgfsetstrokecolor{dialinecolor}
    \node at (41.052593\du,15.336939\du){\scriptsize turbias};
    \pgfsetlinewidth{0.100000\du}
    \pgfsetdash{}{0pt}
    \pgfsetdash{}{0pt}
    \pgfsetmiterjoin
    \pgfsetbuttcap
    {
    \definecolor{dialinecolor}{rgb}{0.101961, 0.682353, 0.623529}
    \pgfsetfillcolor{dialinecolor}
    % was here!!!
    \definecolor{dialinecolor}{rgb}{0.101961, 0.682353, 0.623529}
    \pgfsetstrokecolor{dialinecolor}
    \pgfpathmoveto{\pgfpoint{41.051566\du}{15.612004\du}}
    \pgfpathcurveto{\pgfpoint{41.021836\du}{16.934992\du}}{\pgfpoint{43.179744\du}{16.052293\du}}{\pgfpoint{43.150014\du}{16.973925\du}}
    \pgfusepath{stroke}
    }
    \pgfsetlinewidth{0.100000\du}
    \pgfsetdash{}{0pt}
    \pgfsetdash{}{0pt}
    \pgfsetmiterjoin
    \pgfsetbuttcap
    {
    \definecolor{dialinecolor}{rgb}{0.101961, 0.682353, 0.623529}
    \pgfsetfillcolor{dialinecolor}
    % was here!!!
    \definecolor{dialinecolor}{rgb}{0.101961, 0.682353, 0.623529}
    \pgfsetstrokecolor{dialinecolor}
    \pgfpathmoveto{\pgfpoint{41.050014\du}{15.573925\du}}
    \pgfpathcurveto{\pgfpoint{41.020284\du}{16.896912\du}}{\pgfpoint{39.079744\du}{16.052293\du}}{\pgfpoint{39.050014\du}{16.973925\du}}
    \pgfusepath{stroke}
    }
    % setfont left to latex
    \definecolor{dialinecolor}{rgb}{0.000000, 0.000000, 0.000000}
    \pgfsetstrokecolor{dialinecolor}
    \node at (39.053553\du,17.532319\du){\tiny Desbordamientos};
    % setfont left to latex
    \definecolor{dialinecolor}{rgb}{0.000000, 0.000000, 0.000000}
    \pgfsetstrokecolor{dialinecolor}
    \node at (43.175107\du,17.346439\du){\tiny Daños};
    % setfont left to latex
    \definecolor{dialinecolor}{rgb}{0.000000, 0.000000, 0.000000}
    \pgfsetstrokecolor{dialinecolor}
    \node at (43.175107\du,17.769773\du){\tiny ambientales};
    \pgfsetlinewidth{0.100000\du}
    \pgfsetdash{}{0pt}
    \pgfsetdash{}{0pt}
    \pgfsetbuttcap
    {
    \definecolor{dialinecolor}{rgb}{0.047059, 0.568627, 0.772549}
    \pgfsetfillcolor{dialinecolor}
    % was here!!!
    \definecolor{dialinecolor}{rgb}{0.047059, 0.568627, 0.772549}
    \pgfsetstrokecolor{dialinecolor}
    \draw (23.908419\du,17.990822\du)--(23.908419\du,19.790129\du);
    }
    \pgfsetlinewidth{0.100000\du}
    \pgfsetdash{}{0pt}
    \pgfsetdash{}{0pt}
    \pgfsetbuttcap
    {
    \definecolor{dialinecolor}{rgb}{0.047059, 0.568627, 0.772549}
    \pgfsetfillcolor{dialinecolor}
    % was here!!!
    \definecolor{dialinecolor}{rgb}{0.047059, 0.568627, 0.772549}
    \pgfsetstrokecolor{dialinecolor}
    \draw (28.106801\du,18.045347\du)--(28.106801\du,19.817391\du);
    }
    % setfont left to latex
    \definecolor{dialinecolor}{rgb}{0.000000, 0.000000, 0.000000}
    \pgfsetstrokecolor{dialinecolor}
    \node at (23.883028\du,20.347956\du){\tiny Alto índice de };
    % setfont left to latex
    \definecolor{dialinecolor}{rgb}{0.000000, 0.000000, 0.000000}
    \pgfsetstrokecolor{dialinecolor}
    \node at (23.883028\du,20.771289\du){\tiny basura arrojadas};
    % setfont left to latex
    \definecolor{dialinecolor}{rgb}{0.000000, 0.000000, 0.000000}
    \pgfsetstrokecolor{dialinecolor}
    \node at (28.103020\du,20.340299\du){\tiny Lecho fluvial};
    % setfont left to latex
    \definecolor{dialinecolor}{rgb}{0.000000, 0.000000, 0.000000}
    \pgfsetstrokecolor{dialinecolor}
    \node at (28.103020\du,20.763632\du){\tiny bloqueado};
    \pgfsetlinewidth{0.100000\du}
    \pgfsetdash{}{0pt}
    \pgfsetdash{}{0pt}
    \pgfsetbuttcap
    {
    \definecolor{dialinecolor}{rgb}{0.741176, 0.203922, 0.819608}
    \pgfsetfillcolor{dialinecolor}
    % was here!!!
    \definecolor{dialinecolor}{rgb}{0.741176, 0.203922, 0.819608}
    \pgfsetstrokecolor{dialinecolor}
    \draw (31.037489\du,18.018085\du)--(31.037489\du,19.803760\du);
    }
    \pgfsetlinewidth{0.100000\du}
    \pgfsetdash{}{0pt}
    \pgfsetdash{}{0pt}
    \pgfsetbuttcap
    {
    \definecolor{dialinecolor}{rgb}{0.741176, 0.203922, 0.819608}
    \pgfsetfillcolor{dialinecolor}
    % was here!!!
    \definecolor{dialinecolor}{rgb}{0.741176, 0.203922, 0.819608}
    \pgfsetstrokecolor{dialinecolor}
    \draw (35.126821\du,18.004453\du)--(35.126821\du,19.817391\du);
    }
    % setfont left to latex
    \definecolor{dialinecolor}{rgb}{0.000000, 0.000000, 0.000000}
    \pgfsetstrokecolor{dialinecolor}
    \node at (31.020337\du,20.330066\du){\tiny Miniría};
    % setfont left to latex
    \definecolor{dialinecolor}{rgb}{0.000000, 0.000000, 0.000000}
    \pgfsetstrokecolor{dialinecolor}
    \node at (31.020337\du,20.753400\du){\tiny ilegal};
    % setfont left to latex
    \definecolor{dialinecolor}{rgb}{0.000000, 0.000000, 0.000000}
    \pgfsetstrokecolor{dialinecolor}
    \node at (35.135218\du,20.353942\du){\tiny Aumento de};
    % setfont left to latex
    \definecolor{dialinecolor}{rgb}{0.000000, 0.000000, 0.000000}
    \pgfsetstrokecolor{dialinecolor}
    \node at (35.135218\du,20.777276\du){\tiny nivel de caudal};
    \pgfsetlinewidth{0.100000\du}
    \pgfsetdash{}{0pt}
    \pgfsetdash{}{0pt}
    \pgfsetbuttcap
    {
    \definecolor{dialinecolor}{rgb}{0.101961, 0.682353, 0.623529}
    \pgfsetfillcolor{dialinecolor}
    % was here!!!
    \definecolor{dialinecolor}{rgb}{0.101961, 0.682353, 0.623529}
    \pgfsetstrokecolor{dialinecolor}
    \draw (39.066211\du,18.018085\du)--(39.066211\du,19.762866\du);
    }
    \pgfsetlinewidth{0.100000\du}
    \pgfsetdash{}{0pt}
    \pgfsetdash{}{0pt}
    \pgfsetbuttcap
    {
    \definecolor{dialinecolor}{rgb}{0.101961, 0.682353, 0.623529}
    \pgfsetfillcolor{dialinecolor}
    % was here!!!
    \definecolor{dialinecolor}{rgb}{0.101961, 0.682353, 0.623529}
    \pgfsetstrokecolor{dialinecolor}
    \draw (43.169174\du,18.031716\du)--(43.169174\du,19.817391\du);
    }
    % setfont left to latex
    \definecolor{dialinecolor}{rgb}{0.000000, 0.000000, 0.000000}
    \pgfsetstrokecolor{dialinecolor}
    \node at (39.060928\du,20.325111\du){\tiny Sedimento};
    % setfont left to latex
    \definecolor{dialinecolor}{rgb}{0.000000, 0.000000, 0.000000}
    \pgfsetstrokecolor{dialinecolor}
    \node at (39.060928\du,20.748444\du){\tiny débil};
    % setfont left to latex
    \definecolor{dialinecolor}{rgb}{0.000000, 0.000000, 0.000000}
    \pgfsetstrokecolor{dialinecolor}
    \node at (43.175809\du,20.348987\du){\tiny Poca atención};
    % setfont left to latex
    \definecolor{dialinecolor}{rgb}{0.000000, 0.000000, 0.000000}
    \pgfsetstrokecolor{dialinecolor}
    \node at (43.175809\du,20.772321\du){\tiny de la población};
    \pgfsetlinewidth{0.100000\du}
    \pgfsetdash{}{0pt}
    \pgfsetdash{}{0pt}
    \pgfsetmiterjoin
    \pgfsetbuttcap
    {
    \definecolor{dialinecolor}{rgb}{0.450980, 0.058824, 0.764706}
    \pgfsetfillcolor{dialinecolor}
    % was here!!!
    \definecolor{dialinecolor}{rgb}{0.450980, 0.058824, 0.764706}
    \pgfsetstrokecolor{dialinecolor}
    \pgfpathmoveto{\pgfpoint{36.185715\du}{10.190703\du}}
    \pgfpathcurveto{\pgfpoint{36.132683\du}{9.077022\du}}{\pgfpoint{31.172566\du}{9.436242\du}}{\pgfpoint{31.187431\du}{8.217318\du}}
    \pgfusepath{stroke}
    }
    % setfont left to latex
    \definecolor{dialinecolor}{rgb}{0.000000, 0.000000, 0.000000}
    \pgfsetstrokecolor{dialinecolor}
    \node at (31.228516\du,7.281573\du){\scriptsize Descontento};
    % setfont left to latex
    \definecolor{dialinecolor}{rgb}{0.000000, 0.000000, 0.000000}
    \pgfsetstrokecolor{dialinecolor}
    \node at (31.228516\du,7.810740\du){\scriptsize social};
    \pgfsetlinewidth{0.100000\du}
    \pgfsetdash{}{0pt}
    \pgfsetdash{}{0pt}
    \pgfsetmiterjoin
    \pgfsetbuttcap
    {
    \definecolor{dialinecolor}{rgb}{0.909804, 0.513726, 0.227451}
    \pgfsetfillcolor{dialinecolor}
    % was here!!!
    \definecolor{dialinecolor}{rgb}{0.909804, 0.513726, 0.227451}
    \pgfsetstrokecolor{dialinecolor}
    \pgfpathmoveto{\pgfpoint{36.722635\du}{10.200639\du}}
    \pgfpathcurveto{\pgfpoint{36.722635\du}{8.847931\du}}{\pgfpoint{35.046268\du}{9.568792\du}}{\pgfpoint{35.031403\du}{8.230949\du}}
    \pgfusepath{stroke}
    }
    % setfont left to latex
    \definecolor{dialinecolor}{rgb}{0.000000, 0.000000, 0.000000}
    \pgfsetstrokecolor{dialinecolor}
    \node at (35.011403\du,7.311303\du){\scriptsize Transparencia};
    % setfont left to latex
    \definecolor{dialinecolor}{rgb}{0.000000, 0.000000, 0.000000}
    \pgfsetstrokecolor{dialinecolor}
    \node at (35.011403\du,7.840470\du){\scriptsize del agua};
    \pgfsetlinewidth{0.100000\du}
    \pgfsetdash{}{0pt}
    \pgfsetdash{}{0pt}
    \pgfsetmiterjoin
    \pgfsetbuttcap
    {
    \definecolor{dialinecolor}{rgb}{0.827451, 0.270588, 0.356863}
    \pgfsetfillcolor{dialinecolor}
    % was here!!!
    \definecolor{dialinecolor}{rgb}{0.827451, 0.270588, 0.356863}
    \pgfsetstrokecolor{dialinecolor}
    \pgfpathmoveto{\pgfpoint{37.792909\du}{10.208071\du}}
    \pgfpathcurveto{\pgfpoint{37.739877\du}{9.094390\du}}{\pgfpoint{42.677221\du}{9.449873\du}}{\pgfpoint{42.692086\du}{8.230949\du}}
    \pgfusepath{stroke}
    }
    \pgfsetlinewidth{0.100000\du}
    \pgfsetdash{}{0pt}
    \pgfsetdash{}{0pt}
    \pgfsetmiterjoin
    \pgfsetbuttcap
    {
    \definecolor{dialinecolor}{rgb}{0.968627, 0.764706, 0.145098}
    \pgfsetfillcolor{dialinecolor}
    % was here!!!
    \definecolor{dialinecolor}{rgb}{0.968627, 0.764706, 0.145098}
    \pgfsetstrokecolor{dialinecolor}
    \pgfpathmoveto{\pgfpoint{37.187972\du}{10.214029\du}}
    \pgfpathcurveto{\pgfpoint{37.187972\du}{8.861321\du}}{\pgfpoint{39.040183\du}{9.514268\du}}{\pgfpoint{39.025318\du}{8.176425\du}}
    \pgfusepath{stroke}
    }
    % setfont left to latex
    \definecolor{dialinecolor}{rgb}{0.000000, 0.000000, 0.000000}
    \pgfsetstrokecolor{dialinecolor}
    \node at (39.010677\du,7.320545\du){\scriptsize Aumento de};
    % setfont left to latex
    \definecolor{dialinecolor}{rgb}{0.000000, 0.000000, 0.000000}
    \pgfsetstrokecolor{dialinecolor}
    \node at (39.010677\du,7.849712\du){\scriptsize las plagas};
    % setfont left to latex
    \definecolor{dialinecolor}{rgb}{0.000000, 0.000000, 0.000000}
    \pgfsetstrokecolor{dialinecolor}
    \node at (42.633431\du,7.323176\du){\scriptsize Disminución de};
    % setfont left to latex
    \definecolor{dialinecolor}{rgb}{0.000000, 0.000000, 0.000000}
    \pgfsetstrokecolor{dialinecolor}
    \node at (42.633431\du,7.852342\du){\scriptsize fuentes potables};
    \pgfsetlinewidth{0.000000\du}
    \pgfsetdash{}{0pt}
    \pgfsetdash{}{0pt}
    \pgfsetmiterjoin
    \definecolor{dialinecolor}{rgb}{1.000000, 1.000000, 1.000000}
    \pgfsetfillcolor{dialinecolor}
    \fill (33.546020\du,10.038947\du)--(33.546020\du,11.611709\du)--(40.509038\du,11.611709\du)--(40.509038\du,10.038947\du)--cycle;
    \definecolor{dialinecolor}{rgb}{1.000000, 1.000000, 1.000000}
    \pgfsetstrokecolor{dialinecolor}
    \draw (33.546020\du,10.038947\du)--(33.546020\du,11.611709\du)--(40.509038\du,11.611709\du)--(40.509038\du,10.038947\du)--cycle;
    % setfont left to latex
    \definecolor{dialinecolor}{rgb}{0.380392, 0.380392, 0.380392}
    \pgfsetstrokecolor{dialinecolor}
    \node at (37.016098\du,10.992318\du){\small \textbf{\textcolor{titulo}{Contaminación Hídrica}}};
    \pgfsetlinewidth{0.100000\du}
    \pgfsetdash{}{0pt}
    \pgfsetdash{}{0pt}
    \pgfsetmiterjoin
    \definecolor{dialinecolor}{rgb}{0.450980, 0.058824, 0.764706}
    \pgfsetfillcolor{dialinecolor}
    \fill (18.366490\du,6.993744\du)--(18.366490\du,8.906754\du)--(22.171489\du,8.906754\du)--(22.171489\du,6.993744\du)--cycle;
    \definecolor{dialinecolor}{rgb}{1.000000, 1.000000, 1.000000}
    \pgfsetstrokecolor{dialinecolor}
    \draw (18.366490\du,6.993744\du)--(18.366490\du,8.906754\du)--(22.171489\du,8.906754\du)--(22.171489\du,6.993744\du)--cycle;
    % setfont left to latex
    \definecolor{dialinecolor}{rgb}{1.000000, 1.000000, 1.000000}
    \pgfsetstrokecolor{dialinecolor}
    \node at (20.268990\du,8.139448\du){\textcolor{white}{Efectos}};
    \pgfsetlinewidth{0.100000\du}
    \pgfsetdash{}{0pt}
    \pgfsetdash{}{0pt}
    \pgfsetbuttcap
    {
    \definecolor{dialinecolor}{rgb}{0.470588, 0.533333, 0.588235}
    \pgfsetfillcolor{dialinecolor}
    % was here!!!
    \pgfsetarrowsend{stealth}
    \definecolor{dialinecolor}{rgb}{0.470588, 0.533333, 0.588235}
    \pgfsetstrokecolor{dialinecolor}
    \draw (20.258479\du,13.006062\du)--(20.258479\du,9.432307\du);
    }
    \pgfsetlinewidth{0.100000\du}
    \pgfsetdash{}{0pt}
    \pgfsetdash{}{0pt}
    \pgfsetmiterjoin
    \definecolor{dialinecolor}{rgb}{0.827451, 0.270588, 0.356863}
    \pgfsetfillcolor{dialinecolor}
    \fill (18.345015\du,13.518548\du)--(18.345015\du,15.431559\du)--(22.150014\du,15.431559\du)--(22.150014\du,13.518548\du)--cycle;
    \definecolor{dialinecolor}{rgb}{1.000000, 1.000000, 1.000000}
    \pgfsetstrokecolor{dialinecolor}
    \draw (18.345015\du,13.518548\du)--(18.345015\du,15.431559\du)--(22.150014\du,15.431559\du)--(22.150014\du,13.518548\du)--cycle;
    % setfont left to latex
    \definecolor{dialinecolor}{rgb}{1.000000, 1.000000, 1.000000}
    \pgfsetstrokecolor{dialinecolor}
    \node at (20.247515\du,14.664252\du){\textcolor{white}{Causas}};
    \pgfsetlinewidth{0.100000\du}
    \pgfsetdash{}{0pt}
    \pgfsetdash{}{0pt}
    \pgfsetbuttcap
    {
    \definecolor{dialinecolor}{rgb}{0.470588, 0.533333, 0.588235}
    \pgfsetfillcolor{dialinecolor}
    % was here!!!
    \pgfsetarrowsstart{stealth}
    \definecolor{dialinecolor}{rgb}{0.470588, 0.533333, 0.588235}
    \pgfsetstrokecolor{dialinecolor}
    \draw (20.188586\du,19.550626\du)--(20.188586\du,15.976870\du);
    }
\end{tikzpicture}  
    A partir del planteamiento del problema de investigación que se llevara acabo en este proyecto se identificaron ciertos problemas que resultan ser causa de la presencia de la Contaminación Hídrica, este problema será de interés central para esta actividad.

    Como se menciona en el planteamiento del problema la causa a este fenómeno radica mayormente en el descuido de la protección ambiental por parte de los habitantes de la ciudad por donde transitan estos afluentes. Las causas importantes que se identifican están relacionadas con la presencia de un gran volumen de residuos en las cuencas, aumento de ciertos agentes que resultan ser amenazas para la calidad del agua, grados altos de turbidez y la mala gestión de las basuras de los habitantes de la ciudad.

    A su vez de estas causas se pueden identificar de cierta manera a partir de unas posibles fuentes que las acontecen, entre las cuales se pueden percibir el alto índice de basuras arrojadas, el lecho fluvial bloqueado, derrame de productos de limpieza, aumento de nivel de caudal, sedimento débil y en particular la poca atención de la población. 

\section{Análisis de Objetivos}
% Graphic for TeX using PGF
% Title: /home/jnicolaschc/GitHub/Metodología de la Investigación/enfoque_marco_logico/Esquemas/Árbol de Objetivos/pgf/arbolObjetivos.dia
% Creator: Dia v0.97+git
% CreationDate: Mon Aug 16 12:03:15 2021
% For: jnicolaschc
% \usepackage{tikz}
% The following commands are not supported in PSTricks at present
% We define them conditionally, so when they are implemented,
% this pgf file will use them.
\begin{figure}[H]
	\ifx\du\undefined
		\newlength{\du}
	\fi
	\setlength{\du}{15\unitlength}
	\begin{tikzpicture}[scale = 1.5]
		\pgftransformxscale{1.000000}
		\pgftransformyscale{-1.000000}
		\definecolor{dialinecolor}{rgb}{0.000000, 0.000000, 0.000000}
		\pgfsetstrokecolor{dialinecolor}
		\pgfsetstrokeopacity{1.000000}
		\definecolor{diafillcolor}{rgb}{1.000000, 1.000000, 1.000000}
		\pgfsetfillcolor{diafillcolor}
		\pgfsetfillopacity{1.000000}
		% setfont left to latex
		\definecolor{dialinecolor}{rgb}{0.000000, 0.000000, 0.000000}
		\pgfsetstrokecolor{dialinecolor}
		\pgfsetstrokeopacity{1.000000}
		\definecolor{diafillcolor}{rgb}{0.000000, 0.000000, 0.000000}
		\pgfsetfillcolor{diafillcolor}
		\pgfsetfillopacity{1.000000}
		\node[anchor=base,inner sep=0pt, outer sep=0pt,color=dialinecolor] at (2.042159\du,-11.237636\du){\tiny{Limpiar los ríos de todo factor}};
		% setfont left to latex
		\definecolor{dialinecolor}{rgb}{0.000000, 0.000000, 0.000000}
		\pgfsetstrokecolor{dialinecolor}
		\pgfsetstrokeopacity{1.000000}
		\definecolor{diafillcolor}{rgb}{0.000000, 0.000000, 0.000000}
		\pgfsetfillcolor{diafillcolor}
		\pgfsetfillopacity{1.000000}
		\node[anchor=base,inner sep=0pt, outer sep=0pt,color=dialinecolor] at (2.042159\du,-10.955414\du){\tiny{contaminante que afecte al agua y}};
		% setfont left to latex
		\definecolor{dialinecolor}{rgb}{0.000000, 0.000000, 0.000000}
		\pgfsetstrokecolor{dialinecolor}
		\pgfsetstrokeopacity{1.000000}
		\definecolor{diafillcolor}{rgb}{0.000000, 0.000000, 0.000000}
		\pgfsetfillcolor{diafillcolor}
		\pgfsetfillopacity{1.000000}
		\node[anchor=base,inner sep=0pt, outer sep=0pt,color=dialinecolor] at (2.042159\du,-10.673192\du){\tiny{generar bienestar ambiental para las}};
		% setfont left to latex
		\definecolor{dialinecolor}{rgb}{0.000000, 0.000000, 0.000000}
		\pgfsetstrokecolor{dialinecolor}
		\pgfsetstrokeopacity{1.000000}
		\definecolor{diafillcolor}{rgb}{0.000000, 0.000000, 0.000000}
		\pgfsetfillcolor{diafillcolor}
		\pgfsetfillopacity{1.000000}
		\node[anchor=base,inner sep=0pt, outer sep=0pt,color=dialinecolor] at (2.042159\du,-10.390970\du){\tiny{zonas circundantes a dichas fuentes de}};
		% setfont left to latex
		\definecolor{dialinecolor}{rgb}{0.000000, 0.000000, 0.000000}
		\pgfsetstrokecolor{dialinecolor}
		\pgfsetstrokeopacity{1.000000}
		\definecolor{diafillcolor}{rgb}{0.000000, 0.000000, 0.000000}
		\pgfsetfillcolor{diafillcolor}
		\pgfsetfillopacity{1.000000}
		\node[anchor=base,inner sep=0pt, outer sep=0pt,color=dialinecolor] at (2.042159\du,-10.108747\du){\tiny{vida}};
		% setfont left to latex
		\definecolor{dialinecolor}{rgb}{0.000000, 0.000000, 0.000000}
		\pgfsetstrokecolor{dialinecolor}
		\pgfsetstrokeopacity{1.000000}
		\definecolor{diafillcolor}{rgb}{0.000000, 0.000000, 0.000000}
		\pgfsetfillcolor{diafillcolor}
		\pgfsetfillopacity{1.000000}
		\node[anchor=base,inner sep=0pt, outer sep=0pt,color=dialinecolor] at (-5.139835\du,-15.063162\du){\tiny{Organizar movimientos de}};
		% setfont left to latex
		\definecolor{dialinecolor}{rgb}{0.000000, 0.000000, 0.000000}
		\pgfsetstrokecolor{dialinecolor}
		\pgfsetstrokeopacity{1.000000}
		\definecolor{diafillcolor}{rgb}{0.000000, 0.000000, 0.000000}
		\pgfsetfillcolor{diafillcolor}
		\pgfsetfillopacity{1.000000}
		\node[anchor=base,inner sep=0pt, outer sep=0pt,color=dialinecolor] at (-5.139835\du,-14.780940\du){\tiny{voluntarios ambientales con}};
		% setfont left to latex
		\definecolor{dialinecolor}{rgb}{0.000000, 0.000000, 0.000000}
		\pgfsetstrokecolor{dialinecolor}
		\pgfsetstrokeopacity{1.000000}
		\definecolor{diafillcolor}{rgb}{0.000000, 0.000000, 0.000000}
		\pgfsetfillcolor{diafillcolor}
		\pgfsetfillopacity{1.000000}
		\node[anchor=base,inner sep=0pt, outer sep=0pt,color=dialinecolor] at (-5.139835\du,-14.498717\du){\tiny{respecto a los ríos mal olientes}};
		% setfont left to latex
		\definecolor{dialinecolor}{rgb}{0.000000, 0.000000, 0.000000}
		\pgfsetstrokecolor{dialinecolor}
		\pgfsetstrokeopacity{1.000000}
		\definecolor{diafillcolor}{rgb}{0.000000, 0.000000, 0.000000}
		\pgfsetfillcolor{diafillcolor}
		\pgfsetfillopacity{1.000000}
		\node[anchor=base,inner sep=0pt, outer sep=0pt,color=dialinecolor] at (-5.139835\du,-14.216495\du){\tiny{a través de universidades y}};
		% setfont left to latex
		\definecolor{dialinecolor}{rgb}{0.000000, 0.000000, 0.000000}
		\pgfsetstrokecolor{dialinecolor}
		\pgfsetstrokeopacity{1.000000}
		\definecolor{diafillcolor}{rgb}{0.000000, 0.000000, 0.000000}
		\pgfsetfillcolor{diafillcolor}
		\pgfsetfillopacity{1.000000}
		\node[anchor=base,inner sep=0pt, outer sep=0pt,color=dialinecolor] at (-5.139835\du,-13.934273\du){\tiny{el gobierno local.}};
		\pgfsetlinewidth{0.050000\du}
		\pgfsetdash{}{0pt}
		\pgfsetmiterjoin
		\pgfsetbuttcap
		{
			\definecolor{diafillcolor}{rgb}{0.172549, 0.533333, 0.850980}
			\pgfsetfillcolor{diafillcolor}
			\pgfsetfillopacity{1.000000}
			% was here!!!
			\definecolor{dialinecolor}{rgb}{0.172549, 0.533333, 0.850980}
			\pgfsetstrokecolor{dialinecolor}
			\pgfsetstrokeopacity{1.000000}
			\pgfpathmoveto{\pgfpoint{-0.184604\du}{-10.920245\du}}
			\pgfpathcurveto{\pgfpoint{-1.330841\du}{-10.931707\du}}{\pgfpoint{-1.479852\du}{-16.788978\du}}{\pgfpoint{-4.987337\du}{-15.459343\du}}
			\pgfusepath{stroke}
		}
		% setfont left to latex
		\definecolor{dialinecolor}{rgb}{0.000000, 0.000000, 0.000000}
		\pgfsetstrokecolor{dialinecolor}
		\pgfsetstrokeopacity{1.000000}
		\definecolor{diafillcolor}{rgb}{0.000000, 0.000000, 0.000000}
		\pgfsetfillcolor{diafillcolor}
		\pgfsetfillopacity{1.000000}
		\node[anchor=base,inner sep=0pt, outer sep=0pt,color=dialinecolor] at (-7.159953\du,-12.314161\du){\tiny{Brindar información clave}};
		% setfont left to latex
		\definecolor{dialinecolor}{rgb}{0.000000, 0.000000, 0.000000}
		\pgfsetstrokecolor{dialinecolor}
		\pgfsetstrokeopacity{1.000000}
		\definecolor{diafillcolor}{rgb}{0.000000, 0.000000, 0.000000}
		\pgfsetfillcolor{diafillcolor}
		\pgfsetfillopacity{1.000000}
		\node[anchor=base,inner sep=0pt, outer sep=0pt,color=dialinecolor] at (-7.159953\du,-12.031939\du){\tiny{que ayude a las personas}};
		% setfont left to latex
		\definecolor{dialinecolor}{rgb}{0.000000, 0.000000, 0.000000}
		\pgfsetstrokecolor{dialinecolor}
		\pgfsetstrokeopacity{1.000000}
		\definecolor{diafillcolor}{rgb}{0.000000, 0.000000, 0.000000}
		\pgfsetfillcolor{diafillcolor}
		\pgfsetfillopacity{1.000000}
		\node[anchor=base,inner sep=0pt, outer sep=0pt,color=dialinecolor] at (-7.159953\du,-11.749717\du){\tiny{a dar un adecuado proceso}};
		% setfont left to latex
		\definecolor{dialinecolor}{rgb}{0.000000, 0.000000, 0.000000}
		\pgfsetstrokecolor{dialinecolor}
		\pgfsetstrokeopacity{1.000000}
		\definecolor{diafillcolor}{rgb}{0.000000, 0.000000, 0.000000}
		\pgfsetfillcolor{diafillcolor}
		\pgfsetfillopacity{1.000000}
		\node[anchor=base,inner sep=0pt, outer sep=0pt,color=dialinecolor] at (-7.159953\du,-11.467494\du){\tiny{de eliminación de este tipo}};
		% setfont left to latex
		\definecolor{dialinecolor}{rgb}{0.000000, 0.000000, 0.000000}
		\pgfsetstrokecolor{dialinecolor}
		\pgfsetstrokeopacity{1.000000}
		\definecolor{diafillcolor}{rgb}{0.000000, 0.000000, 0.000000}
		\pgfsetfillcolor{diafillcolor}
		\pgfsetfillopacity{1.000000}
		\node[anchor=base,inner sep=0pt, outer sep=0pt,color=dialinecolor] at (-7.159953\du,-11.185272\du){\tiny{de basuras.}};
		\pgfsetlinewidth{0.050000\du}
		\pgfsetdash{}{0pt}
		\pgfsetmiterjoin
		\pgfsetbuttcap
		{
			\definecolor{diafillcolor}{rgb}{0.172549, 0.533333, 0.850980}
			\pgfsetfillcolor{diafillcolor}
			\pgfsetfillopacity{1.000000}
			% was here!!!
			\definecolor{dialinecolor}{rgb}{0.172549, 0.533333, 0.850980}
			\pgfsetstrokecolor{dialinecolor}
			\pgfsetstrokeopacity{1.000000}
			\pgfpathmoveto{\pgfpoint{-5.109155\du}{-13.858457\du}}
			\pgfpathcurveto{\pgfpoint{-5.117260\du}{-12.837202\du}}{\pgfpoint{-7.149480\du}{-13.364778\du}}{\pgfpoint{-7.154943\du}{-12.601037\du}}
			\pgfusepath{stroke}
		}
		% setfont left to latex
		\definecolor{dialinecolor}{rgb}{0.000000, 0.000000, 0.000000}
		\pgfsetstrokecolor{dialinecolor}
		\pgfsetstrokeopacity{1.000000}
		\definecolor{diafillcolor}{rgb}{0.000000, 0.000000, 0.000000}
		\pgfsetfillcolor{diafillcolor}
		\pgfsetfillopacity{1.000000}
		\node[anchor=base,inner sep=0pt, outer sep=0pt,color=dialinecolor] at (-3.141461\du,-12.292480\du){\tiny{Realizar reparaciones en los}};
		% setfont left to latex
		\definecolor{dialinecolor}{rgb}{0.000000, 0.000000, 0.000000}
		\pgfsetstrokecolor{dialinecolor}
		\pgfsetstrokeopacity{1.000000}
		\definecolor{diafillcolor}{rgb}{0.000000, 0.000000, 0.000000}
		\pgfsetfillcolor{diafillcolor}
		\pgfsetfillopacity{1.000000}
		\node[anchor=base,inner sep=0pt, outer sep=0pt,color=dialinecolor] at (-3.141461\du,-12.010257\du){\tiny{canales naturales de los}};
		% setfont left to latex
		\definecolor{dialinecolor}{rgb}{0.000000, 0.000000, 0.000000}
		\pgfsetstrokecolor{dialinecolor}
		\pgfsetstrokeopacity{1.000000}
		\definecolor{diafillcolor}{rgb}{0.000000, 0.000000, 0.000000}
		\pgfsetfillcolor{diafillcolor}
		\pgfsetfillopacity{1.000000}
		\node[anchor=base,inner sep=0pt, outer sep=0pt,color=dialinecolor] at (-3.141461\du,-11.728035\du){\tiny{arroyos urbanos para evitar}};
		% setfont left to latex
		\definecolor{dialinecolor}{rgb}{0.000000, 0.000000, 0.000000}
		\pgfsetstrokecolor{dialinecolor}
		\pgfsetstrokeopacity{1.000000}
		\definecolor{diafillcolor}{rgb}{0.000000, 0.000000, 0.000000}
		\pgfsetfillcolor{diafillcolor}
		\pgfsetfillopacity{1.000000}
		\node[anchor=base,inner sep=0pt, outer sep=0pt,color=dialinecolor] at (-3.141461\du,-11.445813\du){\tiny{interrumpir su flujo.}};
		\pgfsetlinewidth{0.050000\du}
		\pgfsetdash{}{0pt}
		\pgfsetmiterjoin
		\pgfsetbuttcap
		{
			\definecolor{diafillcolor}{rgb}{0.172549, 0.533333, 0.850980}
			\pgfsetfillcolor{diafillcolor}
			\pgfsetfillopacity{1.000000}
			% was here!!!
			\definecolor{dialinecolor}{rgb}{0.172549, 0.533333, 0.850980}
			\pgfsetstrokecolor{dialinecolor}
			\pgfsetstrokeopacity{1.000000}
			\pgfpathmoveto{\pgfpoint{-5.101050\du}{-13.834141\du}}
			\pgfpathcurveto{\pgfpoint{-5.076734\du}{-12.893939\du}}{\pgfpoint{-3.108686\du}{-13.236953\du}}{\pgfpoint{-3.092476\du}{-12.588537\du}}
			\pgfusepath{stroke}
		}
		% setfont left to latex
		\definecolor{dialinecolor}{rgb}{0.000000, 0.000000, 0.000000}
		\pgfsetstrokecolor{dialinecolor}
		\pgfsetstrokeopacity{1.000000}
		\definecolor{diafillcolor}{rgb}{0.000000, 0.000000, 0.000000}
		\pgfsetfillcolor{diafillcolor}
		\pgfsetfillopacity{1.000000}
		\node[anchor=base,inner sep=0pt, outer sep=0pt,color=dialinecolor] at (-7.136146\du,-9.535176\du){\tiny{Aumentar la cantidad de señales}};
		% setfont left to latex
		\definecolor{dialinecolor}{rgb}{0.000000, 0.000000, 0.000000}
		\pgfsetstrokecolor{dialinecolor}
		\pgfsetstrokeopacity{1.000000}
		\definecolor{diafillcolor}{rgb}{0.000000, 0.000000, 0.000000}
		\pgfsetfillcolor{diafillcolor}
		\pgfsetfillopacity{1.000000}
		\node[anchor=base,inner sep=0pt, outer sep=0pt,color=dialinecolor] at (-7.136146\du,-9.252953\du){\tiny{informativas cerca a las zonas de}};
		% setfont left to latex
		\definecolor{dialinecolor}{rgb}{0.000000, 0.000000, 0.000000}
		\pgfsetstrokecolor{dialinecolor}
		\pgfsetstrokeopacity{1.000000}
		\definecolor{diafillcolor}{rgb}{0.000000, 0.000000, 0.000000}
		\pgfsetfillcolor{diafillcolor}
		\pgfsetfillopacity{1.000000}
		\node[anchor=base,inner sep=0pt, outer sep=0pt,color=dialinecolor] at (-7.136146\du,-8.970731\du){\tiny{los ríos, en especial los puentes}};
		% setfont left to latex
		\definecolor{dialinecolor}{rgb}{0.000000, 0.000000, 0.000000}
		\pgfsetstrokecolor{dialinecolor}
		\pgfsetstrokeopacity{1.000000}
		\definecolor{diafillcolor}{rgb}{0.000000, 0.000000, 0.000000}
		\pgfsetfillcolor{diafillcolor}
		\pgfsetfillopacity{1.000000}
		\node[anchor=base,inner sep=0pt, outer sep=0pt,color=dialinecolor] at (-7.136146\du,-8.688509\du){\tiny{de transito.}};
		\pgfsetlinewidth{0.050000\du}
		\pgfsetdash{}{0pt}
		\pgfsetbuttcap
		{
			\definecolor{diafillcolor}{rgb}{0.172549, 0.533333, 0.850980}
			\pgfsetfillcolor{diafillcolor}
			\pgfsetfillopacity{1.000000}
			% was here!!!
			\definecolor{dialinecolor}{rgb}{0.172549, 0.533333, 0.850980}
			\pgfsetstrokecolor{dialinecolor}
			\pgfsetstrokeopacity{1.000000}
			\draw (-7.122515\du,-11.011320\du)--(-7.122515\du,-9.777693\du);
		}
		% setfont left to latex
		\definecolor{dialinecolor}{rgb}{0.000000, 0.000000, 0.000000}
		\pgfsetstrokecolor{dialinecolor}
		\pgfsetstrokeopacity{1.000000}
		\definecolor{diafillcolor}{rgb}{0.000000, 0.000000, 0.000000}
		\pgfsetfillcolor{diafillcolor}
		\pgfsetfillopacity{1.000000}
		\node[anchor=base,inner sep=0pt, outer sep=0pt,color=dialinecolor] at (-3.005526\du,-9.572675\du){\tiny{Generar una serie de dirreción}};
		% setfont left to latex
		\definecolor{dialinecolor}{rgb}{0.000000, 0.000000, 0.000000}
		\pgfsetstrokecolor{dialinecolor}
		\pgfsetstrokeopacity{1.000000}
		\definecolor{diafillcolor}{rgb}{0.000000, 0.000000, 0.000000}
		\pgfsetfillcolor{diafillcolor}
		\pgfsetfillopacity{1.000000}
		\node[anchor=base,inner sep=0pt, outer sep=0pt,color=dialinecolor] at (-3.005526\du,-9.290453\du){\tiny{artificial de los ríos que permita la}};
		% setfont left to latex
		\definecolor{dialinecolor}{rgb}{0.000000, 0.000000, 0.000000}
		\pgfsetstrokecolor{dialinecolor}
		\pgfsetstrokeopacity{1.000000}
		\definecolor{diafillcolor}{rgb}{0.000000, 0.000000, 0.000000}
		\pgfsetfillcolor{diafillcolor}
		\pgfsetfillopacity{1.000000}
		\node[anchor=base,inner sep=0pt, outer sep=0pt,color=dialinecolor] at (-3.005526\du,-9.008231\du){\tiny{circulación normal para evitar}};
		% setfont left to latex
		\definecolor{dialinecolor}{rgb}{0.000000, 0.000000, 0.000000}
		\pgfsetstrokecolor{dialinecolor}
		\pgfsetstrokeopacity{1.000000}
		\definecolor{diafillcolor}{rgb}{0.000000, 0.000000, 0.000000}
		\pgfsetfillcolor{diafillcolor}
		\pgfsetfillopacity{1.000000}
		\node[anchor=base,inner sep=0pt, outer sep=0pt,color=dialinecolor] at (-3.005526\du,-8.726009\du){\tiny{estancamientos.}};
		\pgfsetlinewidth{0.050000\du}
		\pgfsetdash{}{0pt}
		\pgfsetbuttcap
		{
			\definecolor{diafillcolor}{rgb}{0.172549, 0.533333, 0.850980}
			\pgfsetfillcolor{diafillcolor}
			\pgfsetfillopacity{1.000000}
			% was here!!!
			\definecolor{dialinecolor}{rgb}{0.172549, 0.533333, 0.850980}
			\pgfsetstrokecolor{dialinecolor}
			\pgfsetstrokeopacity{1.000000}
			\draw (-3.053020\du,-11.156766\du)--(-3.054976\du,-9.838559\du);
		}
		\pgfsetlinewidth{0.050000\du}
		\pgfsetdash{}{0pt}
		\pgfsetmiterjoin
		\pgfsetbuttcap
		{
			\definecolor{diafillcolor}{rgb}{0.776471, 0.317647, 0.843137}
			\pgfsetfillcolor{diafillcolor}
			\pgfsetfillopacity{1.000000}
			% was here!!!
			\definecolor{dialinecolor}{rgb}{0.776471, 0.317647, 0.843137}
			\pgfsetstrokecolor{dialinecolor}
			\pgfsetstrokeopacity{1.000000}
			\pgfpathmoveto{\pgfpoint{1.855315\du}{-9.860305\du}}
			\pgfpathcurveto{\pgfpoint{1.792815\du}{-9.085311\du}}{\pgfpoint{-1.957155\du}{-8.735314\du}}{\pgfpoint{-2.007154\du}{-8.016570\du}}
			\pgfusepath{stroke}
		}
		% setfont left to latex
		\definecolor{dialinecolor}{rgb}{0.000000, 0.000000, 0.000000}
		\pgfsetstrokecolor{dialinecolor}
		\pgfsetstrokeopacity{1.000000}
		\definecolor{diafillcolor}{rgb}{0.000000, 0.000000, 0.000000}
		\pgfsetfillcolor{diafillcolor}
		\pgfsetfillopacity{1.000000}
		\node[anchor=base,inner sep=0pt, outer sep=0pt,color=dialinecolor] at (-2.010510\du,-7.697561\du){\tiny{Tomar muestras para estudiar}};
		% setfont left to latex
		\definecolor{dialinecolor}{rgb}{0.000000, 0.000000, 0.000000}
		\pgfsetstrokecolor{dialinecolor}
		\pgfsetstrokeopacity{1.000000}
		\definecolor{diafillcolor}{rgb}{0.000000, 0.000000, 0.000000}
		\pgfsetfillcolor{diafillcolor}
		\pgfsetfillopacity{1.000000}
		\node[anchor=base,inner sep=0pt, outer sep=0pt,color=dialinecolor] at (-2.010510\du,-7.415339\du){\tiny{e investigar que agentes}};
		% setfont left to latex
		\definecolor{dialinecolor}{rgb}{0.000000, 0.000000, 0.000000}
		\pgfsetstrokecolor{dialinecolor}
		\pgfsetstrokeopacity{1.000000}
		\definecolor{diafillcolor}{rgb}{0.000000, 0.000000, 0.000000}
		\pgfsetfillcolor{diafillcolor}
		\pgfsetfillopacity{1.000000}
		\node[anchor=base,inner sep=0pt, outer sep=0pt,color=dialinecolor] at (-2.010510\du,-7.133116\du){\tiny{bacterianos están presentes en}};
		% setfont left to latex
		\definecolor{dialinecolor}{rgb}{0.000000, 0.000000, 0.000000}
		\pgfsetstrokecolor{dialinecolor}
		\pgfsetstrokeopacity{1.000000}
		\definecolor{diafillcolor}{rgb}{0.000000, 0.000000, 0.000000}
		\pgfsetfillcolor{diafillcolor}
		\pgfsetfillopacity{1.000000}
		\node[anchor=base,inner sep=0pt, outer sep=0pt,color=dialinecolor] at (-2.010510\du,-6.850894\du){\tiny{los cuerpos de agua de la ciudad.}};
		\pgfsetlinewidth{0.050000\du}
		\pgfsetdash{}{0pt}
		\pgfsetmiterjoin
		\pgfsetbuttcap{
			\definecolor{diafillcolor}{rgb}{0.776471, 0.317647, 0.843137}
			\pgfsetfillcolor{diafillcolor}
			\pgfsetfillopacity{1.000000}
			% was here!!!
			\definecolor{dialinecolor}{rgb}{0.776471, 0.317647, 0.843137}
			\pgfsetstrokecolor{dialinecolor}
			\pgfsetstrokeopacity{1.000000}
			\pgfpathmoveto{\pgfpoint{-1.993995\du}{-6.600722\du}}
			\pgfpathcurveto{\pgfpoint{-2.002100\du}{-5.579467\du}}{\pgfpoint{-4.034320\du}{-6.107043\du}}{\pgfpoint{-4.039783\du}{-5.343302\du}}
			\pgfusepath{stroke}
		}
		\pgfsetlinewidth{0.050000\du}
		\pgfsetdash{}{0pt}
		\pgfsetmiterjoin
		\pgfsetbuttcap
		{
			\definecolor{diafillcolor}{rgb}{0.776471, 0.317647, 0.843137}
			\pgfsetfillcolor{diafillcolor}
			\pgfsetfillopacity{1.000000}
			% was here!!!
			\definecolor{dialinecolor}{rgb}{0.776471, 0.317647, 0.843137}
			\pgfsetstrokecolor{dialinecolor}
			\pgfsetstrokeopacity{1.000000}
			\pgfpathmoveto{\pgfpoint{-1.985889\du}{-6.576406\du}}
			\pgfpathcurveto{\pgfpoint{-1.961574\du}{-5.636203\du}}{\pgfpoint{0.006474\du}{-5.979218\du}}{\pgfpoint{0.022685\du}{-5.330802\du}}
			\pgfusepath{stroke}
		}
		% setfont left to latex
		\definecolor{dialinecolor}{rgb}{0.000000, 0.000000, 0.000000}
		\pgfsetstrokecolor{dialinecolor}
		\pgfsetstrokeopacity{1.000000}
		\definecolor{diafillcolor}{rgb}{0.000000, 0.000000, 0.000000}
		\pgfsetfillcolor{diafillcolor}
		\pgfsetfillopacity{1.000000}
		\node[anchor=base,inner sep=0pt, outer sep=0pt,color=dialinecolor] at (-4.085599\du,-4.993030\du){\tiny{Aspirar a aplicar una ley que}};
		% setfont left to latex
		\definecolor{dialinecolor}{rgb}{0.000000, 0.000000, 0.000000}
		\pgfsetstrokecolor{dialinecolor}
		\pgfsetstrokeopacity{1.000000}
		\definecolor{diafillcolor}{rgb}{0.000000, 0.000000, 0.000000}
		\pgfsetfillcolor{diafillcolor}
		\pgfsetfillopacity{1.000000}
		\node[anchor=base,inner sep=0pt, outer sep=0pt,color=dialinecolor] at (-4.085599\du,-4.710807\du){\tiny{existe en países desarrollasdos}};
		% setfont left to latex
		\definecolor{dialinecolor}{rgb}{0.000000, 0.000000, 0.000000}
		\pgfsetstrokecolor{dialinecolor}
		\pgfsetstrokeopacity{1.000000}
		\definecolor{diafillcolor}{rgb}{0.000000, 0.000000, 0.000000}
		\pgfsetfillcolor{diafillcolor}
		\pgfsetfillopacity{1.000000}
		\node[anchor=base,inner sep=0pt, outer sep=0pt,color=dialinecolor] at (-4.085599\du,-4.428585\du){\tiny{en nuestro país para el control de}};
		% setfont left to latex
		\definecolor{dialinecolor}{rgb}{0.000000, 0.000000, 0.000000}
		\pgfsetstrokecolor{dialinecolor}
		\pgfsetstrokeopacity{1.000000}
		\definecolor{diafillcolor}{rgb}{0.000000, 0.000000, 0.000000}
		\pgfsetfillcolor{diafillcolor}
		\pgfsetfillopacity{1.000000}
		\node[anchor=base,inner sep=0pt, outer sep=0pt,color=dialinecolor] at (-4.085599\du,-4.146363\du){\tiny{estos contaminantes.}};
		% setfont left to latex
		\definecolor{dialinecolor}{rgb}{0.000000, 0.000000, 0.000000}
		\pgfsetstrokecolor{dialinecolor}
		\pgfsetstrokeopacity{1.000000}
		\definecolor{diafillcolor}{rgb}{0.000000, 0.000000, 0.000000}
		\pgfsetfillcolor{diafillcolor}
		\pgfsetfillopacity{1.000000}
		\node[anchor=base,inner sep=0pt, outer sep=0pt,color=dialinecolor] at (0.015199\du,-5.039907\du){\tiny{Utilizar un sistema que tenga}};
		% setfont left to latex
		\definecolor{dialinecolor}{rgb}{0.000000, 0.000000, 0.000000}
		\pgfsetstrokecolor{dialinecolor}
		\pgfsetstrokeopacity{1.000000}
		\definecolor{diafillcolor}{rgb}{0.000000, 0.000000, 0.000000}
		\pgfsetfillcolor{diafillcolor}
		\pgfsetfillopacity{1.000000}
		\node[anchor=base,inner sep=0pt, outer sep=0pt,color=dialinecolor] at (0.015199\du,-4.757685\du){\tiny{la capacidad de filtrar y gestionar}};
		% setfont left to latex
		\definecolor{dialinecolor}{rgb}{0.000000, 0.000000, 0.000000}
		\pgfsetstrokecolor{dialinecolor}
		\pgfsetstrokeopacity{1.000000}
		\definecolor{diafillcolor}{rgb}{0.000000, 0.000000, 0.000000}
		\pgfsetfillcolor{diafillcolor}
		\pgfsetfillopacity{1.000000}
		\node[anchor=base,inner sep=0pt, outer sep=0pt,color=dialinecolor] at (0.015199\du,-4.475463\du){\tiny{los desechos macroplásticos y}};
		% setfont left to latex
		\definecolor{dialinecolor}{rgb}{0.000000, 0.000000, 0.000000}
		\pgfsetstrokecolor{dialinecolor}
		\pgfsetstrokeopacity{1.000000}
		\definecolor{diafillcolor}{rgb}{0.000000, 0.000000, 0.000000}
		\pgfsetfillcolor{diafillcolor}
		\pgfsetfillopacity{1.000000}
		\node[anchor=base,inner sep=0pt, outer sep=0pt,color=dialinecolor] at (0.015199\du,-4.193241\du){\tiny{microplásticos que ponen en peligro}};
		% setfont left to latex
		\definecolor{dialinecolor}{rgb}{0.000000, 0.000000, 0.000000}
		\pgfsetstrokecolor{dialinecolor}
		\pgfsetstrokeopacity{1.000000}
		\definecolor{diafillcolor}{rgb}{0.000000, 0.000000, 0.000000}
		\pgfsetfillcolor{diafillcolor}
		\pgfsetfillopacity{1.000000}
		\node[anchor=base,inner sep=0pt, outer sep=0pt,color=dialinecolor] at (0.015199\du,-3.911018\du){\tiny{a las especies del lugar.}};
		\pgfsetlinewidth{0.050000\du}
		\pgfsetdash{}{0pt}
		\pgfsetbuttcap{
			\definecolor{diafillcolor}{rgb}{0.776471, 0.317647, 0.843137}
			\pgfsetfillcolor{diafillcolor}
			\pgfsetfillopacity{1.000000}
			% was here!!!
			\definecolor{dialinecolor}{rgb}{0.776471, 0.317647, 0.843137}
			\pgfsetstrokecolor{dialinecolor}
			\pgfsetstrokeopacity{1.000000}
			\draw (-4.053133\du,-3.920434\du)--(-4.043127\du,-2.543587\du);
		}
		\pgfsetlinewidth{0.050000\du}
		\pgfsetdash{}{0pt}
		\pgfsetbuttcap
		{
			\definecolor{diafillcolor}{rgb}{0.776471, 0.317647, 0.843137}
			\pgfsetfillcolor{diafillcolor}
			\pgfsetfillopacity{1.000000}
			% was here!!!
			\definecolor{dialinecolor}{rgb}{0.776471, 0.317647, 0.843137}
			\pgfsetstrokecolor{dialinecolor}
			\pgfsetstrokeopacity{1.000000}
			\draw (0.026368\du,-3.854504\du)--(0.021963\du,-2.470507\du);
		}
		% setfont left to latex
		\definecolor{dialinecolor}{rgb}{0.000000, 0.000000, 0.000000}
		\pgfsetstrokecolor{dialinecolor}
		\pgfsetstrokeopacity{1.000000}
		\definecolor{diafillcolor}{rgb}{0.000000, 0.000000, 0.000000}
		\pgfsetfillcolor{diafillcolor}
		\pgfsetfillopacity{1.000000}
		\node[anchor=base,inner sep=0pt, outer sep=0pt,color=dialinecolor] at (-4.044092\du,-2.318979\du){\tiny{Generar un proyecto de un}};
		% setfont left to latex
		\definecolor{dialinecolor}{rgb}{0.000000, 0.000000, 0.000000}
		\pgfsetstrokecolor{dialinecolor}
		\pgfsetstrokeopacity{1.000000}
		\definecolor{diafillcolor}{rgb}{0.000000, 0.000000, 0.000000}
		\pgfsetfillcolor{diafillcolor}
		\pgfsetfillopacity{1.000000}
		\node[anchor=base,inner sep=0pt, outer sep=0pt,color=dialinecolor] at (-4.044092\du,-2.036757\du){\tiny{decreto para el buen uso de}};
		% setfont left to latex
		\definecolor{dialinecolor}{rgb}{0.000000, 0.000000, 0.000000}
		\pgfsetstrokecolor{dialinecolor}
		\pgfsetstrokeopacity{1.000000}
		\definecolor{diafillcolor}{rgb}{0.000000, 0.000000, 0.000000}
		\pgfsetfillcolor{diafillcolor}
		\pgfsetfillopacity{1.000000}
		\node[anchor=base,inner sep=0pt, outer sep=0pt,color=dialinecolor] at (-4.044092\du,-1.754535\du){\tiny{las aguas residuales según la}};
		% setfont left to latex
		\definecolor{dialinecolor}{rgb}{0.000000, 0.000000, 0.000000}
		\pgfsetstrokecolor{dialinecolor}
		\pgfsetstrokeopacity{1.000000}
		\definecolor{diafillcolor}{rgb}{0.000000, 0.000000, 0.000000}
		\pgfsetfillcolor{diafillcolor}
		\pgfsetfillopacity{1.000000}
		\node[anchor=base,inner sep=0pt, outer sep=0pt,color=dialinecolor] at (-4.044092\du,-1.472313\du){\tiny{actividad que se este realizando.}};
		% setfont left to latex
		\definecolor{dialinecolor}{rgb}{0.000000, 0.000000, 0.000000}
		\pgfsetstrokecolor{dialinecolor}
		\pgfsetstrokeopacity{1.000000}
		\definecolor{diafillcolor}{rgb}{0.000000, 0.000000, 0.000000}
		\pgfsetfillcolor{diafillcolor}
		\pgfsetfillopacity{1.000000}
		\node[anchor=base,inner sep=0pt, outer sep=0pt,color=dialinecolor] at (0.032391\du,-2.229089\du){\tiny{Extraer basuras del fondo de los}};
		% setfont left to latex
		\definecolor{dialinecolor}{rgb}{0.000000, 0.000000, 0.000000}
		\pgfsetstrokecolor{dialinecolor}
		\pgfsetstrokeopacity{1.000000}
		\definecolor{diafillcolor}{rgb}{0.000000, 0.000000, 0.000000}
		\pgfsetfillcolor{diafillcolor}
		\pgfsetfillopacity{1.000000}
		\node[anchor=base,inner sep=0pt, outer sep=0pt,color=dialinecolor] at (0.032391\du,-1.946867\du){\tiny{ríos para que el nivel del agua no}};
		% setfont left to latex
		\definecolor{dialinecolor}{rgb}{0.000000, 0.000000, 0.000000}
		\pgfsetstrokecolor{dialinecolor}
		\pgfsetstrokeopacity{1.000000}
		\definecolor{diafillcolor}{rgb}{0.000000, 0.000000, 0.000000}
		\pgfsetfillcolor{diafillcolor}
		\pgfsetfillopacity{1.000000}
		\node[anchor=base,inner sep=0pt, outer sep=0pt,color=dialinecolor] at (0.032391\du,-1.664644\du){\tiny{suba y permita que se desborde.}};
		\pgfsetlinewidth{0.050000\du}
		\pgfsetdash{}{0pt}
		\pgfsetmiterjoin
		\pgfsetbuttcap
		{
			\definecolor{diafillcolor}{rgb}{0.101961, 0.682353, 0.623529}
			\pgfsetfillcolor{diafillcolor}
			\pgfsetfillopacity{1.000000}
			% was here!!!
			\definecolor{dialinecolor}{rgb}{0.101961, 0.682353, 0.623529}
			\pgfsetstrokecolor{dialinecolor}
			\pgfsetstrokeopacity{1.000000}
			\pgfpathmoveto{\pgfpoint{2.213970\du}{-9.868533\du}}
			\pgfpathcurveto{\pgfpoint{2.213970\du}{-9.081885\du}}{\pgfpoint{5.767144\du}{-8.790207\du}}{\pgfpoint{5.988113\du}{-8.047753\du}}
			\pgfusepath{stroke}
		}
		% setfont left to latex
		\definecolor{dialinecolor}{rgb}{0.000000, 0.000000, 0.000000}
		\pgfsetstrokecolor{dialinecolor}
		\pgfsetstrokeopacity{1.000000}
		\definecolor{diafillcolor}{rgb}{0.000000, 0.000000, 0.000000}
		\pgfsetfillcolor{diafillcolor}
		\pgfsetfillopacity{1.000000}
		\node[anchor=base,inner sep=0pt, outer sep=0pt,color=dialinecolor] at (5.996065\du,-7.785150\du){\tiny{Aplicar sistemas de monitoreo}};
		% setfont left to latex
		\definecolor{dialinecolor}{rgb}{0.000000, 0.000000, 0.000000}
		\pgfsetstrokecolor{dialinecolor}
		\pgfsetstrokeopacity{1.000000}
		\definecolor{diafillcolor}{rgb}{0.000000, 0.000000, 0.000000}
		\pgfsetfillcolor{diafillcolor}
		\pgfsetfillopacity{1.000000}
		\node[anchor=base,inner sep=0pt, outer sep=0pt,color=dialinecolor] at (5.996065\du,-7.502928\du){\tiny{para medir en continuo los valores}};
		% setfont left to latex
		\definecolor{dialinecolor}{rgb}{0.000000, 0.000000, 0.000000}
		\pgfsetstrokecolor{dialinecolor}
		\pgfsetstrokeopacity{1.000000}
		\definecolor{diafillcolor}{rgb}{0.000000, 0.000000, 0.000000}
		\pgfsetfillcolor{diafillcolor}
		\pgfsetfillopacity{1.000000}
		\node[anchor=base,inner sep=0pt, outer sep=0pt,color=dialinecolor] at (5.996065\du,-7.220706\du){\tiny{de turbidez y fijar un control de}};
		% setfont left to latex
		\definecolor{dialinecolor}{rgb}{0.000000, 0.000000, 0.000000}
		\pgfsetstrokecolor{dialinecolor}
		\pgfsetstrokeopacity{1.000000}
		\definecolor{diafillcolor}{rgb}{0.000000, 0.000000, 0.000000}
		\pgfsetfillcolor{diafillcolor}
		\pgfsetfillopacity{1.000000}
		\node[anchor=base,inner sep=0pt, outer sep=0pt,color=dialinecolor] at (5.996065\du,-6.938483\du){\tiny{estos valores para contrarrestar el}};
		% setfont left to latex
		\definecolor{dialinecolor}{rgb}{0.000000, 0.000000, 0.000000}
		\pgfsetstrokecolor{dialinecolor}
		\pgfsetstrokeopacity{1.000000}
		\definecolor{diafillcolor}{rgb}{0.000000, 0.000000, 0.000000}
		\pgfsetfillcolor{diafillcolor}
		\pgfsetfillopacity{1.000000}
		\node[anchor=base,inner sep=0pt, outer sep=0pt,color=dialinecolor] at (5.996065\du,-6.656261\du){\tiny{fenómeno.}};
		\pgfsetlinewidth{0.050000\du}
		\pgfsetdash{}{0pt}
		\pgfsetmiterjoin
		\pgfsetbuttcap
		{
			\definecolor{diafillcolor}{rgb}{0.101961, 0.682353, 0.623529}
			\pgfsetfillcolor{diafillcolor}
			\pgfsetfillopacity{1.000000}
			% was here!!!
			\definecolor{dialinecolor}{rgb}{0.101961, 0.682353, 0.623529}
			\pgfsetstrokecolor{dialinecolor}
			\pgfsetstrokeopacity{1.000000}
			\pgfpathmoveto{\pgfpoint{6.033239\du}{-6.599518\du}}
			\pgfpathcurveto{\pgfpoint{6.025134\du}{-5.578263\du}}{\pgfpoint{3.992914\du}{-6.105839\du}}{\pgfpoint{3.987451\du}{-5.342098\du}}
			\pgfusepath{stroke}
		}
		\pgfsetlinewidth{0.050000\du}
		\pgfsetdash{}{0pt}
		\pgfsetmiterjoin
		\pgfsetbuttcap
		{
			\definecolor{diafillcolor}{rgb}{0.101961, 0.682353, 0.623529}
			\pgfsetfillcolor{diafillcolor}
			\pgfsetfillopacity{1.000000}
			% was here!!!
			\definecolor{dialinecolor}{rgb}{0.101961, 0.682353, 0.623529}
			\pgfsetstrokecolor{dialinecolor}
			\pgfsetstrokeopacity{1.000000}
			\pgfpathmoveto{\pgfpoint{6.041345\du}{-6.575202\du}}
			\pgfpathcurveto{\pgfpoint{6.065660\du}{-5.635000\du}}{\pgfpoint{8.033708\du}{-5.978014\du}}{\pgfpoint{8.049919\du}{-5.329598\du}}
			\pgfusepath{stroke}
		}
		% setfont left to latex
		\definecolor{dialinecolor}{rgb}{0.000000, 0.000000, 0.000000}
		\pgfsetstrokecolor{dialinecolor}
		\pgfsetstrokeopacity{1.000000}
		\definecolor{diafillcolor}{rgb}{0.000000, 0.000000, 0.000000}
		\pgfsetfillcolor{diafillcolor}
		\pgfsetfillopacity{1.000000}
		\node[anchor=base,inner sep=0pt, outer sep=0pt,color=dialinecolor] at (3.990096\du,-5.016473\du){\tiny{Disminuir los desechos en los}};
		% setfont left to latex
		\definecolor{dialinecolor}{rgb}{0.000000, 0.000000, 0.000000}
		\pgfsetstrokecolor{dialinecolor}
		\pgfsetstrokeopacity{1.000000}
		\definecolor{diafillcolor}{rgb}{0.000000, 0.000000, 0.000000}
		\pgfsetfillcolor{diafillcolor}
		\pgfsetfillopacity{1.000000}
		\node[anchor=base,inner sep=0pt, outer sep=0pt,color=dialinecolor] at (3.990096\du,-4.734251\du){\tiny{ríos y fortificar las zonas más}};
		% setfont left to latex
		\definecolor{dialinecolor}{rgb}{0.000000, 0.000000, 0.000000}
		\pgfsetstrokecolor{dialinecolor}
		\pgfsetstrokeopacity{1.000000}
		\definecolor{diafillcolor}{rgb}{0.000000, 0.000000, 0.000000}
		\pgfsetfillcolor{diafillcolor}
		\pgfsetfillopacity{1.000000}
		\node[anchor=base,inner sep=0pt, outer sep=0pt,color=dialinecolor] at (3.990096\du,-4.452029\du){\tiny{débiles que pueden tender a}};
		% setfont left to latex
		\definecolor{dialinecolor}{rgb}{0.000000, 0.000000, 0.000000}
		\pgfsetstrokecolor{dialinecolor}
		\pgfsetstrokeopacity{1.000000}
		\definecolor{diafillcolor}{rgb}{0.000000, 0.000000, 0.000000}
		\pgfsetfillcolor{diafillcolor}
		\pgfsetfillopacity{1.000000}
		\node[anchor=base,inner sep=0pt, outer sep=0pt,color=dialinecolor] at (3.990096\du,-4.169807\du){\tiny{producirse un desbordamiento.}};
		% setfont left to latex
		\definecolor{dialinecolor}{rgb}{0.000000, 0.000000, 0.000000}
		\pgfsetstrokecolor{dialinecolor}
		\pgfsetstrokeopacity{1.000000}
		\definecolor{diafillcolor}{rgb}{0.000000, 0.000000, 0.000000}
		\pgfsetfillcolor{diafillcolor}
		\pgfsetfillopacity{1.000000}
		\node[anchor=base,inner sep=0pt, outer sep=0pt,color=dialinecolor] at (8.054149\du,-5.014205\du){\tiny{Las limpiezas de las zonas verdes}};
		% setfont left to latex
		\definecolor{dialinecolor}{rgb}{0.000000, 0.000000, 0.000000}
		\pgfsetstrokecolor{dialinecolor}
		\pgfsetstrokeopacity{1.000000}
		\definecolor{diafillcolor}{rgb}{0.000000, 0.000000, 0.000000}
		\pgfsetfillcolor{diafillcolor}
		\pgfsetfillopacity{1.000000}
		\node[anchor=base,inner sep=0pt, outer sep=0pt,color=dialinecolor] at (8.054149\du,-4.731983\du){\tiny{cercanas a los ríos y arroyos de}};
		% setfont left to latex
		\definecolor{dialinecolor}{rgb}{0.000000, 0.000000, 0.000000}
		\pgfsetstrokecolor{dialinecolor}
		\pgfsetstrokeopacity{1.000000}
		\definecolor{diafillcolor}{rgb}{0.000000, 0.000000, 0.000000}
		\pgfsetfillcolor{diafillcolor}
		\pgfsetfillopacity{1.000000}
		\node[anchor=base,inner sep=0pt, outer sep=0pt,color=dialinecolor] at (8.054149\du,-4.449760\du){\tiny{las ciudad son importantes para}};
		% setfont left to latex
		\definecolor{dialinecolor}{rgb}{0.000000, 0.000000, 0.000000}
		\pgfsetstrokecolor{dialinecolor}
		\pgfsetstrokeopacity{1.000000}
		\definecolor{diafillcolor}{rgb}{0.000000, 0.000000, 0.000000}
		\pgfsetfillcolor{diafillcolor}
		\pgfsetfillopacity{1.000000}
		\node[anchor=base,inner sep=0pt, outer sep=0pt,color=dialinecolor] at (8.054149\du,-4.167538\du){\tiny{prevenir contaminaciones indirectas.}};
		\pgfsetlinewidth{0.050000\du}
		\pgfsetdash{}{0pt}
		\pgfsetbuttcap
		{
			\definecolor{diafillcolor}{rgb}{0.101961, 0.682353, 0.623529}
			\pgfsetfillcolor{diafillcolor}
			\pgfsetfillopacity{1.000000}
			% was here!!!
			\definecolor{dialinecolor}{rgb}{0.101961, 0.682353, 0.623529}
			\pgfsetstrokecolor{dialinecolor}
			\pgfsetstrokeopacity{1.000000}
			\draw (3.996293\du,-4.058397\du)--(4.006299\du,-2.681549\du);
		}
		\pgfsetlinewidth{0.050000\du}
		\pgfsetdash{}{0pt}
		\pgfsetbuttcap
		{
			\definecolor{diafillcolor}{rgb}{0.101961, 0.682353, 0.623529}
			\pgfsetfillcolor{diafillcolor}
			\pgfsetfillopacity{1.000000}
			% was here!!!
			\definecolor{dialinecolor}{rgb}{0.101961, 0.682353, 0.623529}
			\pgfsetstrokecolor{dialinecolor}
			\pgfsetstrokeopacity{1.000000}
			\draw (8.075794\du,-4.080853\du)--(8.071389\du,-2.696857\du);
		}
		% setfont left to latex
		\definecolor{dialinecolor}{rgb}{0.000000, 0.000000, 0.000000}
		\pgfsetstrokecolor{dialinecolor}
		\pgfsetstrokeopacity{1.000000}
		\definecolor{diafillcolor}{rgb}{0.000000, 0.000000, 0.000000}
		\pgfsetfillcolor{diafillcolor}
		\pgfsetfillopacity{1.000000}
		\node[anchor=base,inner sep=0pt, outer sep=0pt,color=dialinecolor] at (4.004964\du,-2.391367\du){\tiny{Tener control acerca del suelo}};
		% setfont left to latex
		\definecolor{dialinecolor}{rgb}{0.000000, 0.000000, 0.000000}
		\pgfsetstrokecolor{dialinecolor}
		\pgfsetstrokeopacity{1.000000}
		\definecolor{diafillcolor}{rgb}{0.000000, 0.000000, 0.000000}
		\pgfsetfillcolor{diafillcolor}
		\pgfsetfillopacity{1.000000}
		\node[anchor=base,inner sep=0pt, outer sep=0pt,color=dialinecolor] at (4.004964\du,-2.109145\du){\tiny{en los cuales los ríos circulan}};
		% setfont left to latex
		\definecolor{dialinecolor}{rgb}{0.000000, 0.000000, 0.000000}
		\pgfsetstrokecolor{dialinecolor}
		\pgfsetstrokeopacity{1.000000}
		\definecolor{diafillcolor}{rgb}{0.000000, 0.000000, 0.000000}
		\pgfsetfillcolor{diafillcolor}
		\pgfsetfillopacity{1.000000}
		\node[anchor=base,inner sep=0pt, outer sep=0pt,color=dialinecolor] at (4.004964\du,-1.826923\du){\tiny{para evitar deslizamientos}};
		% setfont left to latex
		\definecolor{dialinecolor}{rgb}{0.000000, 0.000000, 0.000000}
		\pgfsetstrokecolor{dialinecolor}
		\pgfsetstrokeopacity{1.000000}
		\definecolor{diafillcolor}{rgb}{0.000000, 0.000000, 0.000000}
		\pgfsetfillcolor{diafillcolor}
		\pgfsetfillopacity{1.000000}
		\node[anchor=base,inner sep=0pt, outer sep=0pt,color=dialinecolor] at (4.004964\du,-1.544701\du){\tiny{que afecten otros factores.}};
		% setfont left to latex
		\definecolor{dialinecolor}{rgb}{0.000000, 0.000000, 0.000000}
		\pgfsetstrokecolor{dialinecolor}
		\pgfsetstrokeopacity{1.000000}
		\definecolor{diafillcolor}{rgb}{0.000000, 0.000000, 0.000000}
		\pgfsetfillcolor{diafillcolor}
		\pgfsetfillopacity{1.000000}
		\node[anchor=base,inner sep=0pt, outer sep=0pt,color=dialinecolor] at (8.077856\du,-2.391367\du){\tiny{Usar los medios de comunicación}};
		% setfont left to latex
		\definecolor{dialinecolor}{rgb}{0.000000, 0.000000, 0.000000}
		\pgfsetstrokecolor{dialinecolor}
		\pgfsetstrokeopacity{1.000000}
		\definecolor{diafillcolor}{rgb}{0.000000, 0.000000, 0.000000}
		\pgfsetfillcolor{diafillcolor}
		\pgfsetfillopacity{1.000000}
		\node[anchor=base,inner sep=0pt, outer sep=0pt,color=dialinecolor] at (8.077856\du,-2.109145\du){\tiny{para generar mayor conciencia en las}};
		% setfont left to latex
		\definecolor{dialinecolor}{rgb}{0.000000, 0.000000, 0.000000}
		\pgfsetstrokecolor{dialinecolor}
		\pgfsetstrokeopacity{1.000000}
		\definecolor{diafillcolor}{rgb}{0.000000, 0.000000, 0.000000}
		\pgfsetfillcolor{diafillcolor}
		\pgfsetfillopacity{1.000000}
		\node[anchor=base,inner sep=0pt, outer sep=0pt,color=dialinecolor] at (8.077856\du,-1.826923\du){\tiny{personas acerca del cuidado del estado}};
		% setfont left to latex
		\definecolor{dialinecolor}{rgb}{0.000000, 0.000000, 0.000000}
		\pgfsetstrokecolor{dialinecolor}
		\pgfsetstrokeopacity{1.000000}
		\definecolor{diafillcolor}{rgb}{0.000000, 0.000000, 0.000000}
		\pgfsetfillcolor{diafillcolor}
		\pgfsetfillopacity{1.000000}
		\node[anchor=base,inner sep=0pt, outer sep=0pt,color=dialinecolor] at (8.077856\du,-1.544701\du){\tiny{ambiental de los ríos.}};
		\pgfsetlinewidth{0.050000\du}
		\pgfsetdash{}{0pt}
		\pgfsetmiterjoin
		\pgfsetbuttcap
		{
			\definecolor{diafillcolor}{rgb}{0.396078, 0.345098, 0.960784}
			\pgfsetfillcolor{diafillcolor}
			\pgfsetfillopacity{1.000000}
			% was here!!!
			\definecolor{dialinecolor}{rgb}{0.396078, 0.345098, 0.960784}
			\pgfsetstrokecolor{dialinecolor}
			\pgfsetstrokeopacity{1.000000}
			\pgfpathmoveto{\pgfpoint{4.251332\du}{-10.885858\du}}
			\pgfpathcurveto{\pgfpoint{5.294408\du}{-10.599299\du}}{\pgfpoint{7.300322\du}{-14.565278\du}}{\pgfpoint{9.581334\du}{-12.708374\du}}
			\pgfusepath{stroke}
		}
		% setfont left to latex
		\definecolor{dialinecolor}{rgb}{0.000000, 0.000000, 0.000000}
		\pgfsetstrokecolor{dialinecolor}
		\pgfsetstrokeopacity{1.000000}
		\definecolor{diafillcolor}{rgb}{0.000000, 0.000000, 0.000000}
		\pgfsetfillcolor{diafillcolor}
		\pgfsetfillopacity{1.000000}
		\node[anchor=base,inner sep=0pt, outer sep=0pt,color=dialinecolor] at (9.659043\du,-12.355739\du){\tiny{La implementación de impuestos de}};
		% setfont left to latex
		\definecolor{dialinecolor}{rgb}{0.000000, 0.000000, 0.000000}
		\pgfsetstrokecolor{dialinecolor}
		\pgfsetstrokeopacity{1.000000}
		\definecolor{diafillcolor}{rgb}{0.000000, 0.000000, 0.000000}
		\pgfsetfillcolor{diafillcolor}
		\pgfsetfillopacity{1.000000}
		\node[anchor=base,inner sep=0pt, outer sep=0pt,color=dialinecolor] at (9.659043\du,-12.073517\du){\tiny{``pago por uso'' es una estrategia}};
		% setfont left to latex
		\definecolor{dialinecolor}{rgb}{0.000000, 0.000000, 0.000000}
		\pgfsetstrokecolor{dialinecolor}
		\pgfsetstrokeopacity{1.000000}
		\definecolor{diafillcolor}{rgb}{0.000000, 0.000000, 0.000000}
		\pgfsetfillcolor{diafillcolor}
		\pgfsetfillopacity{1.000000}
		\node[anchor=base,inner sep=0pt, outer sep=0pt,color=dialinecolor] at (9.659043\du,-11.791295\du){\tiny{efectiva para incentivar a las personas}};
		% setfont left to latex
		\definecolor{dialinecolor}{rgb}{0.000000, 0.000000, 0.000000}
		\pgfsetstrokecolor{dialinecolor}
		\pgfsetstrokeopacity{1.000000}
		\definecolor{diafillcolor}{rgb}{0.000000, 0.000000, 0.000000}
		\pgfsetfillcolor{diafillcolor}
		\pgfsetfillopacity{1.000000}
		\node[anchor=base,inner sep=0pt, outer sep=0pt,color=dialinecolor] at (9.659043\du,-11.509073\du){\tiny{a una buena gestión de la basura.}};
		\pgfsetlinewidth{0.050000\du}
		\pgfsetdash{}{0pt}
		\pgfsetmiterjoin
		\pgfsetbuttcap
		{
			\definecolor{diafillcolor}{rgb}{0.396078, 0.345098, 0.960784}
			\pgfsetfillcolor{diafillcolor}
			\pgfsetfillopacity{1.000000}
			% was here!!!
			\definecolor{dialinecolor}{rgb}{0.396078, 0.345098, 0.960784}
			\pgfsetstrokecolor{dialinecolor}
			\pgfsetstrokeopacity{1.000000}
			\pgfpathmoveto{\pgfpoint{9.670862\du}{-11.407475\du}}
			\pgfpathcurveto{\pgfpoint{9.662756\du}{-10.386221\du}}{\pgfpoint{7.717244\du}{-10.779408\du}}{\pgfpoint{7.711781\du}{-10.015667\du}}
			\pgfusepath{stroke}
		}
		\pgfsetlinewidth{0.050000\du}
		\pgfsetdash{}{0pt}
		\pgfsetmiterjoin
		\pgfsetbuttcap
		{
			\definecolor{diafillcolor}{rgb}{0.396078, 0.345098, 0.960784}
			\pgfsetfillcolor{diafillcolor}
			\pgfsetfillopacity{1.000000}
			% was here!!!
			\definecolor{dialinecolor}{rgb}{0.396078, 0.345098, 0.960784}
			\pgfsetstrokecolor{dialinecolor}
			\pgfsetstrokeopacity{1.000000}
			\pgfpathmoveto{\pgfpoint{9.678967\du}{-11.383160\du}}
			\pgfpathcurveto{\pgfpoint{9.703282\du}{-10.442957\du}}{\pgfpoint{11.969221\du}{-10.634353\du}}{\pgfpoint{11.985432\du}{-9.985937\du}}
			\pgfusepath{stroke}
		}
		% setfont left to latex
		\definecolor{dialinecolor}{rgb}{0.000000, 0.000000, 0.000000}
		\pgfsetstrokecolor{dialinecolor}
		\pgfsetstrokeopacity{1.000000}
		\definecolor{diafillcolor}{rgb}{0.000000, 0.000000, 0.000000}
		\pgfsetfillcolor{diafillcolor}
		\pgfsetfillopacity{1.000000}
		\node[anchor=base,inner sep=0pt, outer sep=0pt,color=dialinecolor] at (7.743873\du,-9.728056\du){\tiny{Realizar estrategias de acercamiento}};
		% setfont left to latex
		\definecolor{dialinecolor}{rgb}{0.000000, 0.000000, 0.000000}
		\pgfsetstrokecolor{dialinecolor}
		\pgfsetstrokeopacity{1.000000}
		\definecolor{diafillcolor}{rgb}{0.000000, 0.000000, 0.000000}
		\pgfsetfillcolor{diafillcolor}
		\pgfsetfillopacity{1.000000}
		\node[anchor=base,inner sep=0pt, outer sep=0pt,color=dialinecolor] at (7.743873\du,-9.445834\du){\tiny{a la comunidad por parte de los entes}};
		% setfont left to latex
		\definecolor{dialinecolor}{rgb}{0.000000, 0.000000, 0.000000}
		\pgfsetstrokecolor{dialinecolor}
		\pgfsetstrokeopacity{1.000000}
		\definecolor{diafillcolor}{rgb}{0.000000, 0.000000, 0.000000}
		\pgfsetfillcolor{diafillcolor}
		\pgfsetfillopacity{1.000000}
		\node[anchor=base,inner sep=0pt, outer sep=0pt,color=dialinecolor] at (7.743873\du,-9.163612\du){\tiny{gubernamentales para conocer las}};
		% setfont left to latex
		\definecolor{dialinecolor}{rgb}{0.000000, 0.000000, 0.000000}
		\pgfsetstrokecolor{dialinecolor}
		\pgfsetstrokeopacity{1.000000}
		\definecolor{diafillcolor}{rgb}{0.000000, 0.000000, 0.000000}
		\pgfsetfillcolor{diafillcolor}
		\pgfsetfillopacity{1.000000}
		\node[anchor=base,inner sep=0pt, outer sep=0pt,color=dialinecolor] at (7.743873\du,-8.881389\du){\tiny{exigencias del cuidado de las cuencas.}};
		% setfont left to latex
		\definecolor{dialinecolor}{rgb}{0.000000, 0.000000, 0.000000}
		\pgfsetstrokecolor{dialinecolor}
		\pgfsetstrokeopacity{1.000000}
		\definecolor{diafillcolor}{rgb}{0.000000, 0.000000, 0.000000}
		\pgfsetfillcolor{diafillcolor}
		\pgfsetfillopacity{1.000000}
		\node[anchor=base,inner sep=0pt, outer sep=0pt,color=dialinecolor] at (11.908787\du,-9.716907\du){\tiny{Proveer atención médica a}};
		% setfont left to latex
		\definecolor{dialinecolor}{rgb}{0.000000, 0.000000, 0.000000}
		\pgfsetstrokecolor{dialinecolor}
		\pgfsetstrokeopacity{1.000000}
		\definecolor{diafillcolor}{rgb}{0.000000, 0.000000, 0.000000}
		\pgfsetfillcolor{diafillcolor}
		\pgfsetfillopacity{1.000000}
		\node[anchor=base,inner sep=0pt, outer sep=0pt,color=dialinecolor] at (11.908787\du,-9.434685\du){\tiny{zonas donde la contaminación}};
		% setfont left to latex
		\definecolor{dialinecolor}{rgb}{0.000000, 0.000000, 0.000000}
		\pgfsetstrokecolor{dialinecolor}
		\pgfsetstrokeopacity{1.000000}
		\definecolor{diafillcolor}{rgb}{0.000000, 0.000000, 0.000000}
		\pgfsetfillcolor{diafillcolor}
		\pgfsetfillopacity{1.000000}
		\node[anchor=base,inner sep=0pt, outer sep=0pt,color=dialinecolor] at (11.908787\du,-9.152463\du){\tiny{es evidente para controlar}};
		% setfont left to latex
		\definecolor{dialinecolor}{rgb}{0.000000, 0.000000, 0.000000}
		\pgfsetstrokecolor{dialinecolor}
		\pgfsetstrokeopacity{1.000000}
		\definecolor{diafillcolor}{rgb}{0.000000, 0.000000, 0.000000}
		\pgfsetfillcolor{diafillcolor}
		\pgfsetfillopacity{1.000000}
		\node[anchor=base,inner sep=0pt, outer sep=0pt,color=dialinecolor] at (11.908787\du,-8.870241\du){\tiny{los casos de enfermedades.}};
	\end{tikzpicture}
	\caption{Diagrama del árbol de objetivos.}
	\label{arbolObjetivo}
\end{figure}


\begin{enumerate}
	\item \textbf{Malos olores}\\
	      Los ríos y arroyos de las ciudades en el pasar de los tiempos y con el aumento de la población han sido victimas de daños ambientales, los cuales han emitido malos olores y colores casi negros debido a esta problemática. Por lo que se ha demostrado que esto presenta ser un problema complejo y difícil de tratar a corto plazo por esto es importante realizar movimientos de estrategias ambientales para mantener los ríos y arroyos saludables con ayuda de universidades y el gobierno local de la ciudad.
	      \begin{itemize}
		      \item \textbf{Alto índice de basura arrojadas}\\

		      \item \textbf{Lecho fluvial reducido} \\
		            En los canales naturales de los ríos y arroyos de las ciudades existen varios tamaños de canales con diferentes propósitos el lecho menor donde la circulación de aguas siempre esta presente y el lecho mayor cuando el nivel del agua ha aumentado y este debe ser suficiente para evitar un derrame de canal. Sin embargo dada la contaminación que se suele presentar en las regiones del lecho mayor se puede llegar a dar un caso de represamiento e inundación, por lo que crear un canal artificial alterno sera una buena idea para evitar estos peligros.
		      \item \textbf{Desechos de residuos sólidos}\\

		      \item \textbf{Flujo hídrico represado}\\
		            Los canales naturales de los ríos y arroyos de las ciudades suelen ser lugares donde las plantas, llanos y arboles crecen en abundancia cuando en tiempos de falta de lluvia se refiere, el creciendo de estos suele ser en algunos casos tan abundante hasta ser parte de generar un bloqueo al flujo hídrico de los cuerpos de agua.
	      \end{itemize}
	\item \textbf{Presencia de bacterias}\\
	      Con el fin de monitorear los organismos bacterianos que se pueden identificar en el rió detectando las bacterias que son dañinas par el ambiente y las que pueden aportar algún tipo de beneficio.
	      \begin{itemize}
		      \item \textbf{Aguas residuales} \\
		            Con dicho decreto, lograr que la evacuación de las aguas residuales que necesiten un desagüe tengan una circulación adecuada para que no tenga contacto con aguas que son importantes para el habitad de diferentes animales y para el bienestar de la misma población urbana aledaña a los ríos.
		      \item \textbf{Aumento de nivel de caudal} \\
		            Al extraer las basuras cada determinado tiempo se genera un tipo de control sobre el nivel del agua que circula por los ríos evitando daños ambientales y urbanos a causa de algún desbordamiento

		      \item \textbf{Químicos dañinos}\\
		            Ley que tendrá como fin  proteger el agua de los diferentes químicos que pueden tender a derramarse en los ríos debido a actividades comerciales o cotidianas y el cual establezca un tipo de amonestación que genere reflexión en las personas.
		      \item \textbf{Destrucción de flora y fauna}\\
		            El sistema tendrá como fin agrupar los desechos plásticos y extraerlos para evitar que los posibles seres vivos que circulan por el rió sean victimas de estos desechos y que las diferentes platas presentes sean cubiertas por estos mismos.
	      \end{itemize}
	\item \textbf{Aguas turbias}\\
	      Limitar los valores de turbidez en los cuerpos hídricos de las ciudades es importante a la hora de realizar desinfecciones y mantener controlados los solidos en suspensión que se encuentran ellos.
	      \begin{itemize}
		      \item \textbf{Sedimento débil} \\
		            Tener un mejor control acerca de la etapa en el que se encuentre el suelo para prevenir daños externos y otros factores.
		      \item \textbf{Poca atención de la población}\\
		            Contar con planes de desarrollo para dar a conocer a la población la importancia del cuidado de estas aguas hídricas.
		      \item \textbf{Desbordamientos} \\
		            El origen de un desbordamiento en los ríos, se producen más que todo por la acumulación de materiales desechables o no residuales, afectando de manera critica a las ciudades que se encuentran ubicadas en las riberas de los ríos.
		      \item \textbf{Daños ambientales}\\
		            La contaminación es producida por un proceso industrial a causas de orden operacional, en la mayoría de los casos a la falta de una conciencia y de una política ambiental a nivel público y privado. Las causas y consecuencias van ligadas a provocar o generar un tipo de desbordamiento.
	      \end{itemize}
	\item \textbf{Mala gestión de la basura}\\
	      Realizar una buena gestión de la basura en los hogares es una tarea compleja, sin embargo estudios muestran que aplicar impuestos de ``pago por uso'' reduce significativamente la cantidad de desechos que generan las personas.
	      \begin{itemize}
		      \item  \textbf{Apoyo insuficiente}\\

		      \item \textbf{Problemas sanitarios}\\
		            Debido al nacimiento de diferentes plagas ocasionadas en estos ríos o zonas verdes, dado que tienen un acercamiento o contacto con la comunidad, se debería contar con un centro medico para poder combatir estas enfermedades ocasionadas por estas plagas.
	      \end{itemize}
\end{enumerate}

\section{Objetivo del Proyecto}
\section{Matriz de Marco Lógico}
\begin{table}[H]
	\centering
	\footnotesize\sf
	\begin{tabular}{| g | Sc | Sc |} \hline
		\rowcolor[gray]{.8}                                                                     & \scriptsize \thead{Resumen                                                                                                                                                   \\Narrativo} & \scriptsize \thead{Indicadores Objetivamente\\Verificables} \\ \hline

		\multirow{2}{*}[1.2mm]{\cellcolor[gray]{.8} \scriptsize \thead{Propósito General(Fin)}} & \tiny \multirow{2}{*}{\makecell{Reducir la contaminación presente en las cuencas de la ciudad de                                                                             \\Popayán, aumentando el índice de actividades de limpieza y el uso de\\tecnologías adecuadas con el fin de obtener un mejoramiento del $50\%$\\para el $2023$.}} & \cellcolor[gray]{.8} \scriptsize \textbf{Indicadores de Impacto} \\ \cline{3-3}
		\cellcolor[gray]{.8}                                                                    &                                                                                                       & \tiny \makecell{Disminución de la contaminación hídrica en la ciudad \\ y con ello una disminución considerable de los malos olores,\\ color de aguas negras y presencia de plagas en las\\ zonas colindantes.} \\ \hline

		\multirow{2}{*}[1.2mm]{\cellcolor[gray]{.8} \scriptsize \thead{Objetivo del Proyecto}}  & \tiny \multirow{2}{*}{\makecell{Proponer el uso de la tecnología para el monitoreo y el control de la                                                                        \\contaminación en las fuentes hídricas que circulan por la ciudad.}} & \cellcolor[gray]{.8} \scriptsize \textbf{Indicadores de Efectividad} \\ \cline{3-3}
		\cellcolor[gray]{.8}                                                                    &                                                                                                       & \tiny \makecell{Metodos de evealuación que permita comparar          \\los estados antes y después de haber\\introducido el sistema, y observar que cambios\\se han producido en un determinado tiempo\\(lapso de tiempo corto).} \\ \hline

		\multirow{2}{*}[1.2mm]{\cellcolor[gray]{.8} \scriptsize \thead{Resultado (Productos)}}  & \tiny \multirow{2}{*}{\makecell{Se reducirán los desechos en los ríos y se fortalecerán las zonas más                                                                        \\débiles, como también extraer las basuras del fondo de los ríos para\\prevenir el aumento del nivel del agua.}} & \cellcolor[gray]{.8} \scriptsize \textbf{Indicadores de Gestión} \\ \cline{3-3}
		\cellcolor[gray]{.8}                                                                    &                                                                                                       & \tiny \makecell{Campañas informativas acerca del manejo de           \\basuras y para generar conciencia en las\\personas sobre el cuidado del medio ambiente,\\acompañados por resultados de monitores\\realizados recientemente.} \\ \hline

		\multirow{2}{*}[1.2mm]{\cellcolor[gray]{.8} \scriptsize \thead{Actividades}}            & \tiny \multirow{2}{*}{\makecell{Programa ``Base de datos'' tener un visto bueno por alguna empresa                                                                           \\relacionada con el tema de bases de datos, contratar personal capacitado\\con experiencia en la creación de``Base de datos''. Diseñar un sistema\\que se actualice a diario para tener una información numeraria y útil/\\programa ``análisis de datos'' contratar un analista con experiencia en\\``Bases de datos'', implementar una disciplina de trabajo diariamente,\\para sacar datos más precisos y acertados/crear un programa donde se\\pueda visualizar el entorno del ``análisis de datos'' ejercido por un\\personal especializado con experiencias de usuario, para brindar\\un ambiente agradable entre lo proyectado y el usuario.}} & \cellcolor[gray]{.8} \scriptsize \textbf{Insumo/Bienes y servicios requeridos} \\ \cline{3-3}
		\cellcolor[gray]{.8}                                                                    &                                                                                                       & \tiny \makecell{*Un máster en Big Data Analytics de                  \\tiempo completo por 2 años. *Un estudiante con\\doctorado en DBMS por 3 meses. *Un experto en\\desarrollo de software por 4 meses. *Una oficina\\espaciosa ** e computadores. *Conexión a\\internet. *Licencias de software de análisis\\estadístico. *Un servidor de host para la base de\\datos. *Papelería.} \\ \hline
	\end{tabular}
	\caption{Matriz de marco lógico}
	\label{matrizML}
\end{table}


\section{Planteamiento del Problema}
\subsection{Definición del Problema}
El problema que se comprende en este documento trata principalmente en la contaminación hídrica de los ríos y arroyos de la ciudad de Popayán y sus principales causas. En la ciudad existen instituciones como la Alcaldía, la Secretaria de Salud, empresa prestadora del servicio de acueducto y alcantarillado y la empresa encargada del aseo de la ciudad, las cuales se encargan de realizar actividades de inspección, vigilancia, limpieza y control de riesgos asociados a las condiciones de la calidad del agua y de las cuencas \cite{SecretariadeSalud2019}.

Las actividades que realizan estas instituciones son enfocadas para mantener el agua potable por debajo de los niveles permitidos de la calidad del agua(Indice de Riesgo de la Calidad del Agua -- IRCA) y son ejecutadas en los tramos antes del proceso de captación de agua de las plantas de tratamiento. Sin embargo, estas actividades a pesar de cumplir con su propósito se presencia un grave descuido del estado del agua y de las cuencas al atravesar estas por la ciudad, después de la captación de aguas para las plantas de tratamiento.

Por ello, esta investigación pretende explicar cuáles son las causas de la contaminación presente en los ríos y arroyos de la ciudad después del proceso de captación del agua de las platas de tratamiento. Inicialmente se identifica el tipo de contaminantes que alberga estos cuerpos hídricos entre los cuales se conoce la presencia de contaminación por residuos físicos y contaminación de sustancias químicas, que son dañinas para el estado natural de estas aguas. La contaminación física hace referencia a las basuras que son arrojados al agua de manera directa o que llegan por otras vías, como las aguas residuales, y la contaminación química que son los contaminantes que no se aprecian, pero que se encuentran presentes en el agua, como por ejemplo, los pesticidas químicos, bacterias fecales o productos procedentes de la industria \cite{tipocontaminacion}.

A pesar de que los ríos y arroyos que atraviesan la ciudad sufren por la presencia de desechos solidos en su canal no es la única causa por la que las corrientes de agua se interrumpen, la flora y la vegetación debido al calentamiento global que se padece en las últimas décadas trae consecuencia de un crecimiento acelerado de arbustos, plantas y arboles invadiendo gran parte del lecho fluvial de los ríos. Debido a este crecimiento de la vegetación la profundidad y el ancho del lecho fluvial de los ríos se empieza a reducir generando un flujo lento por la zona. Cuando el flujo hídrico de los ríos tienen un comportamiento de transito lento este tiende a estancarse y el nivel del agua asciende provocando que la terraza de inundación no sea suficiente para contener el agua dentro del canal \cite{Stepien2019}.

Las consecuencias que generan los excesos de basuras o de vegetación en los canales de las cuencas urbanas son múltiples, pero en este caso es importante mencionar que los flujos hídricos en estos casos tiende a represarse y con ello generar nuevos efectos que evidencian en gran medida la presencia de la contaminación. Cuando el agua de los ríos se encuentra en una situación de cero posibilidades de circulación esta empieza a tomar cambios físicos debido a la suspensión de bacterias en la zona en cuestión, los efectos físicos más evidentes de estos escenarios se dan atreves del olor y el color del liquido. La generación de estas condiciones físicas en el agua son perjudiciales debido a que atraen diferentes tipos de plagas, en su mayoría presencia de todo tipo de mosquitos, los cuales es necesario tener precauciones por las enormes cantidades de enfermedades que están relacionadas con los mosquitos \cite{SecretariadeSalud2020}. También hay que resaltar que los malos olores de los ríos en estas situaciones tienden a que los habitantes eviten acercarse al máximo a estas aguas y reduce la disponibilidad del número de personas comprometidas con el cuidado de estos cuerpos hídricos \cite{Yu2021}.



Por otro lado en la sección del efecto en el cual se analiza la presencia de bacterias producto de diferentes factores que son causas directas e indirectas en la aparición de dichas bacterias el cual pueden generar malestar en el ambiente y en la salud de las personas produciendo un daño en la población aledaña a estos cauces tanto en animales como en las mismas personas. Se observa también que a nivel internacional se ha venido realizando durante mucho tiempo el estudio del agua en los ríos ya que en países desarrollados es de mucha importancia este recurso natural, que en ocasiones escasea en muchas partes del mundo por lo cual se han venido desarrollando tecnologías para detectar y evaluar la calidad del agua, determinando la presencia de bacterias que habitan en los ríos.

Cuando se habla de evaluar la calidad del agua en los ríos identificando la existencia de organismos bacteriológicos se tiene como primera instancia la necesidad de extraer muestras para un determinado estudio en algunos casos el análisis de dicho recurso se realiza en el mismo lugar pero en otros casos como hicieron los científicos en el departamento del cesar, en Colombia, que tomaron muestras hídricas en frascos para que posteriormente se le realizara un estudio de dicha sustancia en laboratorios especializados para dicha labor, siendo ambas formas válidas para llevar a cabo la identificación de bacterias. Es importante precisar que dichas muestras deben ser refrigeradas y transportadas en un tiempo menor a 6 horas para ser procesadas inmediatamente debido a que algunos organismos pueden asentarse, diluirse, morir o desaparecer por lo cual podrían pasar desapercibidos y así mismo estropear el análisis.

Luego de este proceso se determina que organismos bacteriológicos están presentes en el recurso hídrico a estudiar, se comparan los resultados con otras sustancias haciendo uso también de las normativas que rigen en su sector acerca de la evaluación de la calidad del agua, que para el caso de los científicos del cesar que tomaron y analizaron las muestras de los ríos Manaure y Casacara se rigieron bajo las normativas de Colombia (Resolución 2115 del 2007 y Decretos 1594 del 1984 y 3930 de 2010). Que permite saber que valores de bacterias son en cierta medida aceptados y cuáles son los que ya generan un riesgo para el ambiente y las personas.

Se encontraron concentraciones de coliformes totales, coliformes fecales, enterococos fecales, Pseudomonas, aeruginosa y Salmonella a través del método del sustrato definido, todo esto se debe en gran medida al inadecuado manejo de sus cuencas y las actividades antrópicas como la tala, quema, invasión de riveras, erosión y vertimiento de residuos sólidos y líquidos. Estos ríos en algunos sectores son utilizados como puntos de descargas de desechos sólidos, aguas residuales domésticas y agroindustriales, provocando su degradación y contaminación. Así mismo, propician el desequilibrio de los ecosistemas y del medio ambiente, además de las implicaciones que esto puede tener en la salud de los asentamientos humanos aledañas a lo largo de toda su cuenca.

-Debido a que hay ríos que son fuentes hídricas las cuales suministran agua a sectores poblados surge de ahí la necesidad de establecer modelos de evaluación y gestión integral que garantice su calidad. Entonces es importante tener presente los principales indicadores microbiológicos que se emplean para evaluar el agua potable, esto es clave para poder proponer un nuevo esquema para el monitoreo en Colombia. Los resultados permiten considerar como bioindicadores, además de las bacterias y protozoos establecidos en la norma, algunos agentes microbianos como virus u otras bacterias y parásitos. Por otro lado indican la necesidad de establecer valores de referencia y definir los microorganismos a emplear con base en evaluaciones específicas de la situación microbiana del agua en monitoreos de validación, operación y verificación. Es por ello que el agua que es utilizada para uso doméstico en la población debe estar libre de microorganismos causantes de enfermedades por lo que las posibles consecuencias de la contaminación microbiana para la salud son tales que su control debe ser objetivo primordial y nunca debe comprometerse.

Algunos de los principales factores que provocan el aumento de bacterias, parásitos, virus y hongos en el agua son por los cambios en el ambiente o en la misma población por desarrollar actividades de urbanización no controlada, crecimiento industrial, pobreza y la disposición inadecuada de excretas humanas y animales según la investigación realizada por la Doctora en Ciencias Básicas Biomédicas, Microbióloga de la universidad pontifica bolivariana. Con base en los criterios mencionados los indicadores microbiológicos de  contaminación del agua generalmente han sido bacterias de la flora saprófita intestinal, entre las que se encuentran Bacteroides fragilis, bacterias mesófilas, coliformes totales, y fecales [termotolerantes], Escherichia coli y estreptococos fecales. Algunas de estas, de origen animal [generalmente de explotaciones pecuarias], representan un alto potencial zoonótico, siendo abundantes estreptococos fecales y parásitos como Giardia intestinalis y Cryptosporidium, que tienen una mayor resistencia a los procesos de tratamiento y desinfección del agua para consumo humano

-También se encontró que la contaminación del agua es producto de sustancias químicas tales como los que producen los residuos farmacéuticos, desechos de medicamentos caducos, contaminación en rellenos sanitarios por residuos farmacéuticos que contaminan directa e indirectamente el agua, se determinó en una investigación realizada en la ciudad de México que el agua potable de algunas zonas de su ciudad contenían microcontaminantes de origen farmacéutico en concentraciones relativamente bajas, sustancias tales como estrógenos, ibuprofeno, gemfibrozil, ketoprofeno, ácido salicílico, diclofenaco, di-2-etilhexilftalato (DEHP), butilbencilftalato (BBP), triclosán, bisfenol A (BPA) y 4-nonilfenol (4-NP). Con esto se da a entender que muchas veces las industrias farmacéuticas y los pequeñas farmacias del sector comercial en la ciudad no tienen una gestión adecuada de sus desechos, puede que dichas empresas no están obligadas a eliminar los residuos farmacológicos de manera amigable con el ambiente gracias a la carencia de leyes que promuevan un buen manejo de estos desechos.

Los resultados del estudio realizado en el río Tlalnepantla del Estado de México, México, indicaron que los músculos de C. carpio tenían residuos de diclofenaco, ibuprofeno y naproxeno, a concentraciones de 0.08 a 0.21 ppm, lo que les ocasionó estrés oxidativo y alteraciones genéticas. No hay, hasta este momento, un procedimiento para la correcta eliminación de residuos farmacológicos por parte de las empresas farmacéuticas, hospitales o casas-habitación de nuestro país (Gracia-Vásquez et al. 2014). La presencia de residuos farmacológicos en diferentes reservorios del suelo y acuíferos es un peligro latente para la salud tanto de los seres humanos como de la flora y fauna de todo el planeta cuya presencia en el agua en grandes cantidades pueden causar graves daños en los ecosistemas acuáticos, reduciendo la biodiversidad. Provienen de los vertidos domésticos, agrícolas e industriales, que pueden contener distintos compuestos químicos. En ocasiones, son liberados directamente a la atmósfera e incorporados por la lluvia. Los esfuerzos de varios países no sólo desarrollados sino también aquellos en vías de desarrollo por establecer regulaciones para este tipo de problema como lo son los residuos farmacéuticos y que cada vez la demanda de este tipo de productos es mayor.
\subsection{Justificación}

\section{Objetivos}
Determinar según las investigaciones realizadas y con los datos obtenidos si existe o no, algún grado de contaminación en los ríos circundantes a nuestra ciudad, que pueda afectar directa o indirectamente la salud de las personas y el bienestar de los ecosistemas que ahí se desarrollan. En el caso de que exista algún riesgo considerable, es importante tomar acciones para desarrollar un plan estratégico acorde a los problemas que se presenten en aquella zona para buscar y poner a prueba un sistema que mejore la calidad del agua.
\subsection{Objetivo General}
Buscar el plan más coherente para la descontaminación de los Ríos según las necesidades presentadas en la zona y que son identificadas gracias a los análisis realizados en dichos ríos.
\subsection{Objetivos Específicos}
\begin{itemize}
	\item Aumentar los niveles de limpieza en los cauces.
	\item Ejercer un tipo de inspección obligatoria alrededor de los depósitos de basura y los rellenos sanitarios.
	\item Incentivar una expansión optimizada de los puntos de muestreo y el monitoreo continuo, para tener una mejor verificación de estado de estos ríos.
	\item Reestructurar el sistema de alcantarillado, de manera que genere una mejor calidad de vida para sus habitantes.
	\item Gestionar campañas informativas para dar una mayor conciencia y explicar la situación crítica que se genera a raíz de la contaminación.
	\item Dar un mejor control con los abonos que se utilizan cercano o próximos al rio.
	\item Disminuir el porcentaje de concentración bacteriológica en el agua.
	\item Monitorear la circulación natural de los ríos para evitar estancamientos y posteriores desbordes.
	\item Incrementar señalizaciones que indiquen la importancia del cuidado de estas zonas.
	\item Proliferar planes de gestión de basuras a la población.
	\item Estructurar un grupo encargado del monitoreo regular de dichas zonas.
	\item Reconstruir canales deteriorados por los desbordamientos y demás fuerzas que generan los ríos.
	\item Inspeccionar muestras de agua cada determinado tiempo del año para evaluar la calidad del agua.
	\item Medir el comportamiento que tienen las personas en el cuidado de los arroyos.
	\item Emplear la ley para sancionar a los que contaminen el agua.
	\item Construir un sistema que pueda filtrar los desechos sólidos que contaminan el agua.
	\item Establecer lugares adecuados donde la población aledaña agrupe su basura para evitar que estas paren en los ríos.
\end{itemize}

\printbibliography[title={Bibliografía}]
\end{document}
%---------------------------------------------
