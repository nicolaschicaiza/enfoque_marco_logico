\twocolumn[
	\begin{@twocolumnfalse}
		\maketitle
		\vspace{-5mm}
		\begin{abstract}
			El objetivo de este estudio fue determinar la cantidad, composición y origen de los desechos plásticos en uno de los ríos más grandes de Argentina (América del Sur), centrándose en el impacto de los ríos urbanos, relaciones entre macro, meso y microplásticos, cuestiones sociopolíticas y la ingestión de microplásticos por peces.

			Registramos una gran concentración de detritos macroplásticos de origen doméstico (hasta 5.05 artículos macroplásticos por $m^{2}$) dominados principalmente por bolsas (principalmente polietileno de alta y baja densidad), envoltorios de alimentos (polipropileno y poliestireno), plásticos de espuma (poliestireno expandido) y botellas de bebidas (poliestireno tereftalato de etileno), particularmente aguas abajo de la confluencia con un arroyo urbano. Esta sugerencia señala la inadecuada recolección, procesamiento y disposición final de residuos en la región, lo que lamentablemente recurrente en muchas ciudades del Sur Global y en Argentina en particular.

			Encontramos un promedio de 4654 fragmentos microplásticos $m^{2}$ en los sedimentos costeros del río, que van de 131 a 12687 microplásticos $m^{2}$. A diferencia de otros estudios de países industrializados de Europa y América del Norte, microplásticos secundarios (resultantes de la trituración de partículas más grandes) eran más abundantes que los primarios (microperlas para cosméticos o gránulos para la industria). Esto podría explicarse por las diferencias en los hábitos de consumo y el nivel de industrialización entre sociedades y economías.

			Se registraron partículas microplásticas (en su mayoría fibras) en el tracto digestivo del $100\%$ de las \textit{Prochilodus lineatus} (commercial species).

			Contrariamente a las declaraciones publicadas recientemente por otros investigadores, nuestros resultados sugieren que ni los macroplásticos ni los mesoplásticos servirían como sustitutos de los elementos microplásticos en las encuestas de contaminación, lo que sugiere la necesidad de considerar las tres categorías de tamaño.

			La contaminación plástica masiva encontrada en el río Paraná se debe a una gestión inadecuada de los residuos. Se requieren nuevas acciones para gestionar adecuadamente los residuos desde su inicio hasta su disposición final.
		\end{abstract}

	\end{@twocolumnfalse}
]
