\section{Discusiones}
\label{sec:discusiones}

\subsection{Concentraciones masivas de plástico: cuestiones geopolíticas y sociales}%
Los materiales macroplásticos son la forma más visible de contaminación plástica.~\cite{Blettler2017} reportaron un promedio de 172.5 ítems macroplásticos por transecto de $150 m^{2}$ ($\sim 1.15$ ítems $m^{2}$) en un lago de llanura aluvial del río Paraná, ubicado a solo 18 km de nuestra área de muestreo. En el presente estudio, encontramos casi el doble de esa cantidad: $340.8$ macroplásticos por $150 m^{2}$ ($\sim 2.27 m^{2}$).

Si bien se han realizado varios estudios sobre macroplásticos en la superficie del agua de ríos (~\cite{GASPERI2014163};~\cite{faure2015};~\cite{Baldwin2016};~\cite{LAHENS2018661}) y lagos \parencite{faure2015}, los estudios macroplásticos en sedimentos fluviales aún son escasos, especialmente para las playas. Algunos ejemplos incluyen~\cite{imhof2013contamination} en el lago de Garda (Italia) y~\cite{faure2015}en 6 lagos de Suiza. Sin embargo, la comparación directa con el presente estudio es inviable ya que estos autores consideraron macroplásticos como partículas superiores a $5 mm$ (incluido el tamaño mesoplástico).

La gran cantidad de detritos macroplásticos registrados en la playa Thompson e isla Curupí, así como el origen de los mismos (residuos domiciliarios, Tabla \ref{tab:macroplasticos}), sugieren un deficiente procesamiento de recolección y disposición final d residuos en la ciudad de Paraná. La gestión de residuos es uno de los problemas ambientales clave relacionados con los hidrosistemas urbanos a escala global, sin embargo, en el Sur Global todavía se basa fuertemente en el vertido incontrolado y/o la basura \parencite{GUERRERO2013220}. Como resultado, ocurren serios problemas ambientales \parencite{ALKHATIB20101131} y una creciente contaminación plástica \parencite{Battulga2019}, particularmente en los sistemas de agua dulce. Los municipios de los países de bajos ingresos están gastando una proporción menor de sus presupuestos en la gestión de desechos y, sin embargo, más del $90\%$  de los desechos en los países de bajos ingresos todavía se arrojan abiertamente \parencite{Kaza2018}. Además, el aumento de la población y el aumento de los niveles de consumo han acelerado enormemente la tasa de generación de residuos sólidos en Argentina (tasas de generación de residuos: $1.14 kg$ / cápita / día;~\cite{Kaza2018}). El presente estudio muestra, en parte, esta tendencia global.

La mayoría de los macroplásticos registrados en la presente investigación fueron bolsas de la compra, seguidos de envoltorios de alimentos y envases de espuma (casi el $80\%$; Tabla \ref{tab:macroplasticos}). Las primeras comunidades en adoptar la norma de las bolsas antiplásticas fueron las del Sur Global, mientras que las del Norte Global lo hicieron mucho más recientemente \parencite{Clapp2009}. Sin embargo, no se adoptó una ordenanza municipal contra las bolsas de plástico en la ciudad de Paraná antes de 2017.

Los resultados de los estudios de microplásticos disponibles en sistemas de agua dulce son extremadamente variables de acuerdo con la metodología utilizada (por ejemplo, $m^{2}$, $m^{3}$, $l$, $kg$), medio ambiente (río, lago, embalse, estuario, alcantarillado, etc.), y compartimento de muestreo (superficie o columna de agua, sedimento del fondo o de la playa, etc.). Como resultado, las comparaciones entre estudios mundiales son muy difíciles. Encontramos un promedio de 5239 microplásticos $m^{2}$ (rango de tamaño: $0.35-5 mm$) en los sedimentos del banco del río Paraná, que van desde solo 75 hasta un máximo de $34443$ microplásticos $m^{2}$ (Tabla \ref{tab:microplasticos}).~\cite{doi:10.1139/cjfas-2014-0281} encontraron alrededor de 13832 $m^{2}$ de microperlas de polietileno, retenidas por un tamiz de $0.5mm$, de efluentes industriales en los sedimentos del río San Lorenzo (Canadá).~\cite{Klein2015} tiene un registro de aproximadamente 228-3763 micropartículas $kg^{-1}$ en los sedimentos costeros de los ríos Rin y Main en Alemania (tamaño microplástico: $0.2-5mm$). Además,~\cite{SU2016711}han informado de un rango de $15-1600$ microplásticos $l^{-1}$ ($>0.3 mm$) en el río Yangtze medio-bajo (China),~\cite{WANG20171369} registraron $178-544$ microplásticos $l^{-1}$ ($<5 mm$) en los sedimentos de río Beijiang, y~\cite{PENG2017283} encontraron $410-1600$ microplásticos $kg^{-1}$ ($0.05-5 mm$) en algunos ríos de Shaghai, la mayoría de los fragmentos de dobladillo, esferas y fibras.

~\cite{Blettler2017}, utilizando la misma metodología que el presente estudio, han registrado un promedio mucho más bajo de $704$ microplásticos $m^{-2}$ (rango de tamaño: $0.35-5mm$) en sedimentos de playa de ambientes lénticos del río Paraná (un lago de llanura aluvial ubicado a $18 km$ de distancia del área de muestreo del presente estudio).~\cite{XIONG2018899} reportaron $50-1292$ microplásticos $m^{-2}$ ($>0.1mm$) en el lago Qinghai (China); la mayoría eran películas, fibras y espumas.

A pesar de las limitaciones y debilidades de las comparaciones anteriores (es decir, diferentes rangos de tamaño, unidades, ambientes), la información disponible sugiere una contaminación microplástica significativa presente en los sedimentos del río Paraná.

La variación de la abundancia y el tipo de microplásticos entre los sitios de muestreo fue estadísticamente significativa (Fig. \ref{fig:particulas_microplasticas}), mostrando una clara diferenciación por playa de muestreo. La playa de Thompson mostró la mayor concentración de microplásticos, mientras que Escondida reveló la distribución más heterogénea (las estaciones de muestreo variaron de baja a alta concentración de microplásticos).

El microplástico puede presentarse en forma primaria (perlas) o secundaria (originada por la descomposición de artículos plásticos más grandes;~\cite{COLE20112588}). Aún se desconoce la importancia relativa de las fuentes primarias y secundarias de microplásticos. Encontramos ambos, pero los secundarios fueron considerablemente más abundantes (Tabla \ref{tab:microplasticos})

Se debe prestar especial atención a la ropa sintética, que es una fuente importante de fibras a través del lavado \parencite{Conkle2018}. En nuestro estudio, la fibra fue el único microplástico primario registrado \parencite{Cole2013}. Sin embargo, cabe señalar que algunos autores consideran la fibra como secundaria (por ejemplo:~\cite{dris2015microplastic}). Otros microplásticos primarios como microperlas, cápsulas o gránulos (utilizados en cosméticos y productos de cuidado personal, depuradores industriales utilizados para limpieza con chorro abrasivo y gránulos vírgenes utilizados en procesos de fabricación de plástico, respectivamente) estaban ausentes. Se observó una falta similar de microperlas en el río Yangtze \parencite{ZHANG2015117} y el embalse de las Tres Gargantas \parencite{Zhang2017} en China, el río Saigón en Vietnam \parencite{LAHENS2018661} y el Estuario del río Paraná en Argentina \parencite{PAZOS2018134}. No obstante, se observó una gran presencia de microperlas en los ríos Rin y San Lorenzo (~\cite{Mani2015} y~\cite{doi:10.1139/cjfas-2014-0281}, respectivamente) y en los grandes Lagos Laurentinos \parencite{ERIKSEN2013177}. En algunos países que se benefician de instalaciones avanzadas de tratamiento de residuos (principalmente en Europa y América del Norte), las emisiones de microplásticos secundarios son incluso más bajas que las de los microplásticos primarios \parencite{gouin2015}. Las pérdidas de microplásticos primarios pueden ocurrir durante las etapas de producción, transporte o reciclaje de plásticos, o durante la fase de uso de productos que contienen microplásticos (por ejemplo, microperlas originadas de limpiadores faciales ampliamente utilizados en países desarrollados;~\cite{NAPPER2015178};~\cite{gouin2015}). Esto contrasta con los microplásticos secundarios que se originan principalmente a partir de residuos mal gestionados durante la eliminación de productos que contienen plásticos \parencite{Boucher2017}. La ausencia de microperlas en el sistema del río Paraná podría explicarse por estas diferencias en los hábitos de consumo y la gestión de residuos entre sociedades y países. Aquí, casi el $50\%$ de los microplásticos registrados fueron partículas de película (como producto secundario del proceso avanzado de descomposición de bolsas), $33.1\%$ de fibras (utilizadas en textiles) y $18.7\%$ resultantes de partículas más grandes de plástico de origen incierto que se descomponen en elementos más pequeños (probablemente botella de bebida, envoltorio de comida y espumas)(Tabla \ref{tab:microplasticos}). Por el contrario, otros estudios en ríos de países en desarrollo han informado de un predominio de fibras microplásticas (~\cite{ZHANG2015117};~\cite{LAHENS2018661}), incluso en el estuario del río Paraná \parencite{PAZOS2018134}.

Las proporciones variables entre macro o mesoplásticos en nuestro estudio han demostrado que estos datos no pueden servir como sustitutos para el monitoreo de microplásticos (Tabla \ref{tab:correlaciones}). Esto es importante ya que los voluntarios pueden realizar fácilmente estudios de desechos de macroplásticos, que han desempeñado un papel importante en muchos programas de monitoreo de desechos \parencite{RIBIC2012994}.

\subsection{Papel de las corrientes urbanas en la difusión del plástico}

Los ríos y arroyos urbanos sufren de múltiples factores estresantes interactivos, especialmente en el Sur Global (~\cite{WANG20171369};~\cite{wantzen2019urban}). En este estudio, el arroyo urbano Las Viejas parece jugar un papel crucial transportando grandes cantidades de residuos plásticos y depositándolos en la playa Thompson, inmediatamente aguas abajo hasta la confluencia con el río Paraná (Fig. \ref{fig:playa_thompson}). Esta área de muestreo mostró la mayor concentración de desechos macro y microplásticos (Figs. \ref{fig:densidades} y \ref{fig:particulas_microplasticas}). El arroyo Las Viejas discurre por toda la ciudad de Paraná, concentrando y transportando los residuos sólidos municipales mal gestionados. Según~\cite{Xu2019} el desarrollo de sistemas de alcantarillado no ha alcanzado la velocidad de urbanización en los países en desarrollo, con graves consecuencias para la calidad del agua de los ríos urbanos. Por lo tanto, muchos ríos urbanos se convierten en los puntos finales de la contaminación plástica (\cite{McCormick2014};~\cite{Mccormick2016}). De la misma manera que las lluvias y las inundaciones severas pueden aumentar drásticamente los niveles de plástico en el mar \parencite{GUNDOGDU2018342}, es muy probable que el mismo fenómeno opere en arroyos urbanos que descargan en grandes sistemas fluviales.

Por otro lado, la isla Curupí presentó un promedio de 190 macroplásticos por transecto (contra 780 en el Thompson y solo 52 en la playa escondida; Tabla \ref{tab:macroplasticos}). Este sitio de muestreo estuvo dominado por dos artículos domésticos: botellas de bebidas y fragmentos de empaque de espuma (espuma de poliestireno; Fig. \ref{fig:densidades}). Suponemos que estos plásticos legaron del arroyo Las Viejas. Los desechos flotantes son transportados por la corriente de río Paraná y los vientos dominantes del sur hasta las costas de la isla Curupí. Este proceso podría verse facilitado por la alta flotabilidad de estos elementos (densidad del EPS: $11-32 kg m^{-3}$; mientras que la densidad del PET es de $950 kg m^{-3}$, las botellas inicialmente flotan debido al aire atrapado en el costado). De lo contrario, no se registraron bolsas de la compra y envoltorios de alimentos (artículos más abundantes en la playa Thompson) en la isla, lo que probablemente esté relacionado con su baja flotabilidad (densidad de HDPE: $950 kg m^{-3}$; LDPE: $917-930 kg m^{-3}$; PP: $946 kg m^{-3}$; PS: $1066 kg m^{-3}$).

Finalmente, no existe confluencias de ríos urbanos en la playa Escondida, que fue el área de muestreo menos contaminada. Esta playa mostró una composición de desechos plásticos completamente diferentes. Si bien las bolsas de la compra, la espuma de poliestireno y las botellas de bebidas estaban presentes, el artículo dominante era el hilo de pescar. Sugiere que el principal impacto lo dan los usuarios de la playa, la mayoría pescaderos artesanales y deportivos, y no los residuos municipales mal tratados provenientes de las grandes ciudades río arriba.

Los polímeros plásticos más comunes registrados en este estudio fueron HDPE, LPDE PP, PS Y EPS, que pueden ser muy dañinos para la fauna silvestre \parencite{Kyaw2012}. Además, el PP y el PS se han registrado ampliamente en las partículas de envoltorios de alimentos (Tabla \ref{tab:macroplasticos}). Finalmente, los productos de EPS (a menudo denominados Styrofoam$^{TM}$)(recipientes para llevar, vasos desechables, bandejas de espuma, etc.) se encontraron ampliamente en nuestro estudio (Tabla \ref{tab:macroplasticos}). El EPS se informa comúnmente como uno de los principales elementos de escombros recuperados de costas y playas en todo el mundo (~\cite{LEE2013349};~\cite{ocean}). Como resultado, los productos EPS ahora se discuten para prohibirlos en varios países (UNEP 2018). En el presente estudio, el EPS fue el detrito mesoplástico más abundante (casi el $90\%$; Tabla \ref{tab:mesoplasticos}).~\cite{ZBYSZEWSKI2014288} reportaron una proporción similar en mesoplásticos de los Grandes Lagos.

\subsection{Ingestión de plástico por pescado y posibles impactos}

Un estudio reciente reveló que se ha reportado ingestión de plástico en 427 especies de peces de más de 20 países alrededor del mundo \parencite{Azevedo-Santos2019}, causando bloqueos internos y lesiones en el tracto digestivo de los peces (~\cite{CANNON2016286};~\cite{NADAL2016517}). Registramos microplásticos en el tracto digestivo de $100\%$ de las muestras de \textit{P. lineatus} muestreadas, corroborando un estudio similar en el estuario de río Paraná \parencite{PAZOS201785}). Esto último podría explicarse por la estrategia de alimentación detritívora de esta especie y la gran cantidad de microplásticos registrados en el área de estudio. Por lo tanto, la frecuencia de aparición de microplásticos en peces del río Paraná parece ser más alta que en otros ríos de América del Sur. Por ejemplo, en el estuario del Amazonas y la costa norte de Brasil se encontraron microplásticos en el $13.8\%$ de los tractos digestivos examinados \parencite{Pegado2018}, el $23\%$ y el $13.4\%$ en el estuario de Goiana (\cite{POSSATTO20111098};~\cite{Ramos2012}, respectivamente). Sin embargo, reconocemos que el bajo número de ejemplares aquí estudiados no permite generalizaciones.

En nuestro estudio, la mayoría de los microplásticos registrados en el pescado fueron fibras ($90\%$). De acuerdo, varios estudios a nivel mundial también han reportado mayor número de fibras ingeridas en comparación con otros tipos de microplásticos (\cite{NEVES2015119};~\cite{BELLAS201655};~\cite{NADAL2016517};~\cite{PAZOS201785}). El razonamiento detrás del predominio de las fibras es la naturaleza diversa de este tipo de microplásticos, que puede tener su origen en la degradación de prendas de vestir, muebles y artes de pesca. De hecho, lavar (a través de una lavadora) una sola prenda sintética resultó en la liberación de aproximadamente 2000 microfibras (\cite{AnthonyBrowne2011};~\cite{CarneyAlmroth}). Los mesoplásticos ingeridos por los peces no se registraron en este estudio. De hecho, este tamaño de rango apenas se ha registrado en el tracto digestivo de los peces \parencite{JABEEN2017141}.
